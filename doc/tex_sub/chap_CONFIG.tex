\documentclass[../tex_main/NEMO_manual]{subfiles}
\begin{document}
% ================================================================
% Chapter � Configurations
% ================================================================
\chapter{Configurations}
\label{chap:CFG}
\minitoc

\newpage
$\ $\newline    % force a new ligne

% ================================================================
% Introduction
% ================================================================
\section{Introduction}
\label{sec:CFG_intro}


The purpose of this part of the manual is to introduce the \NEMO reference configurations. 
These configurations are offered as means to explore various numerical and physical options, 
thus allowing the user to verify that the code is performing in a manner consistent with that 
we are running. This form of verification is critical as one adopts the code for his or her particular 
research purposes. The reference configurations also provide a sense for some of the options available 
in the code, though by no means are all options exercised in the reference configurations.

%------------------------------------------namcfg----------------------------------------------------
\forfile{../namelists/namcfg}
%-------------------------------------------------------------------------------------------------------------

% ================================================================
% 1D model configuration
% ================================================================
\section{C1D: 1D Water column model (\protect\key{c1d}) }
\label{sec:CFG_c1d}

$\ $\newline
BE careful: to be re-written according to suppression of jpizoom and jpjzoom !!!!
$\ $\newline

The 1D model option simulates a stand alone water column within the 3D \NEMO system. 
It can be applied to the ocean alone or to the ocean-ice system and can include passive tracers 
or a biogeochemical model. It is set up by defining the position of the 1D water column in the grid 
(see \textit{CONFIG/SHARED/namelist\_ref} ). 
The 1D model is a very useful tool  
\textit{(a)} to learn about the physics and numerical treatment of vertical mixing processes ; 
\textit{(b)} to investigate suitable parameterisations of unresolved turbulence (surface wave
breaking, Langmuir circulation, ...) ; 
\textit{(c)} to compare the behaviour of different vertical mixing schemes  ; 
\textit{(d)} to perform sensitivity studies on the vertical diffusion at a particular point of an ocean domain ; 
\textit{(d)} to produce extra diagnostics, without the large memory requirement of the full 3D model.

The methodology is based on the use of the zoom functionality over the smallest possible 
domain : a 3x3 domain centered on the grid point of interest, 
with some extra routines. There is no need to define a new mesh, bathymetry, 
initial state or forcing, since the 1D model will use those of the configuration it is a zoom of. 
The chosen grid point is set in \textit{\ngn{namcfg}} namelist by setting the \np{jpizoom} and \np{jpjzoom} 
parameters to the indices of the location of the chosen grid point.

The 1D model has some specifies. First, all the horizontal derivatives are assumed to be zero, and
second, the two components of the velocity are moved on a $T$-point. 
Therefore, defining \key{c1d} changes five main things in the code behaviour: 
\begin{description}
\item[(1)] the lateral boundary condition routine (\rou{lbc\_lnk}) set the value of the central column 
of the 3x3 domain is imposed over the whole domain ; 
\item[(3)] a call to \rou{lbc\_lnk} is systematically done when reading input data ($i.e.$ in \mdl{iom}) ; 
\item[(3)] a simplified \rou{stp} routine is used (\rou{stp\_c1d}, see \mdl{step\_c1d} module) in which 
both lateral tendancy terms and lateral physics are not called ; 
\item[(4)] the vertical velocity is zero (so far, no attempt at introducing a Ekman pumping velocity 
has been made) ; 
\item[(5)] a simplified treatment of the Coriolis term is performed as $U$- and $V$-points are the same 
(see \mdl{dyncor\_c1d}).
\end{description}
All the relevant \textit{\_c1d} modules can be found in the NEMOGCM/NEMO/OPA\_SRC/C1D directory of 
the \NEMO distribution.

% to be added:  a test case on the yearlong Ocean Weather Station (OWS) Papa dataset of Martin (1985)

% ================================================================
% ORCA family configurations
% ================================================================
\section{ORCA family: global ocean with tripolar grid}
\label{sec:CFG_orca}

The ORCA family is a series of global ocean configurations that are run together with 
the LIM sea-ice model (ORCA-LIM) and possibly with PISCES biogeochemical model 
(ORCA-LIM-PISCES), using various resolutions.
An appropriate namelist is available in \path{CONFIG/ORCA2_LIM3_PISCES/EXP00/namelist_cfg} 
for ORCA2.
The domain of ORCA2 configuration is defined in \ifile{ORCA\_R2\_zps\_domcfg} file, this file is available in tar file in the wiki of NEMO : \\
https://forge.ipsl.jussieu.fr/nemo/wiki/Users/ReferenceConfigurations/ORCA2\_LIM3\_PISCES \\
In this namelist\_cfg the name of domain input file is set in \ngn{namcfg} block of namelist. 

%>>>>>>>>>>>>>>>>>>>>>>>>>>>>
\begin{figure}[!t]   \begin{center}
\includegraphics[width=0.98\textwidth]{Fig_ORCA_NH_mesh}
\caption{  \protect\label{fig:MISC_ORCA_msh}     
ORCA mesh conception. The departure from an isotropic Mercator grid start poleward of 20\degN.
The two "north pole" are the foci of a series of embedded ellipses (blue curves) 
which are determined analytically and form the i-lines of the ORCA mesh (pseudo latitudes). 
Then, following \citet{Madec_Imbard_CD96}, the normal to the series of ellipses (red curves) is computed 
which provide the j-lines of the mesh (pseudo longitudes).  }
\end{center}   \end{figure}
%>>>>>>>>>>>>>>>>>>>>>>>>>>>>

% -------------------------------------------------------------------------------------------------------------
%       ORCA tripolar grid
% -------------------------------------------------------------------------------------------------------------
\subsection{ORCA tripolar grid}
\label{subsec:CFG_orca_grid}

The ORCA grid is a tripolar is based on the semi-analytical method of \citet{Madec_Imbard_CD96}. 
It allows to construct a global orthogonal curvilinear ocean mesh which has no singularity point inside 
the computational domain since two north mesh poles are introduced and placed on lands.
The method involves defining an analytical set of mesh parallels in the stereographic polar plan, 
computing the associated set of mesh meridians, and projecting the resulting mesh onto the sphere. 
The set of mesh parallels used is a series of embedded ellipses which foci are the two mesh north 
poles (\autoref{fig:MISC_ORCA_msh}). The resulting mesh presents no loss of continuity in 
either the mesh lines or the scale factors, or even the scale factor derivatives over the whole 
ocean domain, as the mesh is not a composite mesh. 
%>>>>>>>>>>>>>>>>>>>>>>>>>>>>
\begin{figure}[!tbp]  \begin{center}
\includegraphics[width=1.0\textwidth]{Fig_ORCA_NH_msh05_e1_e2}
\includegraphics[width=0.80\textwidth]{Fig_ORCA_aniso}
\caption {  \protect\label{fig:MISC_ORCA_e1e2}
\textit{Top}: Horizontal scale factors ($e_1$, $e_2$) and 
\textit{Bottom}: ratio of anisotropy ($e_1 / e_2$)
for ORCA 0.5\deg ~mesh. South of 20\degN a Mercator grid is used ($e_1 = e_2$) 
so that the anisotropy ratio is 1. Poleward of 20\degN, the two "north pole" 
introduce a weak anisotropy over the ocean areas ($< 1.2$) except in vicinity of Victoria Island 
(Canadian Arctic Archipelago). }
\end{center}   \end{figure}
%>>>>>>>>>>>>>>>>>>>>>>>>>>>>


The method is applied to Mercator grid ($i.e.$ same zonal and meridional grid spacing) poleward 
of 20\degN, so that the Equator is a mesh line, which provides a better numerical solution 
for equatorial dynamics. The choice of the series of embedded ellipses (position of the foci and 
variation of the ellipses) is a compromise between maintaining  the ratio of mesh anisotropy 
($e_1 / e_2$) close to one in the ocean (especially in area of strong eddy activities such as 
the Gulf Stream) and keeping the smallest scale factor in the northern hemisphere larger 
than the smallest one in the southern hemisphere.
The resulting mesh is shown in \autoref{fig:MISC_ORCA_msh} and \autoref{fig:MISC_ORCA_e1e2} 
for a half a degree grid (ORCA\_R05).
The smallest ocean scale factor is found in along  Antarctica, while the ratio of anisotropy remains close to one except near the Victoria Island 
in the Canadian Archipelago. 

% -------------------------------------------------------------------------------------------------------------
%       ORCA-LIM(-PISCES) configurations
% -------------------------------------------------------------------------------------------------------------
\subsection{ORCA pre-defined resolution}
\label{subsec:CFG_orca_resolution}


The NEMO system is provided with five built-in ORCA configurations which differ in the 
horizontal resolution. The value of the resolution is given by the resolution at the Equator 
expressed in degrees. Each of configuration is set through the \textit{domain\_cfg} domain configuration file, 
which sets the grid size and configuration name parameters. The NEMO System Team provides only ORCA2 domain input file "\ifile{ORCA\_R2\_zps\_domcfg}" file  (Tab. \autoref{tab:ORCA}).




%--------------------------------------------------TABLE--------------------------------------------------
\begin{table}[!t]     \begin{center}
\begin{tabular}{p{4cm} c c c c}
Horizontal Grid         	             & \np{ORCA\_index} &  \np{jpiglo} & \np{jpjglo} &       \\  
\hline  \hline
\~4\deg	   &        4         &         92     &      76      &       \\
\~2\deg        &        2         &       182     &    149      &        \\
\~1\deg        &        1         &       362     &     292     &        \\
\~0.5\deg     &        05       &       722     &     511     &        \\
\~0.25\deg   &        025     &      1442    &   1021     &        \\
%\key{orca\_r8}       &        8         &      2882    &   2042     &        \\
%\key{orca\_r12}     &      12         &      4322    &   3062      &       \\
\hline   \hline
\end{tabular}
\caption{ \protect\label{tab:ORCA}   
Domain size of ORCA family configurations.
The flag for configurations of ORCA family need to be set in \textit{domain\_cfg} file. }
\end{center}
\end{table}
%--------------------------------------------------------------------------------------------------------------


The ORCA\_R2 configuration has the following specificity : starting from a 2\deg~ORCA mesh, 
local mesh refinements were applied to the Mediterranean, Red, Black and Caspian Seas, 
so that the resolution is 1\deg \time 1\deg there. A local transformation were also applied 
with in the Tropics in order to refine the meridional resolution up to 0.5\deg at the Equator.

The ORCA\_R1 configuration has only a local tropical transformation  to refine the meridional 
resolution up to 1/3\deg~at the Equator. Note that the tropical mesh refinements in ORCA\_R2 
and R1 strongly increases the mesh anisotropy there.

The ORCA\_R05 and higher global configurations do not incorporate any regional refinements.  

For ORCA\_R1 and R025, setting the configuration key to 75 allows to use 75 vertical levels, 
otherwise 46 are used. In the other ORCA configurations, 31 levels are used 
(see \autoref{tab:orca_zgr} \sfcomment{HERE I need to put new table for ORCA2 values} and \autoref{fig:zgr}).

Only the ORCA\_R2 is provided with all its input files in the \NEMO distribution. 
It is very similar to that used as part of the climate model developed at IPSL for the 4th IPCC 
assessment of climate change (Marti et al., 2009). It is also the basis for the \NEMO contribution 
to the Coordinate Ocean-ice Reference Experiments (COREs) documented in \citet{Griffies_al_OM09}. 

This version of ORCA\_R2 has 31 levels in the vertical, with the highest resolution (10m) 
in the upper 150m (see \autoref{tab:orca_zgr} and \autoref{fig:zgr}). 
The bottom topography and the coastlines are derived from the global atlas of Smith and Sandwell (1997). 
The default forcing uses the boundary forcing from \citet{Large_Yeager_Rep04} (see \autoref{subsec:SBC_blk_core}), 
which was developed for the purpose of running global coupled ocean-ice simulations 
without an interactive atmosphere. This \citet{Large_Yeager_Rep04} dataset is available 
through the \href{http://nomads.gfdl.noaa.gov/nomads/forms/mom4/CORE.html}{GFDL web site}. 
The "normal year" of \citet{Large_Yeager_Rep04} has been chosen of the \NEMO distribution 
since release v3.3. 

ORCA\_R2 pre-defined configuration can also be run with an AGRIF zoom over the Agulhas 
current area ( \key{agrif}  defined) and, by setting the appropriate variables, see \path{CONFIG/SHARED/namelist_ref}
a regional Arctic or peri-Antarctic configuration is extracted from an ORCA\_R2 or R05 configurations
using sponge layers at open boundaries. 

% -------------------------------------------------------------------------------------------------------------
%       GYRE family: double gyre basin
% -------------------------------------------------------------------------------------------------------------
\section{GYRE family: double gyre basin }
\label{sec:CFG_gyre}

The GYRE configuration \citep{Levy_al_OM10} has been built to simulate
the seasonal cycle of a double-gyre box model. It consists in an idealized domain 
similar to that used in the studies of \citet{Drijfhout_JPO94} and \citet{Hazeleger_Drijfhout_JPO98, 
Hazeleger_Drijfhout_JPO99, Hazeleger_Drijfhout_JGR00, Hazeleger_Drijfhout_JPO00}, 
over which an analytical seasonal forcing is applied. This allows to investigate the 
spontaneous generation of a large number of interacting, transient mesoscale eddies 
and their contribution to the large scale circulation. 

The domain geometry is a closed rectangular basin on the $\beta$-plane centred 
at $\sim$ 30\degN and rotated by 45\deg, 3180~km long, 2120~km wide 
and 4~km deep (\autoref{fig:MISC_strait_hand}). 
The domain is bounded by vertical walls and by a flat bottom. The configuration is 
meant to represent an idealized North Atlantic or North Pacific basin. 
The circulation is forced by analytical profiles of wind and buoyancy fluxes. 
The applied forcings vary seasonally in a sinusoidal manner between winter 
and summer extrema \citep{Levy_al_OM10}. 
The wind stress is zonal and its curl changes sign at 22\degN and 36\degN. 
It forces a subpolar gyre in the north, a subtropical gyre in the wider part of the domain 
and a small recirculation gyre in the southern corner. 
The net heat flux takes the form of a restoring toward a zonal apparent air 
temperature profile. A portion of the net heat flux which comes from the solar radiation
is allowed to penetrate within the water column. 
The fresh water flux is also prescribed and varies zonally. 
It is determined such as, at each time step, the basin-integrated flux is zero. 
The basin is initialised at rest with vertical profiles of temperature and salinity 
uniformly applied to the whole domain.

The GYRE configuration is set like an analytical configuration. Through \np{ln\_read\_cfg}\forcode{ = .false.} in \textit{namcfg} namelist defined in the reference configuration \path{CONFIG/GYRE/EXP00/namelist_cfg} anaylitical definition of grid in GYRE is done in usrdef\_hrg, usrdef\_zgr routines. Its horizontal resolution 
(and thus the size of the domain) is determined by setting \np{nn\_GYRE} in  \ngn{namusr\_def}: \\
\np{jpiglo} $= 30 \times$ \np{nn\_GYRE} + 2   \\
\np{jpjglo} $= 20 \times$ \np{nn\_GYRE} + 2   \\
Obviously, the namelist parameters have to be adjusted to the chosen resolution, see the Configurations 
pages on the NEMO web site (Using NEMO\/Configurations) .
In the vertical, GYRE uses the default 30 ocean levels (\jp{jpk}\forcode{ = 31}) (\autoref{fig:zgr}).

The GYRE configuration is also used in benchmark test as it is very simple to increase 
its resolution and as it does not requires any input file. For example, keeping a same model size 
on each processor while increasing the number of processor used is very easy, even though the 
physical integrity of the solution can be compromised. Benchmark is activate via \np{ln\_bench}\forcode{ = .true.} in \ngn{namusr\_def} in namelist \path{CONFIG/GYRE/EXP00/namelist_cfg}.

%>>>>>>>>>>>>>>>>>>>>>>>>>>>>
\begin{figure}[!t]   \begin{center}
\includegraphics[width=1.0\textwidth]{Fig_GYRE}
\caption{  \protect\label{fig:GYRE}   
Snapshot of relative vorticity at the surface of the model domain 
in GYRE R9, R27 and R54. From \citet{Levy_al_OM10}.}
\end{center}   \end{figure}
%>>>>>>>>>>>>>>>>>>>>>>>>>>>>

% -------------------------------------------------------------------------------------------------------------
%       AMM configuration
% -------------------------------------------------------------------------------------------------------------
\section{AMM: atlantic margin configuration}
\label{sec:MISC_config_AMM}

The AMM, Atlantic Margins Model, is a regional model covering the
Northwest European Shelf domain on a regular lat-lon grid at
approximately 12km horizontal resolution. The appropriate 
\textit{\&namcfg} namelist  is available in \textit{CONFIG/AMM12/EXP00/namelist\_cfg}.
It is used to build the correct dimensions of the AMM domain.

This configuration tests several features of NEMO functionality specific
to the shelf seas.
In particular, the AMM uses $S$-coordinates in the vertical rather than
$z$-coordinates and is forced with tidal lateral boundary conditions
using a flather boundary condition from the BDY module.
The AMM configuration  uses the GLS (\key{zdfgls}) turbulence scheme, the
VVL non-linear free surface(\key{vvl}) and time-splitting
(\key{dynspg\_ts}).

In addition to the tidal boundary condition the model may also take
open boundary conditions from a North Atlantic model. Boundaries may be
completely omitted by setting \np{ln\_bdy} to false.
Sample surface fluxes, river forcing and a sample initial restart file
are included to test a realistic model run. The Baltic boundary is
included within the river input file and is specified as a river source.
Unlike ordinary river points the Baltic inputs also include salinity and
temperature data.

\end{document}
