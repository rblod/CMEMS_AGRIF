\documentclass[../tex_main/NEMO_manual]{subfiles}
\begin{document}
% ================================================================
% Chapter � Appendix C : Discrete Invariants of the Equations
% ================================================================
\chapter{Discrete Invariants of the Equations}
\label{apdx:C}
\minitoc

%%%  Appendix put in gmcomment as it has not been updated for z* and s coordinate
%I'm writting this appendix. It will be available in a forthcoming release of the documentation

%\gmcomment{

\newpage
$\ $\newline    % force a new ligne

% ================================================================
% Introduction / Notations
% ================================================================
\section{Introduction / Notations}
\label{sec:C.0}

Notation used in this appendix in the demonstations :

fluxes at the faces of a $T$-box:
\begin{equation*}
U = e_{2u}\,e_{3u}\; u  \qquad  V = e_{1v}\,e_{3v}\; v  \qquad W = e_{1w}\,e_{2w}\; \omega     \\
\end{equation*}

volume of cells at $u$-, $v$-, and $T$-points:
\begin{equation*}
b_u = e_{1u}\,e_{2u}\,e_{3u}  \qquad  b_v = e_{1v}\,e_{2v}\,e_{3v}  \qquad b_t = e_{1t}\,e_{2t}\,e_{3t}     \\
\end{equation*}

partial derivative notation: $\partial_\bullet = \frac{\partial}{\partial \bullet}$

$dv=e_1\,e_2\,e_3 \,di\,dj\,dk$  is the volume element, with only $e_3$ that depends on time.
$D$ and $S$ are the ocean domain volume and surface, respectively. 
No wetting/drying is allow ($i.e.$ $\frac{\partial S}{\partial t} = 0$) 
Let $k_s$ and $k_b$ be the ocean surface and bottom, resp. 
($i.e.$ $s(k_s) = \eta$ and $s(k_b)=-H$, where $H$ is the bottom depth).
\begin{flalign*}
 z(k) = \eta - \int\limits_{\tilde{k}=k}^{\tilde{k}=k_s}  e_3(\tilde{k}) \;d\tilde{k}
        = \eta - \int\limits_k^{k_s}  e_3 \;d\tilde{k} 
\end{flalign*}

Continuity equation with the above notation:
\begin{equation*}
\frac{1}{e_{3t}} \partial_t (e_{3t})+ \frac{1}{b_t}  \biggl\{  \delta_i [U] + \delta_j [V] + \delta_k [W] \biggr\} = 0
\end{equation*}

A quantity, $Q$ is conserved when its domain averaged time change is zero, that is when:
\begin{equation*}
\partial_t \left( \int_D{ Q\;dv } \right) =0
\end{equation*}
Noting that the coordinate system used ....  blah blah
\begin{equation*}
\partial_t \left( \int_D {Q\;dv} \right) =  \int_D { \partial_t \left( e_3 \, Q \right) e_1e_2\;di\,dj\,dk }
                                                       =  \int_D { \frac{1}{e_3} \partial_t \left( e_3 \, Q \right) dv } =0
\end{equation*}
equation of evolution of $Q$ written as the time evolution of the vertical content of $Q$
like for tracers, or momentum in flux form, the quadratic quantity $\frac{1}{2}Q^2$ is conserved when :
\begin{flalign*}
\partial_t \left(   \int_D{ \frac{1}{2} \,Q^2\;dv }   \right)
=&  \int_D{ \frac{1}{2} \partial_t \left( \frac{1}{e_3}\left( e_3 \, Q \right)^2 \right) e_1e_2\;di\,dj\,dk } \\
=&  \int_D {         Q   \;\partial_t\left( e_3 \, Q \right) e_1e_2\;di\,dj\,dk }  
-  \int_D { \frac{1}{2} Q^2 \,\partial_t  (e_3) \;e_1e_2\;di\,dj\,dk } \\
\end{flalign*}
that is in a more compact form :
\begin{flalign} \label{eq:Q2_flux}
\partial_t \left( \int_D {\frac{1}{2} Q^2\;dv} \right)
=&                   \int_D { \frac{Q}{e_3}  \partial_t \left( e_3 \, Q \right) dv }  
  -   \frac{1}{2} \int_D {  \frac{Q^2}{e_3} \partial_t (e_3) \;dv }
\end{flalign}
equation of evolution of $Q$ written as the time evolution of $Q$
like for momentum in vector invariant form, the quadratic quantity $\frac{1}{2}Q^2$ is conserved when :
\begin{flalign*}
\partial_t \left( \int_D {\frac{1}{2} Q^2\;dv} \right)
=&  \int_D { \frac{1}{2} \partial_t \left( e_3 \, Q^2 \right) \;e_1e_2\;di\,dj\,dk } \\
=& \int_D {         Q      \partial_t Q  \;e_1e_2e_3\;di\,dj\,dk }  
+  \int_D { \frac{1}{2} Q^2 \, \partial_t e_3  \;e_1e_2\;di\,dj\,dk } \\
\end{flalign*}
that is in a more compact form :
\begin{flalign} \label{eq:Q2_vect}
\partial_t \left( \int_D {\frac{1}{2} Q^2\;dv} \right)
=& \int_D {         Q   \,\partial_t Q  \;dv }  
+   \frac{1}{2} \int_D { \frac{1}{e_3} Q^2 \partial_t e_3 \;dv }
\end{flalign}


% ================================================================
% Continuous Total energy Conservation
% ================================================================
\section{Continuous conservation}
\label{sec:C.1}


The discretization of pimitive equation in $s$-coordinate ($i.e.$ time and space varying 
vertical coordinate) must be chosen so that the discrete equation of the model satisfy 
integral constrains on energy and enstrophy. 


Let us first establish those constraint in the continuous world.
The total energy ($i.e.$ kinetic plus potential energies) is conserved :
\begin{flalign} \label{eq:Tot_Energy}
  \partial_t \left( \int_D \left( \frac{1}{2} {\textbf{U}_h}^2 +  \rho \, g \, z \right) \;dv \right)  = & 0
\end{flalign}
under the following assumptions: no dissipation, no forcing 
(wind, buoyancy flux, atmospheric pressure variations), mass 
conservation, and closed domain. 

This equation can be transformed to obtain several sub-equalities. 
The transformation for the advection term depends on whether 
the vector invariant form or the flux form is used for the momentum equation.
Using \autoref{eq:Q2_vect} and introducing \autoref{apdx:A_dyn_vect} in \autoref{eq:Tot_Energy} 
for the former form and
Using \autoref{eq:Q2_flux} and introducing \autoref{apdx:A_dyn_flux} in \autoref{eq:Tot_Energy} 
for the latter form  leads to:

\begin{subequations} \label{eq:E_tot}

advection term (vector invariant form):
\begin{equation} \label{eq:E_tot_vect_vor_1}
\int\limits_D  \zeta \; \left( \textbf{k} \times \textbf{U}_h  \right) \cdot \textbf{U}_h  \;  dv   = 0   \\
\end{equation}
%
\begin{equation} \label{eq:E_tot_vect_adv_1}
   \int\limits_D  \textbf{U}_h \cdot \nabla_h \left( \frac{{\textbf{U}_h}^2}{2} \right)     dv 
+ \int\limits_D  \textbf{U}_h \cdot \nabla_z \textbf{U}_h  \;dv    
-  \int\limits_D { \frac{{\textbf{U}_h}^2}{2} \frac{1}{e_3} \partial_t e_3 \;dv }   = 0   \\
\end{equation}

advection term (flux form):
\begin{equation} \label{eq:E_tot_flux_metric}
\int\limits_D  \frac{1} {e_1 e_2 } \left( v \,\partial_i e_2 - u \,\partial_j e_1  \right)\; 
 \left(  \textbf{k} \times \textbf{U}_h  \right) \cdot \textbf{U}_h  \;  dv   = 0   \\
\end{equation}

\begin{equation} \label{eq:E_tot_flux_adv}
   \int\limits_D \textbf{U}_h \cdot     \left(                 {{\begin{array} {*{20}c}
\nabla \cdot \left( \textbf{U}\,u \right) \hfill \\
\nabla \cdot \left( \textbf{U}\,v \right) \hfill \\       \end{array}} }           \right)   \;dv 
+   \frac{1}{2} \int\limits_D {  {\textbf{U}_h}^2 \frac{1}{e_3} \partial_t e_3 \;dv } =\;0  \\
\end{equation}

coriolis term
\begin{equation} \label{eq:E_tot_cor}
\int\limits_D  f   \; \left( \textbf{k} \times \textbf{U}_h  \right) \cdot \textbf{U}_h  \;  dv   = 0   \\
\end{equation}

pressure gradient:
\begin{equation} \label{eq:E_tot_pg_1}
   - \int\limits_D  \left. \nabla p \right|_z \cdot \textbf{U}_h \;dv 
= - \int\limits_D \nabla \cdot \left( \rho \,\textbf {U} \right)\;g\;z\;\;dv
   + \int\limits_D g\, \rho \; \partial_t z  \;dv   \\
\end{equation}
\end{subequations}

where $\nabla_h = \left. \nabla \right|_k$ is the gradient along the $s$-surfaces.

blah blah....
$\ $\newline    % force a new ligne
The prognostic ocean dynamics equation can be summarized as follows:
\begin{equation*}
\text{NXT} = \dbinom	{\text{VOR} + \text{KEG} + \text {ZAD} }
						{\text{COR} + \text{ADV}                       }
			+ \text{HPG} + \text{SPG} + \text{LDF} + \text{ZDF}
\end{equation*}
$\ $\newline    % force a new ligne

Vector invariant form:
\begin{subequations} \label{eq:E_tot_vect}
\begin{equation} \label{eq:E_tot_vect_vor_2}
\int\limits_D   \textbf{U}_h \cdot \text{VOR} \;dv   = 0   \\
\end{equation}
\begin{equation} \label{eq:E_tot_vect_adv_2}
   \int\limits_D  \textbf{U}_h \cdot \text{KEG}  \;dv 
+ \int\limits_D  \textbf{U}_h \cdot \text{ZAD}  \;dv    
-  \int\limits_D { \frac{{\textbf{U}_h}^2}{2} \frac{1}{e_3} \partial_t e_3 \;dv }   = 0   \\
\end{equation}
\begin{equation} \label{eq:E_tot_pg_2}
   - \int\limits_D  \textbf{U}_h \cdot (\text{HPG}+ \text{SPG}) \;dv 
= - \int\limits_D \nabla \cdot \left( \rho \,\textbf {U} \right)\;g\;z\;\;dv
   + \int\limits_D g\, \rho \; \partial_t z  \;dv   \\
\end{equation}
\end{subequations}

Flux form:
\begin{subequations} \label{eq:E_tot_flux}
\begin{equation} \label{eq:E_tot_flux_metric_2}
\int\limits_D  \textbf{U}_h \cdot \text {COR} \;  dv   = 0   \\
\end{equation}
\begin{equation} \label{eq:E_tot_flux_adv_2}
   \int\limits_D \textbf{U}_h \cdot \text{ADV}   \;dv 
+   \frac{1}{2} \int\limits_D {  {\textbf{U}_h}^2 \frac{1}{e_3} \partial_t e_3  \;dv } =\;0  \\
\end{equation}
\begin{equation} \label{eq:E_tot_pg_3}
   - \int\limits_D  \textbf{U}_h \cdot (\text{HPG}+ \text{SPG}) \;dv 
= - \int\limits_D \nabla \cdot \left( \rho \,\textbf {U} \right)\;g\;z\;\;dv
   + \int\limits_D g\, \rho \; \partial_t  z  \;dv   \\
\end{equation}
\end{subequations}


$\ $\newline    % force a new ligne


\autoref{eq:E_tot_pg_3} is the balance between the conversion KE to PE and PE to KE. 
Indeed the left hand side of \autoref{eq:E_tot_pg_3} can be transformed as follows:
\begin{flalign*}
\partial_t  \left( \int\limits_D { \rho \, g \, z  \;dv} \right) 
&= + \int\limits_D \frac{1}{e_3} \partial_t (e_3\,\rho) \;g\;z\;\;dv
     +  \int\limits_D g\, \rho \; \partial_t z  \;dv   &&&\\
&= - \int\limits_D \nabla \cdot \left( \rho \,\textbf {U} \right)\;g\;z\;\;dv
     + \int\limits_D g\, \rho \; \partial_t z \;dv   &&&\\
&= + \int\limits_D  \rho \,g \left( \textbf {U}_h \cdot \nabla_h z + \omega \frac{1}{e_3} \partial_k z \right)  \;dv
     + \int\limits_D g\, \rho \; \partial_t z \;dv   &&&\\
&= + \int\limits_D  \rho \,g \left( \omega + \partial_t z + \textbf {U}_h \cdot \nabla_h z  \right)  \;dv  &&&\\
&=+  \int\limits_D g\, \rho \; w \; dv   &&&\\
\end{flalign*}
where the last equality is obtained by noting that the brackets is exactly the expression of $w$, 
the vertical velocity referenced to the fixe $z$-coordinate system (see \autoref{apdx:A_w_s}). 
 
The left hand side of \autoref{eq:E_tot_pg_3} can be transformed as follows:
\begin{flalign*}
- \int\limits_D  \left. \nabla p \right|_z & \cdot \textbf{U}_h \;dv  
= - \int\limits_D  \left( \nabla_h p + \rho \, g \nabla_h z \right) \cdot \textbf{U}_h \;dv   &&&\\
\allowdisplaybreaks
&= - \int\limits_D  \nabla_h  p \cdot \textbf{U}_h \;dv   - \int\limits_D  \rho \, g \, \nabla_h z \cdot \textbf{U}_h \;dv   &&&\\
\allowdisplaybreaks
&= +\int\limits_D p \,\nabla_h \cdot \textbf{U}_h \;dv   + \int\limits_D  \rho \, g \left( \omega - w + \partial_t z \right) \;dv   &&&\\
\allowdisplaybreaks
&= -\int\limits_D p \left( \frac{1}{e_3} \partial_t e_3 + \frac{1}{e_3} \partial_k \omega  \right) \;dv
    +\int\limits_D  \rho \, g \left( \omega - w + \partial_t z \right) \;dv   &&&\\
\allowdisplaybreaks
&= -\int\limits_D \frac{p}{e_3} \partial_t e_3  \;dv    
     +\int\limits_D \frac{1}{e_3} \partial_k p\; \omega \;dv
    +\int\limits_D  \rho \, g \left( \omega - w + \partial_t z \right) \;dv   &&&\\
&= -\int\limits_D \frac{p}{e_3} \partial_t e_3  \;dv    
     -\int\limits_D \rho \, g \, \omega \;dv
    +\int\limits_D  \rho \, g \left( \omega - w + \partial_t z \right) \;dv   &&&\\
&= - \int\limits_D \frac{p}{e_3} \partial_t e_3 \; \;dv    
     - \int\limits_D  \rho \, g \, w \;dv 
     + \int\limits_D   \rho \, g \, \partial_t z \;dv   &&&\\
\allowdisplaybreaks
\intertext{introducing the hydrostatic balance $\partial_k p=-\rho \,g\,e_3$ in the last term,
it becomes:}
&= - \int\limits_D \frac{p}{e_3} \partial_t e_3 \;dv    
     - \int\limits_D  \rho \, g \, w \;dv 
     - \int\limits_D  \frac{1}{e_3} \partial_k p\, \partial_t z \;dv   &&&\\
&= - \int\limits_D \frac{p}{e_3} \partial_t e_3 \;dv    
     - \int\limits_D  \rho \, g \, w \;dv 
     + \int\limits_D \,\frac{p}{e_3}\partial_t ( \partial_k z )  dv   &&&\\
% 
&= - \int\limits_D  \rho \, g \, w \;dv   &&&\\
\end{flalign*}


%gm comment
\gmcomment{
%
The last equality comes from the following equation,
\begin{flalign*}
\int\limits_D p \frac{1}{e_3} \frac{\partial e_3}{\partial t}\; \;dv    
     = \int\limits_D   \rho \, g \, \frac{\partial z }{\partial t} \;dv \quad,  \\ 
\end{flalign*}
that can be demonstrated as follows:

\begin{flalign*}
\int\limits_D   \rho \, g \, \frac{\partial z }{\partial t} \;dv   
&= \int\limits_D    \rho \, g \, \frac{\partial \eta}{\partial t} \;dv   
  -  \int\limits_D    \rho \, g \, \frac{\partial}{\partial t} \left(  \int\limits_k^{k_s}  e_3 \;d\tilde{k} \right) \;dv   &&&\\
&= \int\limits_D    \rho \, g \, \frac{\partial \eta}{\partial t} \;dv   
  -  \int\limits_D    \rho \, g    \left(  \int\limits_k^{k_s}  \frac{\partial e_3}{\partial t} \;d\tilde{k} \right) \;dv   &&&\\
%
\allowdisplaybreaks
\intertext{The second term of the right hand side can be transformed by applying the integration by part rule: 
$\left[ a\,b \right]_{k_b}^{k_s} = \int_{k_b}^{k_s}  a\,\frac{\partial b}{\partial k}       \;dk
                                               + \int_{k_b}^{k_s}      \frac{\partial a}{\partial k} \,b \;dk $
to the following function: $a=  \int_k^{k_s}  \frac{\partial e_3}{\partial t} \;d\tilde{k}$ 
and $b=  \int_k^{k_s}  \rho \, e_3 \;d\tilde{k}$
(note that $\frac{\partial}{\partial k} \left(  \int_k^{k_s}  a \;d\tilde{k}  \right) = - a$ as $k$ is the lower bound of the integral).
This leads to:  }
\end{flalign*}
\begin{flalign*}
&\left[ \int\limits_{k}^{k_s}  \frac{\partial e_3}{\partial t} \,dk \cdot \int\limits_{k}^{k_s}  \rho \, e_3 \,dk   \right]_{k_b}^{k_s}
=-\int\limits_{k_b}^{k_s} \left(  \int\limits_k^{k_s}  \frac{\partial e_3}{\partial t} \;d\tilde{k} \right)  \rho \,e_3 \;dk
  -\int\limits_{k_b}^{k_s}  \frac{\partial e_3}{\partial t}  \left(  \int\limits_k^{k_s}  \rho \, e_3 \;d\tilde{k} \right)   dk
&&&\\
\allowdisplaybreaks
\intertext{Noting that $\frac{\partial \eta}{\partial t} 
		= \frac{\partial}{\partial t}  \left( \int_{k_b}^{k_s}   e_3  \;d\tilde{k}  \right)
		= \int_{k_b}^{k_s}  \frac{\partial e_3}{\partial t} \;d\tilde{k}$
and 
		$p(k) = \int_k^{k_s}  \rho \,g \, e_3 \;d\tilde{k} $, 
but also that $\frac{\partial \eta}{\partial t}$ does not depends on $k$, it comes:
}
& - \int\limits_{k_b}^{k_s}  \rho \, \frac{\partial \eta}{\partial t} \, e_3 \;dk
= - \int\limits_{k_b}^{k_s} \left(  \int\limits_k^{k_s}  \frac{\partial e_3}{\partial t} \;d\tilde{k} \right)   \, \rho \, g   e_3\;dk
   - \int\limits_{k_b}^{k_s}  \frac{\partial e_3}{\partial t} \frac{p}{g}         \;dk       &&&\\
\end{flalign*}
Mutliplying by $g$ and integrating over the $(i,j)$ domain it becomes:
\begin{flalign*}
\int\limits_D  \rho \, g \, \left(  \int\limits_k^{k_s}  \frac{\partial e_3}{\partial t} \;d\tilde{k} \right)    \;dv
=  \int\limits_D  \rho \, g \, \frac{\partial \eta}{\partial t} dv
  - \int\limits_D  \frac{p}{e_3}\frac{\partial e_3}{\partial t}         \;dv
\end{flalign*}
Using this property, we therefore have:
\begin{flalign*}
\int\limits_D   \rho \, g \, \frac{\partial z }{\partial t} \;dv   
&= \int\limits_D    \rho \, g \, \frac{\partial \eta}{\partial t}   \;dv   
  - \left(  \int\limits_D  \rho \, g \, \frac{\partial \eta}{\partial t} dv
           - \int\limits_D  \frac{p}{e_3}\frac{\partial e_3}{\partial t}   \;dv  \right)    &&&\\
%
&=\int\limits_D \frac{p}{e_3} \frac{\partial (e_3\,\rho)}{\partial t}\; \;dv     
\end{flalign*}
% end gm comment
}
%


% ================================================================
% Discrete Total energy Conservation : vector invariant form
% ================================================================
\section{Discrete total energy conservation: vector invariant form}
\label{sec:C.2}

% -------------------------------------------------------------------------------------------------------------
%       Total energy conservation
% -------------------------------------------------------------------------------------------------------------
\subsection{Total energy conservation}
\label{subsec:C_KE+PE_vect}

The discrete form of the total energy conservation, \autoref{eq:Tot_Energy}, is given by:
\begin{flalign*}
\partial_t  \left(  \sum\limits_{i,j,k} \biggl\{ \frac{u^2}{2} \,b_u + \frac{v^2}{2}\, b_v +  \rho \, g \, z_t \,b_t  \biggr\} \right) &=0  \\
\end{flalign*}
which in vector invariant forms, it leads to:
\begin{equation} \label{eq:KE+PE_vect_discrete}   \begin{split}
                        \sum\limits_{i,j,k} \biggl\{   u\,                        \partial_t u         \;b_u 
                                                              + v\,                        \partial_t v          \;b_v  \biggr\}
  + \frac{1}{2} \sum\limits_{i,j,k} \biggl\{  \frac{u^2}{e_{3u}}\partial_t e_{3u} \;b_u 
                                                             +  \frac{v^2}{e_{3v}}\partial_t e_{3v} \;b_v   \biggr\}      \\
= - \sum\limits_{i,j,k} \biggl\{ \frac{1}{e_{3t}}\partial_t (e_{3t} \rho) \, g \, z_t \;b_t  \biggr\} 
     - \sum\limits_{i,j,k} \biggl\{ \rho \,g\,\partial_t (z_t) \,b_t  \biggr\}                                
\end{split} \end{equation}

Substituting the discrete expression of the time derivative of the velocity either in vector invariant,
leads to the discrete equivalent of the four equations \autoref{eq:E_tot_flux}. 

% -------------------------------------------------------------------------------------------------------------
%       Vorticity term (coriolis + vorticity part of the advection)
% -------------------------------------------------------------------------------------------------------------
\subsection{Vorticity term (coriolis + vorticity part of the advection)}
\label{subsec:C_vor}

Let $q$, located at $f$-points, be either the relative ($q=\zeta / e_{3f}$), or  
the planetary ($q=f/e_{3f}$), or the total potential vorticity ($q=(\zeta +f) /e_{3f}$). 
Two discretisation of the vorticity term (ENE and EEN) allows the conservation of 
the kinetic energy.
% -------------------------------------------------------------------------------------------------------------
%       Vorticity Term with ENE scheme
% -------------------------------------------------------------------------------------------------------------
\subsubsection{Vorticity term with ENE scheme (\protect\np{ln\_dynvor\_ene}\forcode{ = .true.})}
\label{subsec:C_vorENE} 

For the ENE scheme, the two components of the vorticity term are given by :
\begin{equation*}
- e_3 \, q \;{\textbf{k}}\times {\textbf {U}}_h    \equiv 
	\left( {{  \begin{array} {*{20}c}
		+ \frac{1} {e_{1u}} \; 
		\overline {\, q \ \overline {\left( e_{1v}\,e_{3v}\,v \right)}^{\,i+1/2}} ^{\,j}        \hfill \\
		- \frac{1} {e_{2v}} \; 
		\overline {\, q \ \overline {\left( e_{2u}\,e_{3u}\,u \right)}^{\,j+1/2}} ^{\,i}       \hfill \\
	\end{array}} }    \right)
\end{equation*}

This formulation does not conserve the enstrophy but it does conserve the 
total kinetic energy. Indeed, the kinetic energy tendency associated to the 
vorticity term and averaged over the ocean domain can be transformed as
follows:
\begin{flalign*}
&\int\limits_D -  \left(  e_3 \, q \;\textbf{k} \times \textbf{U}_h  \right) \cdot \textbf{U}_h  \;  dv &&  \\
& \qquad \qquad {\begin{array}{*{20}l} 
&\equiv \sum\limits_{i,j,k} 	\biggl\{ 	
	  \frac{1} {e_{1u}} \overline { \,q\ \overline{ V }^{\,i+1/2}} ^{\,j} \, u \; b_u 
	- \frac{1} {e_{2v}}\overline { \, q\ \overline{ U }^{\,j+1/2}} ^{\,i} \, v \; b_v \; \biggr\}    \\ 
&\equiv  \sum\limits_{i,j,k} 	\biggl\{ 	
	  \overline { \,q\ \overline{ V }^{\,i+1/2}}^{\,j} \; U 
	- \overline { \,q\ \overline{ U }^{\,j+1/2}}^{\,i} \; V  \; \biggr\}     \\
&\equiv \sum\limits_{i,j,k} q \  \biggl\{  \overline{ V }^{\,i+1/2}\; \overline{ U }^{\,j+1/2}        
		                                        - \overline{ U }^{\,j+1/2}\; \overline{ V }^{\,i+1/2}         \biggr\}  \quad  \equiv 0
\end{array} } 		
\end{flalign*}
In other words, the domain averaged kinetic energy does not change due to the vorticity term.


% -------------------------------------------------------------------------------------------------------------
%       Vorticity Term with EEN scheme
% -------------------------------------------------------------------------------------------------------------
\subsubsection{Vorticity term with EEN scheme (\protect\np{ln\_dynvor\_een}\forcode{ = .true.})}
\label{subsec:C_vorEEN_vect} 

With the EEN scheme, the vorticity terms are represented as: 
\begin{equation} \tag{\ref{eq:dynvor_een}}
\left\{ {    \begin{aligned}
 +q\,e_3 \, v 	&\equiv +\frac{1}{e_{1u} }   \sum_{\substack{i_p,\,k_p}} 
                         {^{i+1/2-i_p}_j}  \mathbb{Q}^{i_p}_{j_p}  \left( e_{1v} e_{3v} \ v  \right)^{i+i_p-1/2}_{j+j_p}   \\
 - q\,e_3 \, u     &\equiv -\frac{1}{e_{2v} }    \sum_{\substack{i_p,\,k_p}} 
                         {^i_{j+1/2-j_p}}  \mathbb{Q}^{i_p}_{j_p}  \left( e_{2u} e_{3u} \ u  \right)^{i+i_p}_{j+j_p-1/2}   \\
\end{aligned}   } \right.
\end{equation} 
where the indices $i_p$ and $j_p$ take the following value: 
$i_p = -1/2$ or $1/2$ and $j_p = -1/2$ or $1/2$,
and the vorticity triads, ${^i_j}\mathbb{Q}^{i_p}_{j_p}$, defined at $T$-point, are given by: 
\begin{equation} \tag{\ref{eq:Q_triads}}
_i^j \mathbb{Q}^{i_p}_{j_p}
= \frac{1}{12} \ \left(   q^{i-i_p}_{j+j_p} + q^{i+j_p}_{j+i_p} + q^{i+i_p}_{j-j_p}  \right)
\end{equation}

This formulation does conserve the total kinetic energy. Indeed,
\begin{flalign*}
&\int\limits_D - \textbf{U}_h \cdot   \left(  \zeta \;\textbf{k} \times \textbf{U}_h  \right)  \;  dv &&  \\
\equiv \sum\limits_{i,j,k} &	\biggl\{ 
	    \left[  \sum_{\substack{i_p,\,k_p}} 
                         {^{i+1/2-i_p}_j}\mathbb{Q}^{i_p}_{j_p} \; V^{i+1/2-i_p}_{j+j_p} \right] U^{i+1/2}_{j}    %   &&\\
	  - \left[  \sum_{\substack{i_p,\,k_p}} 
                         {^i_{j+1/2-j_p}}\mathbb{Q}^{i_p}_{j_p} \; U^{i+i_p}_{j+1/2-j_p}  \right] V^{i}_{j+1/2}    \biggr\}     && \\
\\
\equiv \sum\limits_{i,j,k} &  \sum_{\substack{i_p,\,k_p}} \biggl\{  \ \  
                         {^{i+1/2-i_p}_j}\mathbb{Q}^{i_p}_{j_p} \; V^{i+1/2-i_p}_{j+j_p}  \, U^{i+1/2}_{j}     %  &&\\
                       - {^i_{j+1/2-j_p}}\mathbb{Q}^{i_p}_{j_p} \; U^{i+i_p}_{j+1/2-j_p} \, V^{i}_{j+1/2}     \ \;     \biggr\}     &&  \\
%
\allowdisplaybreaks
\intertext{ Expending the summation on $i_p$ and $k_p$, it becomes:}
%
\equiv \sum\limits_{i,j,k} & \biggl\{  \ \  
                    {^{i+1}_j     }\mathbb{Q}^{-1/2}_{+1/2} \;V^{i+1}_{j+1/2} \; U^{\,i+1/2}_{j}      
                -  {^i_{j}\quad}\mathbb{Q}^{-1/2}_{+1/2} \; U^{i-1/2}_{j}    \; V^{\,i}_{j+1/2}         &&  \\
        &       + {^{i+1}_j     }\mathbb{Q}^{-1/2}_{-1/2} \; V^{i+1}_{j-1/2} \; U^{\,i+1/2}_{j}        
                  - {^i_{j+1}     }\mathbb{Q}^{-1/2}_{-1/2} \; U^{i-1/2}_{j+1} \; V^{\,i}_{j+1/2}        \biggr.     &&  \\
        &       + {^{i}_j\quad}\mathbb{Q}^{+1/2}_{+1/2} \; V^{i}_{j+1/2}   \; U^{\,i+1/2}_{j}      
                  - {^i_{j}\quad}\mathbb{Q}^{+1/2}_{+1/2} \; U^{i+1/2}_{j}   \; V^{\,i}_{j+1/2}          \biggr.        &&  \\
        &       + {^{i}_j\quad}\mathbb{Q}^{+1/2}_{-1/2} \; V^{i}_{j-1/2}     \; U^{\,i+1/2}_{j}       
                 -  {^i_{j+1}     }\mathbb{Q}^{+1/2}_{-1/2} \; U^{i+1/2}_{j+1}\; V^{\,i}_{j+1/2}  \ \;     \biggr\}     &&  \\
%
\allowdisplaybreaks
\intertext{The summation is done over all $i$ and $j$ indices, it is therefore possible to introduce 
a shift of $-1$ either in $i$ or $j$ direction in some of the term of the summation (first term of the 
first and second lines, second term of the second and fourth lines). By doning so, we can regroup 
all the terms of the summation by triad at a ($i$,$j$) point. In other words, we regroup all the terms 
in the neighbourhood  that contain a triad at the same ($i$,$j$) indices. It becomes: }
\allowdisplaybreaks
%
\equiv \sum\limits_{i,j,k} & \biggl\{  \ \  
             {^{i}_j}\mathbb{Q}^{-1/2}_{+1/2}  \left[  V^{i}_{j+1/2}\, U^{\,i-1/2}_{j}    
                                                                       -  U^{i-1/2}_{j} \, V^{\,i}_{j+1/2}      \right]    &&  \\
 &       + {^{i}_j}\mathbb{Q}^{-1/2}_{-1/2}  \left[  V^{i}_{j-1/2} \, U^{\,i-1/2}_{j}        
                                                                     -    U^{i-1/2}_{j} \, V^{\,i}_{j-1/2}      \right]    \biggr.   &&  \\
 &      + {^{i}_j}\mathbb{Q}^{+1/2}_{+1/2}  \left[  V^{i}_{j+1/2} \, U^{\,i+1/2}_{j}      
                                                                     -    U^{i+1/2}_{j} \, V^{\,i}_{j+1/2}     \right]  \biggr.  &&  \\
 &     + {^{i}_j}\mathbb{Q}^{+1/2}_{-1/2}  \left[   V^{i}_{j-1/2} \, U^{\,i+1/2}_{j}                                                                     
                                                                    -    U^{i+1/2}_{j-1} \, V^{\,i}_{j-1/2}  \right]  \ \;   \biggr\}   \qquad
\equiv \ 0   &&
\end{flalign*}


% -------------------------------------------------------------------------------------------------------------
%       Gradient of Kinetic Energy / Vertical Advection
% -------------------------------------------------------------------------------------------------------------
\subsubsection{Gradient of kinetic energy / Vertical advection}
\label{subsec:C_zad} 

The change of Kinetic Energy (KE) due to the vertical advection is exactly 
balanced by the change of KE due to the horizontal gradient of KE~:
\begin{equation*}
    \int_D \textbf{U}_h \cdot \frac{1}{e_3 } \omega \partial_k \textbf{U}_h \;dv
=  -   \int_D \textbf{U}_h \cdot \nabla_h \left( \frac{1}{2}\;{\textbf{U}_h}^2 \right)\;dv
   +   \frac{1}{2} \int_D {  \frac{{\textbf{U}_h}^2}{e_3} \partial_t ( e_3) \;dv }  \\
\end{equation*}
Indeed, using successively \autoref{eq:DOM_di_adj} ($i.e.$ the skew symmetry 
property of the $\delta$ operator) and the continuity equation, then 
\autoref{eq:DOM_di_adj} again, then the commutativity of operators 
$\overline {\,\cdot \,}$ and $\delta$, and finally \autoref{eq:DOM_mi_adj} 
($i.e.$ the symmetry property of the $\overline {\,\cdot \,}$ operator) 
applied in the horizontal and vertical directions, it becomes:
\begin{flalign*}
& - \int_D \textbf{U}_h \cdot \text{KEG}\;dv 	
= - \int_D \textbf{U}_h \cdot \nabla_h \left( \frac{1}{2}\;{\textbf{U}_h}^2 \right)\;dv    &&&\\
%
\equiv  & -  \sum\limits_{i,j,k} \frac{1}{2}  \biggl\{	
	\frac{1} {e_{1u}}  \delta_{i+1/2}   \left[   \overline {u^2}^{\,i} + \overline {v^2}^{\,j}   \right]  u \ b_u 
     + \frac{1} {e_{2v}}  \delta_{j+1/2}   \left[   \overline {u^2}^{\,i} + \overline {v^2}^{\,j}   \right]  v \ b_v   \biggr\}  	&&&  \\ 
%
\equiv & + \sum\limits_{i,j,k} \frac{1}{2}  \left(   \overline {u^2}^{\,i} + \overline {v^2}^{\,j}   \right)\;
                                                              \biggl\{ \delta_{i} \left[  U   \right] +  \delta_{j} \left[  V  \right]    \biggr\}       &&&  \\ 
\allowdisplaybreaks
%
\equiv   & - \sum\limits_{i,j,k}  \frac{1}{2}
	\left(       \overline {u^2}^{\,i} + \overline {v^2}^{\,j}   \right)  \;  
	\biggl\{   \frac{b_t}{e_{3t}} \partial_t (e_{3t})  +  \delta_k \left[  W   \right]    \biggr\}    &&&\\
\allowdisplaybreaks
%
\equiv & +  \sum\limits_{i,j,k} \frac{1}{2} \delta_{k+1/2}   \left[ \overline{ u^2}^{\,i} + \overline{ v^2}^{\,j}   \right] \;  W 
	       -  \sum\limits_{i,j,k} \frac{1}{2} \left(   \overline {u^2}^{\,i} + \overline {v^2}^{\,j}   \right) \;\partial_t b_t   &&& \\
\allowdisplaybreaks
%
\equiv   & + \sum\limits_{i,j,k} \frac{1} {2} \left(    \overline{\delta_{k+1/2} \left[ u^2 \right]}^{\,i} 
	                                                                + \overline{\delta_{k+1/2} \left[ v^2 \right]}^{\,j}    \right) \; W     
	       -  \sum\limits_{i,j,k}  \left(  \frac{u^2}{2}\,\partial_t \overline{b_t}^{\,{i+1/2}}
	                                              + \frac{v^2}{2}\,\partial_t \overline{b_t}^{\,{j+1/2}}   \right)    &&& \\
\allowdisplaybreaks
\intertext{Assuming that $b_u= \overline{b_t}^{\,i+1/2}$ and $b_v= \overline{b_t}^{\,j+1/2}$, or at least that the time
derivative of these two equations is satisfied, it becomes:}
%
\equiv &     \sum\limits_{i,j,k} \frac{1} {2}
	\biggl\{ \; \overline{W}^{\,i+1/2}\;\delta_{k+1/2} \left[ u^2 \right]    
	            + \overline{W}^{\,j+1/2}\;\delta_{k+1/2} \left[ v^2 \right]  \;  \biggr\}   
	       -  \sum\limits_{i,j,k}  \left(  \frac{u^2}{2}\,\partial_t b_u
	                                             + \frac{v^2}{2}\,\partial_t b_v   \right)    &&& \\
\allowdisplaybreaks
%	
\equiv &     \sum\limits_{i,j,k} 
	\biggl\{ \; \overline{W}^{\,i+1/2}\; \overline {u}^{\,k+1/2}\; \delta_{k+1/2}[ u ]    
	            + \overline{W}^{\,j+1/2}\; \overline {v}^{\,k+1/2}\; \delta_{k+1/2}[ v ]  \;  \biggr\} 
	       -  \sum\limits_{i,j,k}  \left(  \frac{u^2}{2}\,\partial_t b_u
	                                             + \frac{v^2}{2}\,\partial_t b_v   \right)    &&& \\
%
\allowdisplaybreaks
\equiv  &  \sum\limits_{i,j,k}  
	\biggl\{ \; \frac{1} {b_u } \; \overline { \overline{W}^{\,i+1/2}\,\delta_{k+1/2}  \left[ u \right] }^{\,k} \;u\;b_u  
                    + \frac{1} {b_v } \; \overline { \overline{W}^{\,j+1/2} \delta_{k+1/2}  \left[ v \right]  }^{\,k} \;v\;b_v  \; \biggr\}  
	       -  \sum\limits_{i,j,k}  \left(  \frac{u^2}{2}\,\partial_t b_u
	                                             + \frac{v^2}{2}\,\partial_t b_v   \right)    &&& \\
%
\intertext{The first term provides the discrete expression for the vertical advection of momentum (ZAD), 
while the second term corresponds exactly to \autoref{eq:KE+PE_vect_discrete}, therefore:}
\equiv&                   \int\limits_D \textbf{U}_h \cdot \text{ZAD} \;dv  
           + \frac{1}{2} \int_D { {\textbf{U}_h}^2 \frac{1}{e_3} \partial_t  (e_3)  \;dv }    &&&\\
\equiv&                   \int\limits_D \textbf{U}_h \cdot w \partial_k \textbf{U}_h \;dv  
           + \frac{1}{2} \int_D { {\textbf{U}_h}^2 \frac{1}{e_3} \partial_t  (e_3)  \;dv }    &&&\\
\end{flalign*}

There is two main points here. First, the satisfaction of this property links the choice of 
the discrete formulation of the vertical advection and of the horizontal gradient 
of KE. Choosing one imposes the other. For example KE can also be discretized 
as $1/2\,({\overline u^{\,i}}^2 + {\overline v^{\,j}}^2)$. This leads to the following 
expression for the vertical advection:
\begin{equation*}
\frac{1} {e_3 }\; \omega\; \partial_k \textbf{U}_h
\equiv \left( {{\begin{array} {*{20}c}
\frac{1} {e_{1u}\,e_{2u}\,e_{3u}} \;  \overline{\overline {e_{1t}\,e_{2t} \,\omega\;\delta_{k+1/2} 
\left[ \overline u^{\,i+1/2} \right]}}^{\,i+1/2,k}  \hfill \\
\frac{1} {e_{1v}\,e_{2v}\,e_{3v}} \;   \overline{\overline {e_{1t}\,e_{2t} \,\omega \;\delta_{k+1/2}
\left[ \overline v^{\,j+1/2} \right]}}^{\,j+1/2,k} \hfill \\
\end{array}} } \right)
\end{equation*}
a formulation that requires an additional horizontal mean in contrast with 
the one used in NEMO. Nine velocity points have to be used instead of 3. 
This is the reason why it has not been chosen.

Second, as soon as the chosen $s$-coordinate depends on time, an extra constraint
arises on the time derivative of the volume at $u$- and $v$-points:
\begin{flalign*}
e_{1u}\,e_{2u}\,\partial_t (e_{3u}) =\overline{ e_{1t}\,e_{2t}\;\partial_t (e_{3t}) }^{\,i+1/2}    \\
e_{1v}\,e_{2v}\,\partial_t (e_{3v})  =\overline{ e_{1t}\,e_{2t}\;\partial_t (e_{3t}) }^{\,j+1/2}
\end{flalign*}
which is (over-)satified by defining the vertical scale factor as follows:
\begin{flalign} \label{eq:e3u-e3v}
e_{3u} = \frac{1}{e_{1u}\,e_{2u}}\;\overline{ e_{1t}^{ }\,e_{2t}^{ }\,e_{3t}^{ } }^{\,i+1/2}    \\
e_{3v} = \frac{1}{e_{1v}\,e_{2v}}\;\overline{ e_{1t}^{ }\,e_{2t}^{ }\,e_{3t}^{ } }^{\,j+1/2} 
\end{flalign}

Blah blah required on the the step representation of bottom topography.....


% -------------------------------------------------------------------------------------------------------------
%       Pressure Gradient Term
% -------------------------------------------------------------------------------------------------------------
\subsection{Pressure gradient term}
\label{subsec:C.2.6}

\gmcomment{
A pressure gradient has no contribution to the evolution of the vorticity as the 
curl of a gradient is zero. In the $z$-coordinate, this property is satisfied locally 
on a C-grid with 2nd order finite differences (property \autoref{eq:DOM_curl_grad}). 
}

When the equation of state is linear ($i.e.$ when an advection-diffusion equation 
for density can be derived from those of temperature and salinity) the change of 
KE due to the work of pressure forces is balanced by the change of potential 
energy due to buoyancy forces: 
\begin{equation*}
- \int_D  \left. \nabla p \right|_z \cdot \textbf{U}_h \;dv 
= - \int_D \nabla \cdot \left( \rho \,\textbf {U} \right) \,g\,z \;dv
  + \int_D g\, \rho \; \partial_t (z)  \;dv
\end{equation*}

This property can be satisfied in a discrete sense for both $z$- and $s$-coordinates. 
Indeed, defining the depth of a $T$-point, $z_t$, as the sum of the vertical scale 
factors at $w$-points starting from the surface, the work of pressure forces can be 
written as:
\begin{flalign*}
&- \int_D  \left. \nabla p \right|_z \cdot \textbf{U}_h \;dv   
\equiv \sum\limits_{i,j,k} \biggl\{ \;  - \frac{1} {e_{1u}}   \Bigl( 
\delta_{i+1/2} [p_t] - g\;\overline \rho^{\,i+1/2}\;\delta_{i+1/2} [z_t]     \Bigr)  \; u\;b_u  
	&&  \\ & \qquad \qquad  \qquad \qquad  \qquad \quad \ \,
					 	      - \frac{1} {e_{2v}}    \Bigl( 
\delta_{j+1/2} [p_t] - g\;\overline \rho^{\,j+1/2}\delta_{j+1/2} [z_t]      \Bigr)  \; v\;b_v \;  \biggr\}   && \\ 
%
\allowdisplaybreaks
\intertext{Using successively \autoref{eq:DOM_di_adj}, $i.e.$ the skew symmetry property of 
the $\delta$ operator, \autoref{eq:wzv}, the continuity equation, \autoref{eq:dynhpg_sco}, 
the hydrostatic equation in the $s$-coordinate, and $\delta_{k+1/2} \left[ z_t \right] \equiv e_{3w} $,
which comes from the definition of $z_t$, it becomes: }
\allowdisplaybreaks
%
\equiv& +  \sum\limits_{i,j,k}   g  \biggl\{ 
		\overline\rho^{\,i+1/2}\,U\,\delta_{i+1/2}[z_t]    
	+  	\overline\rho^{\,j+1/2}\,V\,\delta_{j+1/2}[z_t]     
	+\Bigl(  \delta_i[U] + \delta_j [V]  \Bigr)\;\frac{p_t}{g} \biggr\}  &&\\
%
\equiv& +  \sum\limits_{i,j,k}   g   \biggl\{ 
		\overline\rho^{\,i+1/2}\,U\,\delta_{i+1/2}[z_t]    
	+  	\overline\rho^{\,j+1/2}\,V\,\delta_{j+1/2}[z_t]     
	 -       \left(   \frac{b_t}{e_{3t}} \partial_t (e_{3t})  +  \delta_k \left[ W \right]    \right) \frac{p_t}{g}    \biggr\}   &&&\\ 
%
\equiv& +  \sum\limits_{i,j,k}  g   \biggl\{ 
		\overline\rho^{\,i+1/2}\,U\,\delta_{i+1/2}[z_t]    
	+  	\overline\rho^{\,j+1/2}\,V\,\delta_{j+1/2}[z_t]     
	+	\frac{W}{g}\;\delta_{k+1/2} [p_t] 
	 -        \frac{p_t}{g}\,\partial_t b_t    \biggr\}    &&&\\
%
\equiv& +  \sum\limits_{i,j,k}  g   \biggl\{ 
		\overline\rho^{\,i+1/2}\,U\,\delta_{i+1/2}[z_t]    
	+  	\overline\rho^{\,j+1/2}\,V\,\delta_{j+1/2}[z_t]     
	- 	W\;e_{3w} \overline \rho^{\,k+1/2}         
	 -        \frac{p_t}{g}\,\partial_t b_t    \biggr\}    &&&\\
%
\equiv& +  \sum\limits_{i,j,k}    g   \biggl\{ 
		\overline\rho^{\,i+1/2}\,U\,\delta_{i+1/2}[z_t]    
	+  	\overline\rho^{\,j+1/2}\,V\,\delta_{j+1/2}[z_t]     
	+ 	W\; \overline \rho^{\,k+1/2}\;\delta_{k+1/2} [z_t]   
	 -        \frac{p_t}{g}\,\partial_t b_t    \biggr\}    &&&\\
%
\allowdisplaybreaks
%
\equiv& - \sum\limits_{i,j,k}   g \; z_t      \biggl\{ 
		\delta_i 	\left[ U\;	\overline \rho^{\,i+1/2} 	\right]
	+ 	\delta_j 	\left[ V\;	\overline \rho^{\,j+1/2} 	\right]
	+ 	\delta_k 	\left[ W\;	\overline \rho^{\,k+1/2} 	\right]       \biggr\}    
             - \sum\limits_{i,j,k}       \biggl\{ p_t\;\partial_t b_t    \biggr\}   &&&\\	
%
\equiv& + \sum\limits_{i,j,k}   g \; z_t    \biggl\{      \partial_t ( e_{3t} \,\rho)    \biggr\}  \; b_t    
             -  \sum\limits_{i,j,k}                 \biggl\{  p_t\;\partial_t b_t                     \biggr\}              &&&\\	
%
\end{flalign*}
The first term is exactly the first term of the right-hand-side of \autoref{eq:KE+PE_vect_discrete}.
It remains to demonstrate that the last term, which is obviously a discrete analogue of 
$\int_D \frac{p}{e_3} \partial_t (e_3)\;dv$ is equal to the last term of \autoref{eq:KE+PE_vect_discrete}.
In other words, the following property must be satisfied:
\begin{flalign*}
           \sum\limits_{i,j,k}  \biggl\{  p_t\;\partial_t b_t                  \biggr\}        
\equiv  \sum\limits_{i,j,k}  \biggl\{ \rho \,g\,\partial_t (z_t) \,b_t  \biggr\}                                
\end{flalign*}

Let introduce $p_w$ the pressure at $w$-point such that $\delta_k [p_w] = - \rho \,g\,e_{3t}$. 
The right-hand-side of the above equation can be transformed as follows:

\begin{flalign*}
               \sum\limits_{i,j,k}  \biggl\{ \rho \,g\,\partial_t (z_t) \,b_t  \biggr\}  
&\equiv   - \sum\limits_{i,j,k}  \biggl\{ \delta_k [p_w]\,\partial_t (z_t) \,e_{1t}\,e_{2t}  \biggr\}        &&&\\
%
&\equiv  + \sum\limits_{i,j,k}  \biggl\{  p_w\, \delta_{k+1/2} [\partial_t (z_t)] \,e_{1t}\,e_{2t}  \biggr\}  
  \equiv  + \sum\limits_{i,j,k}  \biggl\{  p_w\, \partial_t (e_{3w}) \,e_{1t}\,e_{2t}  \biggr\}        &&&\\
&\equiv  + \sum\limits_{i,j,k}  \biggl\{  p_w\, \partial_t (b_w) \biggr\}  
 %
% & \equiv     \sum\limits_{i,j,k} \biggl\{  \frac{1}{e_{3t}} \delta_k [p_w]\;\partial_t (z_t) \,b_w   \right)   \biggr\}           &&&\\
% & \equiv     \sum\limits_{i,j,k} \biggl\{   p_w\;\partial_t \left(    \delta_k [z_t]   \right)  e_{1w}\,e_{2w}   \biggr\}           &&&\\
% & \equiv     \sum\limits_{i,j,k} \biggl\{   p_w\;\partial_t b_w   \biggr\}          
\end{flalign*}
therefore, the balance to be satisfied is:
\begin{flalign*}
           \sum\limits_{i,j,k}  \biggl\{  p_t\;\partial_t (b_t) \biggr\}  \equiv  \sum\limits_{i,j,k}  \biggl\{  p_w\, \partial_t (b_w) \biggr\}  
\end{flalign*}
which is a purely vertical balance:
\begin{flalign*}
           \sum\limits_{k}  \biggl\{  p_t\;\partial_t (e_{3t}) \biggr\}  \equiv  \sum\limits_{k}  \biggl\{  p_w\, \partial_t (e_{3w}) \biggr\}  
\end{flalign*}
Defining $p_w = \overline{p_t}^{\,k+1/2}$

%gm comment
\gmcomment{
\begin{flalign*}
 \sum\limits_{i,j,k} \biggl\{   p_t\;\partial_t b_t   \biggr\}                                &&&\\
 %
 & \equiv     \sum\limits_{i,j,k} \biggl\{  \frac{1}{e_{3t}} \delta_k [p_w]\;\partial_t (z_t) \,b_w    \biggr\}           &&&\\
 & \equiv     \sum\limits_{i,j,k} \biggl\{   p_w\;\partial_t \left(    \delta_{k+1/2} [z_t]   \right)  e_{1w}\,e_{2w}   \biggr\}           &&&\\
 & \equiv     \sum\limits_{i,j,k} \biggl\{   p_w\;\partial_t b_w   \biggr\}          
\end{flalign*}


\begin{flalign*}
\int\limits_D   \rho \, g \, \frac{\partial z }{\partial t} \;dv
\equiv&  \sum\limits_{i,j,k}   \biggl\{  \frac{1}{e_{3t}} \frac{\partial e_{3t}}{\partial t} p   \biggr\} \; b_t   &&&\\
\equiv&  \sum\limits_{i,j,k}   \biggl\{      \biggr\} \; b_t   &&&\\
\end{flalign*}

%
\begin{flalign*}
\equiv& - \int_D \nabla \cdot \left( \rho \,\textbf {U} \right)\;g\;z\;\;dv
   + \int\limits_D g\, \rho \; \frac{\partial z}{\partial t}  \;dv     &&& \\
\end{flalign*}
%
}
%end gm comment


Note that this property strongly constrains the discrete expression of both 
the depth of $T-$points and of the term added to the pressure gradient in the
$s$-coordinate. Nevertheless, it is almost never satisfied since a linear equation 
of state is rarely used.







% ================================================================
% Discrete Total energy Conservation : flux form
% ================================================================
\section{Discrete total energy conservation: flux form}
\label{sec:C.3}

% -------------------------------------------------------------------------------------------------------------
%       Total energy conservation
% -------------------------------------------------------------------------------------------------------------
\subsection{Total energy conservation}
\label{subsec:C_KE+PE_flux}

The discrete form of the total energy conservation, \autoref{eq:Tot_Energy}, is given by:
\begin{flalign*}
\partial_t \left(  \sum\limits_{i,j,k} \biggl\{ \frac{u^2}{2} \,b_u + \frac{v^2}{2}\, b_v +  \rho \, g \, z_t \,b_t  \biggr\} \right) &=0  \\
\end{flalign*}
which in flux form, it leads to:
\begin{flalign*}
                        \sum\limits_{i,j,k} \biggl\{  \frac{u    }{e_{3u}}\,\frac{\partial (e_{3u}u)}{\partial t} \,b_u 
                                                             +  \frac{v    }{e_{3v}}\,\frac{\partial (e_{3v}v)}{\partial t} \,b_v  \biggr\}
&  -  \frac{1}{2} \sum\limits_{i,j,k} \biggl\{  \frac{u^2}{e_{3u}}\frac{\partial    e_{3u}    }{\partial t} \,b_u 
                                                             +  \frac{v^2}{e_{3v}}\frac{\partial    e_{3v}    }{\partial t} \,b_v   \biggr\}      \\
&= - \sum\limits_{i,j,k} \biggl\{ \frac{1}{e_3t}\frac{\partial e_{3t} \rho}{\partial t} \, g \, z_t \,b_t  \biggr\} 
     - \sum\limits_{i,j,k} \biggl\{ \rho \,g\,\frac{\partial z_t}{\partial t} \,b_t  \biggr\}                                     \\
\end{flalign*}

Substituting the discrete expression of the time derivative of the velocity either in vector invariant or in flux form,
leads to the discrete equivalent of the 


% -------------------------------------------------------------------------------------------------------------
%       Coriolis and advection terms: flux form
% -------------------------------------------------------------------------------------------------------------
\subsection{Coriolis and advection terms: flux form}
\label{subsec:C.3.2}

% -------------------------------------------------------------------------------------------------------------
%       Coriolis plus ``metric'' Term
% -------------------------------------------------------------------------------------------------------------
\subsubsection{Coriolis plus ``metric'' term}
\label{subsec:C.3.3} 

In flux from the vorticity term reduces to a Coriolis term in which the Coriolis 
parameter has been modified to account for the ``metric'' term. This altered 
Coriolis parameter is discretised at an f-point. It is given by: 
\begin{equation*}
f+\frac{1} {e_1 e_2 } \left( v \frac{\partial e_2 } {\partial i} - u \frac{\partial e_1 } {\partial j}\right)\;
\equiv \;
f+\frac{1} {e_{1f}\,e_{2f}} \left( \overline v^{\,i+1/2} \delta_{i+1/2} \left[ e_{2u} \right] 
                                               -\overline u^{\,j+1/2} \delta_{j+1/2} \left[ e_{1u}  \right] \right)
\end{equation*}

Either the ENE or EEN scheme is then applied to obtain the vorticity term in flux form. 
It therefore conserves the total KE. The derivation is the same as for the 
vorticity term in the vector invariant form (\autoref{subsec:C_vor}).

% -------------------------------------------------------------------------------------------------------------
%       Flux form advection
% -------------------------------------------------------------------------------------------------------------
\subsubsection{Flux form advection}
\label{subsec:C.3.4} 

The flux form operator of the momentum advection is evaluated using a 
centered second order finite difference scheme. Because of the flux form, 
the discrete operator does not contribute to the global budget of linear 
momentum. Because of the centered second order scheme, it conserves 
the horizontal kinetic energy, that is :

\begin{equation} \label{eq:C_ADV_KE_flux}
 -  \int_D \textbf{U}_h \cdot     \left(                 {{\begin{array} {*{20}c}
\nabla \cdot \left( \textbf{U}\,u \right) \hfill \\
\nabla \cdot \left( \textbf{U}\,v \right) \hfill \\       \end{array}} }           \right)   \;dv 
-   \frac{1}{2} \int_D {  {\textbf{U}_h}^2 \frac{1}{e_3} \frac{\partial  e_3 }{\partial t} \;dv } =\;0
\end{equation}

Let us first consider the first term of the scalar product ($i.e.$ just the the terms 
associated with the i-component of the advection) :
\begin{flalign*}
&  - \int_D u \cdot \nabla \cdot \left(   \textbf{U}\,u   \right) \; dv   \\
%
\equiv& - \sum\limits_{i,j,k} \biggl\{    \frac{1} {b_u}    \biggl(   
      \delta_{i+1/2}  \left[   \overline {U}^{\,i}      \;\overline u^{\,i}          \right]   
   + \delta_j           \left[   \overline {V}^{\,i+1/2}\;\overline u^{\,j+1/2}   \right]   
   + \delta_k          \left[   \overline {W}^{\,i+1/2}\;\overline u^{\,k+1/2} \right]  \biggr)   \;   \biggr\} \, b_u \;u &&&  \\ 
%
\equiv& - \sum\limits_{i,j,k} 
	\biggl\{ 
      \delta_{i+1/2} \left[   \overline {U}^{\,i}\;  \overline u^{\,i}  \right]
   + \delta_j          \left[   \overline {V}^{\,i+1/2}\;\overline u^{\,j+1/2}   \right]
   + \delta_k         \left[   \overline {W}^{\,i+12}\;\overline u^{\,k+1/2}  \right]
   	\; \biggr\} \; u     \\
%
\equiv& + \sum\limits_{i,j,k}
	\biggl\{ 
	   \overline {U}^{\,i}\;	\overline u^{\,i} 	\delta_i \left[ u \right] 
        + \overline {V}^{\,i+1/2}\;	\overline u^{\,j+1/2} 	\delta_{j+1/2} \left[ u \right] 
        + \overline {W}^{\,i+1/2}\;	\overline u^{\,k+1/2}	\delta_{k+1/2} 	\left[ u \right]     \biggr\}     && \\
%
\equiv& + \frac{1}{2} \sum\limits_{i,j,k}    \biggl\{ 
	    \overline{U}^{\,i} 		\delta_i 		\left[ u^2 \right] 
	 + \overline{V}^{\,i+1/2} 	\delta_{j+/2} 	\left[ u^2 \right]
	 + \overline{W}^{\,i+1/2} 	\delta_{k+1/2} 	\left[ u^2 \right]      \biggr\} && \\
%
\equiv& -  \sum\limits_{i,j,k}    \frac{1}{2}   \bigg\{ 
	    U  \; \delta_{i+1/2}    \left[ \overline {u^2}^{\,i} \right]
         + V  \; \delta_{j+1/2}    \left[ \overline {u^2}^{\,i} \right]
	 + W \; \delta_{k+1/2}   \left[ \overline {u^2}^{\,i} \right]     \biggr\}    &&& \\
%
\equiv& - \sum\limits_{i,j,k}  \frac{1}{2}  \overline {u^2}^{\,i}     \biggl\{ 
	   \delta_{i+1/2} 	\left[ U  \right]
	+ \delta_{j+1/2} 	\left[ V  \right]
	+ \delta_{k+1/2} 	\left[ W \right]     \biggr\}    &&& \\
%
\equiv& + \sum\limits_{i,j,k}  \frac{1}{2}  \overline {u^2}^{\,i} 
	\biggl\{     \left(   \frac{1}{e_{3t}} \frac{\partial e_{3t}}{\partial t}   \right) \; b_t     \biggr\}    &&& \\
\end{flalign*}
Applying similar manipulation applied to the second term of the scalar product 
leads to :
\begin{equation*}
 -  \int_D \textbf{U}_h \cdot     \left(                 {{\begin{array} {*{20}c}
\nabla \cdot \left( \textbf{U}\,u \right) \hfill \\
\nabla \cdot \left( \textbf{U}\,v \right) \hfill \\       \end{array}} }           \right)   \;dv     
\equiv + \sum\limits_{i,j,k}  \frac{1}{2}  \left( \overline {u^2}^{\,i} + \overline {v^2}^{\,j} \right)
	\biggl\{     \left(   \frac{1}{e_{3t}} \frac{\partial e_{3t}}{\partial t}   \right) \; b_t     \biggr\}    
\end{equation*}
which is the discrete form of 
$ \frac{1}{2} \int_D u \cdot \nabla \cdot \left(   \textbf{U}\,u   \right) \; dv $. 
\autoref{eq:C_ADV_KE_flux} is thus satisfied.


When the UBS scheme is used to evaluate the flux form momentum advection, 
the discrete operator does not contribute to the global budget of linear momentum 
(flux form). The horizontal kinetic energy is not conserved, but forced to decay 
($i.e.$ the scheme is diffusive). 










% ================================================================
% Discrete Enstrophy Conservation
% ================================================================
\section{Discrete enstrophy conservation}
\label{sec:C.4}


% -------------------------------------------------------------------------------------------------------------
%       Vorticity Term with ENS scheme
% -------------------------------------------------------------------------------------------------------------
\subsubsection{Vorticity term with ENS scheme  (\protect\np{ln\_dynvor\_ens}\forcode{ = .true.})}
\label{subsec:C_vorENS} 

In the ENS scheme, the vorticity term is descretized as follows:
\begin{equation} \tag{\ref{eq:dynvor_ens}}
\left\{   \begin{aligned}
+\frac{1}{e_{1u}} & \overline{q}^{\,i}  & {\overline{ \overline{\left( e_{1v}\,e_{3v}\;  v \right) } } }^{\,i, j+1/2}    \\
- \frac{1}{e_{2v}} & \overline{q}^{\,j}  & {\overline{ \overline{\left( e_{2u}\,e_{3u}\; u \right) } } }^{\,i+1/2, j}  
\end{aligned}  \right.
\end{equation} 

The scheme does not allow but the conservation of the total kinetic energy but the conservation 
of $q^2$, the potential enstrophy for a horizontally non-divergent flow ($i.e.$ when $\chi$=$0$). 
Indeed, using the symmetry or skew symmetry properties of the operators ( \autoref{eq:DOM_mi_adj} 
and \autoref{eq:DOM_di_adj}), it can be shown that:
\begin{equation} \label{eq:C_1.1}
\int_D {q\,\;{\textbf{k}}\cdot \frac{1} {e_3} \nabla \times \left( {e_3 \, q \;{\textbf{k}}\times {\textbf{U}}_h } \right)\;dv} \equiv 0
\end{equation}
where $dv=e_1\,e_2\,e_3 \; di\,dj\,dk$ is the volume element. Indeed, using 
\autoref{eq:dynvor_ens}, the discrete form of the right hand side of \autoref{eq:C_1.1} 
can be transformed as follow:
\begin{flalign*} 
&\int_D q \,\; \textbf{k} \cdot \frac{1} {e_3 } \nabla \times 
	\left(  e_3 \, q \; \textbf{k} \times  \textbf{U}_h   \right)\;   dv       \\
%
& \qquad {\begin{array}{*{20}l} 
&\equiv \sum\limits_{i,j,k} 
q \ \left\{  \delta_{i+1/2}  \left[ - \,\overline {q}^{\,i}\;  \overline{\overline  U }^{\,i,j+1/ 2} \right]  
             - \delta_{j+1/2} \left[	 \overline {q}^{\,j}\;  \overline{\overline  V }^{\,i+1/2, j} \right]     \right\}    \\ 
%
&\equiv \sum\limits_{i,j,k} 
	\left\{   \delta_i [q] \; \overline{q}^{\,i} \; \overline{  \overline U  }^{\,i,j+1/2} 
		+ \delta_j [q] \; \overline{q}^{\,j} \; \overline{\overline V }^{\,i+1/2,j}        \right\}       &&  \\ 
%
&\equiv \,\frac{1} {2} \sum\limits_{i,j,k}  
	\left\{         \delta_i  \left[ q^2 \right] \; \overline{\overline U }^{\,i,j+1/2} 
		      + \delta_j  \left[ q^2 \right] \; \overline{\overline V }^{\,i+1/2,j}      \right\}    &&  \\ 
%
&\equiv - \frac{1} {2} \sum\limits_{i,j,k} 	q^2 \;
	\left\{    \delta_{i+1/2}   \left[   \overline{\overline{ U }}^{\,i,j+1/2}    \right]  
	         + \delta_{j+1/2}  \left[   \overline{\overline{ V }}^{\,i+1/2,j}     \right]    \right\}    && \\ 
\end{array} } 		
%
\allowdisplaybreaks
\intertext{ Since $\overline {\;\cdot \;} $ and $\delta $ operators commute: $\delta_{i+1/2} 
\left[ {\overline a^{\,i}} \right] = \overline {\delta_i \left[ a \right]}^{\,i+1/2}$, 
and introducing the horizontal divergence $\chi $, it becomes: }
\allowdisplaybreaks
%
& \qquad {\begin{array}{*{20}l} 
&\equiv \sum\limits_{i,j,k} - \frac{1} {2} q^2 \; \overline{\overline{ e_{1t}\,e_{2t}\,e_{3t}^{}\, \chi}}^{\,i+1/2,j+1/2} 
\quad \equiv 0 &&
\end{array} } 		
\end{flalign*}
The later equality is obtain only when the flow is horizontally non-divergent, $i.e.$ $\chi$=$0$. 


% -------------------------------------------------------------------------------------------------------------
%       Vorticity Term with EEN scheme
% -------------------------------------------------------------------------------------------------------------
\subsubsection{Vorticity Term with EEN scheme (\protect\np{ln\_dynvor\_een}\forcode{ = .true.})}
\label{subsec:C_vorEEN} 

With the EEN scheme, the vorticity terms are represented as: 
\begin{equation} \tag{\ref{eq:dynvor_een}}
\left\{ {    \begin{aligned}
 +q\,e_3 \, v 	&\equiv +\frac{1}{e_{1u} }   \sum_{\substack{i_p,\,k_p}} 
                         {^{i+1/2-i_p}_j}  \mathbb{Q}^{i_p}_{j_p}  \left( e_{1v} e_{3v} \ v  \right)^{i+i_p-1/2}_{j+j_p}   \\
 - q\,e_3 \, u     &\equiv -\frac{1}{e_{2v} }    \sum_{\substack{i_p,\,k_p}} 
                         {^i_{j+1/2-j_p}}  \mathbb{Q}^{i_p}_{j_p}  \left( e_{2u} e_{3u} \ u  \right)^{i+i_p}_{j+j_p-1/2}   \\
\end{aligned}   } \right.
\end{equation} 
where the indices $i_p$ and $k_p$ take the following value: 
$i_p = -1/2$ or $1/2$ and $j_p = -1/2$ or $1/2$,
and the vorticity triads, ${^i_j}\mathbb{Q}^{i_p}_{j_p}$, defined at $T$-point, are given by: 
\begin{equation} \tag{\ref{eq:Q_triads}}
_i^j \mathbb{Q}^{i_p}_{j_p}
= \frac{1}{12} \ \left(   q^{i-i_p}_{j+j_p} + q^{i+j_p}_{j+i_p} + q^{i+i_p}_{j-j_p}  \right)
\end{equation}


This formulation does conserve the potential enstrophy for a horizontally non-divergent flow ($i.e.$ $\chi=0$). 

Let consider one of the vorticity triad, for example ${^{i}_j}\mathbb{Q}^{+1/2}_{+1/2} $, 
similar manipulation can be done for the 3 others. The discrete form of the right hand 
side of \autoref{eq:C_1.1} applied to this triad only can be transformed as follow:

\begin{flalign*} 
&\int_D {q\,\;{\textbf{k}}\cdot \frac{1} {e_3} \nabla \times \left( {e_3 \, q \;{\textbf{k}}\times {\textbf{U}}_h } \right)\;dv} \\
%
\equiv& \sum\limits_{i,j,k} 
 {q} \    \biggl\{ \;\;
	\delta_{i+1/2} \left[   -\, {{^i_j}\mathbb{Q}^{+1/2}_{+1/2}  \; U^{i+1/2}_{j}}    \right]  
      - \delta_{j+1/2} \left[       {{^i_j}\mathbb{Q}^{+1/2}_{+1/2}  \; V^{i}_{j+1/2}}    \right] 
	\;\;\biggr\}        &&  \\ 
%
\equiv& \sum\limits_{i,j,k} 
	\biggl\{   \delta_i [q] \  {{^i_j}\mathbb{Q}^{+1/2}_{+1/2}  \; U^{i+1/2}_{j}}
		   + \delta_j [q] \  {{^i_j}\mathbb{Q}^{+1/2}_{+1/2}  \; V^{i}_{j+1/2}}   \biggr\}
		&& \\ 
%
... & &&\\
&Demonstation \ to \ be \ done... &&\\
... & &&\\
%
\equiv& \frac{1} {2} \sum\limits_{i,j,k}  
	\biggl\{	\delta_i    \Bigl[    \left(  {{^i_j}\mathbb{Q}^{+1/2}_{+1/2}} \right)^2   \Bigr]\;
			\overline{\overline {U}}^{\,i,j+1/2} 
		      + \delta_j 	\Bigl[    \left( {{^i_j}\mathbb{Q}^{+1/2}_{+1/2}} \right)^2     \Bigr]\; 
			\overline{\overline {V}}^{\,i+1/2,j} 
	\biggr\} 
	&&  \\ 
%
\equiv& - \frac{1} {2} \sum\limits_{i,j,k} 	\left( {{^i_j}\mathbb{Q}^{+1/2}_{+1/2}} \right)^2\;
	\biggl\{    \delta_{i+1/2} 
	      \left[   \overline{\overline {U}}^{\,i,j+1/2}    \right]  
	            + \delta_{j+1/2}
		\left[   \overline{\overline {V}}^{\,i+1/2,j}     \right]  
	\biggr\}    && \\ 
%
\equiv& \sum\limits_{i,j,k} - \frac{1} {2} \left( {{^i_j}\mathbb{Q}^{+1/2}_{+1/2}} \right)^2
                                                            \; \overline{\overline{ b_t^{}\, \chi}}^{\,i+1/2,\,j+1/2}  &&\\
%
\ \ \equiv& \ 0     &&\\
\end{flalign*}





% ================================================================
% Conservation Properties on Tracers
% ================================================================
\section{Conservation properties on tracers}
\label{sec:C.5}


All the numerical schemes used in NEMO are written such that the tracer content 
is conserved by the internal dynamics and physics (equations in flux form). 
For advection, only the CEN2 scheme ($i.e.$ $2^{nd}$ order finite different scheme) 
conserves the global variance of tracer. Nevertheless the other schemes ensure 
that the global variance decreases ($i.e.$ they are at least slightly diffusive). 
For diffusion, all the schemes ensure the decrease of the total tracer variance, 
except the iso-neutral operator. There is generally no strict conservation of mass, 
as the equation of state is non linear with respect to $T$ and $S$. In practice, 
the mass is conserved to a very high accuracy. 
% -------------------------------------------------------------------------------------------------------------
%       Advection Term
% -------------------------------------------------------------------------------------------------------------
\subsection{Advection term}
\label{subsec:C.5.1}

conservation of a tracer, $T$:
\begin{equation*}
\frac{\partial }{\partial t} \left(   \int_D {T\;dv}   \right) 
=  \int_D { \frac{1}{e_3}\frac{\partial \left( e_3 \, T \right)}{\partial t} \;dv }=0
\end{equation*}

conservation of its variance:
\begin{flalign*} 
\frac{\partial }{\partial t} \left( \int_D {\frac{1}{2} T^2\;dv} \right)
=&  \int_D { \frac{1}{e_3} Q      \frac{\partial \left( e_3 \, T \right) }{\partial t} \;dv }  
-   \frac{1}{2} \int_D {  T^2 \frac{1}{e_3} \frac{\partial  e_3 }{\partial t} \;dv }
\end{flalign*}


Whatever the advection scheme considered it conserves of the tracer content as all 
the scheme are written in flux form. Indeed,  let $T$ be the tracer and $\tau_u$, $\tau_v$, 
and $\tau_w$ its interpolated values at velocity point (whatever the interpolation is), 
the conservation of the tracer content due to the advection tendency is obtained as follows: 
\begin{flalign*}
&\int_D { \frac{1}{e_3}\frac{\partial \left( e_3 \, T \right)}{\partial t} \;dv } = - \int_D \nabla \cdot \left( T \textbf{U} \right)\;dv    &&&\\
&\equiv - \sum\limits_{i,j,k}    \biggl\{
	 \frac{1} {b_t}  \left(  \delta_i    \left[   U \;\tau_u   \right]
	                             + \delta_j    \left[   V  \;\tau_v   \right] \right) 
	+ \frac{1} {e_{3t}} \delta_k \left[ w\;\tau_w \right]    \biggl\}  b_t   &&&\\
%
&\equiv - \sum\limits_{i,j,k}     \left\{
	    \delta_i  \left[   U \;\tau_u   \right]
         + \delta_j  \left[   V  \;\tau_v   \right]
 	 + \delta_k \left[ W \;\tau_w \right] \right\}   && \\
&\equiv 0 &&&
\end{flalign*}

The conservation of the variance of tracer due to the advection tendency 
can be achieved only with the CEN2 scheme, $i.e.$ when 
$\tau_u= \overline T^{\,i+1/2}$, $\tau_v= \overline T^{\,j+1/2}$, and $\tau_w= \overline T^{\,k+1/2}$. 
It can be demonstarted as follows:
\begin{flalign*}
&\int_D { \frac{1}{e_3} Q      \frac{\partial \left( e_3 \, T \right) }{\partial t} \;dv }
= - \int\limits_D \tau\;\nabla \cdot \left( T\; \textbf{U} \right)\;dv &&&\\
\equiv& - \sum\limits_{i,j,k} T\;
	\left\{
	   \delta_i  \left[ U  \overline T^{\,i+1/2}  \right]
	+ \delta_j  \left[ V  \overline T^{\,j+1/2}  \right]
	+ \delta_k \left[ W \overline T^{\,k+1/2} \right]          \right\} && \\
\equiv& + \sum\limits_{i,j,k} 
	\left\{     U  \overline T^{\,i+1/2} \,\delta_{i+1/2}  \left[ T \right] 
                 +  V  \overline T^{\,j+1/2} \;\delta_{j+1/2}  \left[ T \right]
                 +  W \overline T^{\,k+1/2}\;\delta_{k+1/2} \left[ T \right]     \right\}      &&\\
\equiv&  + \frac{1} {2}  \sum\limits_{i,j,k}
	\Bigl\{   U  \;\delta_{i+1/2} \left[ T^2 \right]
                 + V  \;\delta_{j+1/2}  \left[ T^2 \right]
                 + W \;\delta_{k+1/2} \left[ T^2 \right]   \Bigr\}     && \\ 
\equiv& - \frac{1} {2}  \sum\limits_{i,j,k} T^2
	\Bigl\{    \delta_i  \left[ U  \right]
                  + \delta_j  \left[ V  \right]
                  + \delta_k \left[ W \right]     \Bigr\}      &&&  \\
\equiv& + \frac{1} {2}  \sum\limits_{i,j,k} T^2
	\Bigl\{   \frac{1}{e_{3t}} \frac{\partial e_{3t}\,T }{\partial t}     \Bigr\}      &&& \\
\end{flalign*}
which is the discrete form of $ \frac{1}{2} \int_D {  T^2 \frac{1}{e_3} \frac{\partial  e_3 }{\partial t} \;dv }$.

% ================================================================
% Conservation Properties on Lateral Momentum Physics
% ================================================================
\section{Conservation properties on lateral momentum physics}
\label{sec:dynldf_properties}


The discrete formulation of the horizontal diffusion of momentum ensures the 
conservation of potential vorticity and the horizontal divergence, and the 
dissipation of the square of these quantities ($i.e.$ enstrophy and the 
variance of the horizontal divergence) as well as the dissipation of the 
horizontal kinetic energy. In particular, when the eddy coefficients are 
horizontally uniform, it ensures a complete separation of vorticity and 
horizontal divergence fields, so that diffusion (dissipation) of vorticity 
(enstrophy) does not generate horizontal divergence (variance of the 
horizontal divergence) and \textit{vice versa}. 

These properties of the horizontal diffusion operator are a direct consequence 
of properties \autoref{eq:DOM_curl_grad} and \autoref{eq:DOM_div_curl}. 
When the vertical curl of the horizontal diffusion of momentum (discrete sense) 
is taken, the term associated with the horizontal gradient of the divergence is 
locally zero. 

% -------------------------------------------------------------------------------------------------------------
%       Conservation of Potential Vorticity
% -------------------------------------------------------------------------------------------------------------
\subsection{Conservation of potential vorticity}
\label{subsec:C.6.1}

The lateral momentum diffusion term conserves the potential vorticity :
\begin{flalign*}
&\int \limits_D \frac{1} {e_3 } \textbf{k} \cdot \nabla \times 
	\Bigl[    \nabla_h  \left( A^{\,lm}\;\chi  \right)
	        - \nabla_h \times \left( A^{\,lm}\;\zeta \; \textbf{k} \right)    \Bigr]\;dv   \\ 
%\end{flalign*}
%%%%%%%%%%  recheck here....  (gm)
%\begin{flalign*}
=& \int \limits_D  -\frac{1} {e_3 } \textbf{k} \cdot \nabla \times 
	\Bigl[ \nabla_h \times \left( A^{\,lm}\;\zeta \; \textbf{k} \right)  \Bigr]\;dv  \\ 
%\end{flalign*}
%\begin{flalign*}
\equiv& \sum\limits_{i,j}
	\left\{
	  \delta_{i+1/2} \left[  \frac {e_{2v}} {e_{1v}\,e_{3v}}  \delta_i \left[ A_f^{\,lm} e_{3f} \zeta  \right]  \right]
 	+ \delta_{j+1/2} \left[  \frac {e_{1u}} {e_{2u}\,e_{3u}}  \delta_j \left[ A_f^{\,lm} e_{3f} \zeta  \right]  \right]
	\right\} 	 \\ 
%
\intertext{Using \autoref{eq:DOM_di_adj}, it follows:}
%
\equiv& \sum\limits_{i,j,k} 
	-\,\left\{
	   \frac{e_{2v}} {e_{1v}\,e_{3v}}  \delta_i	\left[ A_f^{\,lm} e_{3f} \zeta  \right]\;\delta_i \left[ 1\right]
	 + \frac{e_{1u}} {e_{2u}\,e_{3u}}  \delta_j 	\left[ A_f^{\,lm} e_{3f} \zeta  \right]\;\delta_j \left[ 1\right]
	\right\} \quad \equiv 0 
	 \\ 
\end{flalign*}

% -------------------------------------------------------------------------------------------------------------
%       Dissipation of Horizontal Kinetic Energy
% -------------------------------------------------------------------------------------------------------------
\subsection{Dissipation of horizontal kinetic energy}
\label{subsec:C.6.2}

The lateral momentum diffusion term dissipates the horizontal kinetic energy:
%\begin{flalign*}
\begin{equation*}
\begin{split}
\int_D \textbf{U}_h \cdot 
	\left[ \nabla_h 		\right.   &     \left.       \left( A^{\,lm}\;\chi \right)      
	- \nabla_h \times  \left( A^{\,lm}\;\zeta \;\textbf{k} \right)     \right] \; dv    \\
\\  %%%
\equiv& \sum\limits_{i,j,k} 
	\left\{
	  \frac{1} {e_{1u}}               \delta_{i+1/2} \left[  A_T^{\,lm}          \chi     \right]
	- \frac{1} {e_{2u}\,e_{3u}}  \delta_j           \left[ A_f^{\,lm} e_{3f} \zeta   \right]
	\right\} \; e_{1u}\,e_{2u}\,e_{3u} \;u     \\
&\;\; + 	\left\{
	   \frac{1} {e_{2u}}             \delta_{j+1/2}	\left[ A_T^{\,lm}          \chi    \right] 
	+ \frac{1} {e_{1v}\,e_{3v}} \delta_i            \left[ A_f^{\,lm} e_{3f} \zeta  \right] 
	\right\} \; e_{1v}\,e_{2u}\,e_{3v} \;v     \qquad \\ 
\\  %%%
\equiv& \sum\limits_{i,j,k} 
 	\Bigl\{
	  e_{2u}\,e_{3u} \;u\;  \delta_{i+1/2} \left[ A_T^{\,lm}           \chi    \right]
	- e_{1u}             \;u\;  \delta_j           \left[ A_f^{\,lm} e_{3f} \zeta  \right]
	 \Bigl\} 
	 \\ 
&\;\; + \Bigl\{
	   e_{1v}\,e_{3v} \;v\;  \delta_{j+1/2}  \left[ A_T^{\,lm}           \chi    \right]
	+ e_{2v}             \;v\;  \delta_i            \left[ A_f^{\,lm} e_{3f} \zeta  \right]
	\Bigl\}      \\ 
\\  %%%
\equiv& \sum\limits_{i,j,k} 
	- \Bigl(
	  \delta_i   \left[  e_{2u}\,e_{3u} \;u  \right]
	+ \delta_j  \left[  e_{1v}\,e_{3v}  \;v  \right] 
	     \Bigr) \;  A_T^{\,lm} \chi   \\ 
&\;\; - \Bigl(
	  \delta_{i+1/2}  \left[  e_{2v}  \;v  \right]
	- \delta_{j+1/2}  \left[  e_{1u} \;u  \right] 
	     \Bigr)\;  A_f^{\,lm} e_{3f} \zeta      \\ 
\\  %%%
\equiv& \sum\limits_{i,j,k} 
	- A_T^{\,lm}  \,\chi^2   \;e_{1t}\,e_{2t}\,e_{3t}
	- A_f ^{\,lm}  \,\zeta^2 \;e_{1f }\,e_{2f }\,e_{3f}  
\quad \leq 0       \\
\end{split}
\end{equation*}

% -------------------------------------------------------------------------------------------------------------
%       Dissipation of Enstrophy
% -------------------------------------------------------------------------------------------------------------
\subsection{Dissipation of enstrophy}
\label{subsec:C.6.3}

The lateral momentum diffusion term dissipates the enstrophy when the eddy 
coefficients are horizontally uniform:
\begin{flalign*}
&\int\limits_D  \zeta \; \textbf{k} \cdot \nabla \times 
	\left[   \nabla_h \left( A^{\,lm}\;\chi  \right)
	       - \nabla_h \times \left( A^{\,lm}\;\zeta \; \textbf{k} \right)   \right]\;dv &&&\\
&\quad = A^{\,lm} \int \limits_D \zeta \textbf{k} \cdot \nabla \times 
 	\left[    \nabla_h \times \left( \zeta \; \textbf{k} \right)   \right]\;dv &&&\\
&\quad \equiv A^{\,lm} \sum\limits_{i,j,k}  \zeta \;e_{3f} 
	\left\{     \delta_{i+1/2} \left[  \frac{e_{2v}} {e_{1v}\,e_{3v}} \delta_i \left[ e_{3f} \zeta  \right]   \right]
	          + \delta_{j+1/2} \left[  \frac{e_{1u}} {e_{2u}\,e_{3u}} \delta_j \left[ e_{3f} \zeta  \right]   \right]      \right\}   &&&\\ 
%
\intertext{Using \autoref{eq:DOM_di_adj}, it follows:}
%
&\quad \equiv  - A^{\,lm} \sum\limits_{i,j,k} 
	\left\{    \left(  \frac{1} {e_{1v}\,e_{3v}}  \delta_i \left[ e_{3f} \zeta  \right]  \right)^2   b_v
	         + \left(  \frac{1} {e_{2u}\,e_{3u}}  \delta_j \left[ e_{3f} \zeta  \right] \right)^2   b_u  \right\}  \quad \leq \;0    &&&\\
\end{flalign*}

% -------------------------------------------------------------------------------------------------------------
%       Conservation of Horizontal Divergence
% -------------------------------------------------------------------------------------------------------------
\subsection{Conservation of horizontal divergence}
\label{subsec:C.6.4}

When the horizontal divergence of the horizontal diffusion of momentum 
(discrete sense) is taken, the term associated with the vertical curl of the 
vorticity is zero locally, due to \autoref{eq:DOM_div_curl}. 
The resulting term conserves the $\chi$ and dissipates $\chi^2$ 
when the eddy coefficients are horizontally uniform.
\begin{flalign*}
& \int\limits_D  \nabla_h \cdot 
	\Bigl[     \nabla_h \left( A^{\,lm}\;\chi \right)
	          - \nabla_h \times \left( A^{\,lm}\;\zeta \;\textbf{k} \right)    \Bigr]  dv
= \int\limits_D  \nabla_h \cdot \nabla_h \left( A^{\,lm}\;\chi  \right)   dv   \\
%
&\equiv \sum\limits_{i,j,k} 
	\left\{   \delta_i \left[ A_u^{\,lm} \frac{e_{2u}\,e_{3u}} {e_{1u}}  \delta_{i+1/2} \left[ \chi \right]  \right]
	        + \delta_j \left[ A_v^{\,lm} \frac{e_{1v}\,e_{3v}} {e_{2v}}  \delta_{j+1/2} \left[ \chi \right]  \right]    \right\}    \\ 
%
\intertext{Using \autoref{eq:DOM_di_adj}, it follows:}
%
&\equiv \sum\limits_{i,j,k} 
	- \left\{   \frac{e_{2u}\,e_{3u}} {e_{1u}}  A_u^{\,lm} \delta_{i+1/2} \left[ \chi \right] \delta_{i+1/2} \left[ 1 \right] 
	          + \frac{e_{1v}\,e_{3v}} {e_{2v}}  A_v^{\,lm} \delta_{j+1/2} \left[ \chi \right] \delta_{j+1/2} \left[ 1 \right]    \right\} 
	\quad \equiv 0      \\ 
\end{flalign*}

% -------------------------------------------------------------------------------------------------------------
%       Dissipation of Horizontal Divergence Variance
% -------------------------------------------------------------------------------------------------------------
\subsection{Dissipation of horizontal divergence variance}
\label{subsec:C.6.5}

\begin{flalign*}
&\int\limits_D \chi \;\nabla_h \cdot 
	\left[    \nabla_h              \left( A^{\,lm}\;\chi                    \right)
	        - \nabla_h   \times  \left( A^{\,lm}\;\zeta \;\textbf{k} \right)    \right]\;  dv
 = A^{\,lm}   \int\limits_D \chi \;\nabla_h \cdot \nabla_h \left( \chi \right)\;  dv    \\ 
%
&\equiv A^{\,lm}  \sum\limits_{i,j,k}  \frac{1} {e_{1t}\,e_{2t}\,e_{3t}}  \chi 
	\left\{
	   \delta_i  \left[   \frac{e_{2u}\,e_{3u}} {e_{1u}}  \delta_{i+1/2} \left[ \chi \right]   \right]
	+ \delta_j  \left[   \frac{e_{1v}\,e_{3v}} {e_{2v}}   \delta_{j+1/2} \left[ \chi \right]   \right]
	\right\} \;   e_{1t}\,e_{2t}\,e_{3t}    \\ 
%
\intertext{Using \autoref{eq:DOM_di_adj}, it turns out to be:}
%
&\equiv - A^{\,lm} \sum\limits_{i,j,k}
	\left\{    \left(  \frac{1} {e_{1u}}  \delta_{i+1/2}  \left[ \chi \right]  \right)^2  b_u
            + \left(  \frac{1} {e_{2v}}  \delta_{j+1/2}  \left[ \chi \right]  \right)^2  b_v    \right\}    
\quad \leq 0             \\
\end{flalign*}

% ================================================================
% Conservation Properties on Vertical Momentum Physics
% ================================================================
\section{Conservation properties on vertical momentum physics}
\label{sec:C.7}

As for the lateral momentum physics, the continuous form of the vertical diffusion 
of momentum satisfies several integral constraints. The first two are associated 
with the conservation of momentum and the dissipation of horizontal kinetic energy:
\begin{align*}
 \int\limits_D   \frac{1} {e_3 }\; \frac{\partial } {\partial k} 
       \left(   \frac{A^{\,vm}} {e_3 }\; \frac{\partial \textbf{U}_h } {\partial k}   \right)\;  dv 
	\qquad \quad &= \vec{\textbf{0}}    \\
%
\intertext{and}
%
\int\limits_D 
	\textbf{U}_h \cdot   \frac{1} {e_3 }\; \frac{\partial } {\partial k}
	\left(   \frac{A^{\,vm}} {e_3 }\; \frac{\partial \textbf{U}_h } {\partial k}   \right)\; dv    \quad &\leq 0     \\
\end{align*}

The first property is obvious. The second results from:
\begin{flalign*}
\int\limits_D 
	\textbf{U}_h \cdot   \frac{1} {e_3 }\; \frac{\partial } {\partial k}
	\left( \frac{A^{\,vm}} {e_3 }\; \frac{\partial \textbf{U}_h } {\partial k}   \right)\;dv    &&&\\
\end{flalign*}
\begin{flalign*}
&\equiv \sum\limits_{i,j,k} 
	\left( 
	   u\; \delta_k \left[ \frac{A_u^{\,vm}} {e_{3uw}} \delta_{k+1/2}  \left[ u \right]  \right]\;  e_{1u}\,e_{2u} 
	+ v\; \delta_k \left[ \frac{A_v^{\,vm}} {e_{3vw}} \delta_{k+1/2}   \left[ v \right]  \right]\;  e_{1v}\,e_{2v} \right)   &&&\\ 
%
\intertext{since the horizontal scale factor does not depend on $k$, it follows:}
%
&\equiv - \sum\limits_{i,j,k} 
	\left(  \frac{A_u^{\,vm}} {e_{3uw}} \left( \delta_{k+1/2} \left[ u \right] \right)^2\; e_{1u}\,e_{2u} 
	      + \frac{A_v^{\,vm}} {e_{3vw}}  \left( \delta_{k+1/2} \left[ v \right] \right)^2\; e_{1v}\,e_{2v}  \right)
\quad \leq 0   &&&\\
\end{flalign*}

The vorticity is also conserved. Indeed:
\begin{flalign*}
\int \limits_D 
	\frac{1} {e_3 } \textbf{k} \cdot \nabla \times 
		\left( \frac{1} {e_3 }\; \frac{\partial } {\partial k}  \left( 
		        \frac{A^{\,vm}} {e_3}\; \frac{\partial \textbf{U}_h } {\partial k}  
		  \right)  \right)\; dv   &&&\\ 
\end{flalign*}
\begin{flalign*}
\equiv \sum\limits_{i,j,k}  \frac{1} {e_{3f}}\; \frac{1} {e_{1f}\,e_{2f}}
	\bigg\{    \biggr.   \quad
	\delta_{i+1/2} 
		&\left(   \frac{e_{2v}} {e_{3v}} \delta_k  \left[  \frac{1} {e_{3vw}} \delta_{k+1/2} \left[ v \right]  \right]  \right)   &&\\
	\biggl. 
	- \delta_{j+1/2} 
		&\left(   \frac{e_{1u}} {e_{3u}} \delta_k \left[  \frac{1} {e_{3uw}}\delta_{k+1/2} \left[ u \right]  \right]   \right)
	\biggr\} \;
	e_{1f}\,e_{2f}\,e_{3f} \; \equiv 0   && \\
\end{flalign*}

If the vertical diffusion coefficient is uniform over the whole domain, the 
enstrophy is dissipated, $i.e.$
\begin{flalign*}
\int\limits_D \zeta \, \textbf{k} \cdot \nabla \times 
	\left(   \frac{1} {e_3}\; \frac{\partial } {\partial k}
		\left( \frac{A^{\,vm}} {e_3 }\; \frac{\partial \textbf{U}_h } {\partial k}   \right)   \right)\; dv = 0   &&&\\
\end{flalign*}

This property is only satisfied in $z$-coordinates:
\begin{flalign*}
\int\limits_D \zeta \, \textbf{k} \cdot \nabla \times 
	\left(  \frac{1} {e_3}\; \frac{\partial } {\partial k}
		\left( \frac{A^{\,vm}} {e_3 }\; \frac{\partial \textbf{U}_h } {\partial k}  \right)   \right)\; dv   &&& \\ 
\end{flalign*}
\begin{flalign*}
\equiv \sum\limits_{i,j,k} \zeta \;e_{3f} \;
	\biggl\{ 	\biggr.	\quad
	\delta_{i+1/2} 
		&\left(   \frac{e_{2v}} {e_{3v}} \delta_k \left[ \frac{A_v^{\,vm}} {e_{3vw}} \delta_{k+1/2}[v]  \right]   \right)   &&\\
	- \delta_{j+1/2} 
		&\biggl.
		\left(   \frac{e_{1u}} {e_{3u}} \delta_k \left[ \frac{A_u^{\,vm}} {e_{3uw}} \delta_{k+1/2} [u]  \right]    \right) \biggr\}   &&\\ 
\end{flalign*}
\begin{flalign*}
\equiv \sum\limits_{i,j,k} \zeta \;e_{3f} 
	\biggl\{		\biggr.	\quad
	\frac{1} {e_{3v}} \delta_k 
		&\left[ \frac{A_v^{\,vm}} {e_{3vw}} \delta_{k+1/2} \left[ \delta_{i+1/2} \left[ e_{2v}\,v \right] \right]   \right]    &&\\ 
 	 \biggl. 
	- \frac{1} {e_{3u}} \delta_k 
		&\left[  \frac{A_u^{\,vm}} {e_{3uw}} \delta_{k+1/2} \left[ \delta_{j+1/2} \left[ e_{1u}\,u \right] \right]  \right]  \biggr\}  &&\\ 
\end{flalign*}
Using the fact that the vertical diffusion coefficients are uniform, and that in 
$z$-coordinate, the vertical scale factors do not depend on $i$ and $j$ so 
that: $e_{3f} =e_{3u} =e_{3v} =e_{3t} $ and $e_{3w} =e_{3uw} =e_{3vw} $, 
it follows:
\begin{flalign*}
\equiv A^{\,vm} \sum\limits_{i,j,k} \zeta \;\delta_k 
	\left[   \frac{1} {e_{3w}} \delta_{k+1/2} \Bigl[   \delta_{i+1/2} \left[ e_{2v}\,v \right]
		                          - \delta_{j+1/ 2} \left[ e_{1u}\,u \right]   \Bigr]    \right]    &&&\\
\end{flalign*}
\begin{flalign*}
\equiv - A^{\,vm} \sum\limits_{i,j,k} \frac{1} {e_{3w}}
	\left( \delta_{k+1/2} \left[ \zeta  \right] \right)^2 \; e_{1f}\,e_{2f}  \; \leq 0    &&&\\
\end{flalign*}
Similarly, the horizontal divergence is obviously conserved:

\begin{flalign*}
\int\limits_D \nabla \cdot 
\left( \frac{1} {e_3 }\; \frac{\partial } {\partial k}
		\left( \frac{A^{\,vm}} {e_3 }\; \frac{\partial \textbf{U}_h } {\partial k} \right) \right)\; dv = 0    &&&\\
\end{flalign*}
and the square of the horizontal divergence decreases ($i.e.$ the horizontal 
divergence is dissipated) if the vertical diffusion coefficient is uniform over the 
whole domain:

\begin{flalign*}
\int\limits_D \chi \;\nabla \cdot 
\left( \frac{1} {e_3 }\; \frac{\partial } {\partial k}
		\left( \frac{A^{\,vm}} {e_3 }\; \frac{\partial \textbf{U}_h } {\partial k}  \right) \right)\;  dv = 0  &&&\\
\end{flalign*}
This property is only satisfied in the $z$-coordinate:
\begin{flalign*}
\int\limits_D \chi \;\nabla \cdot 
\left( \frac{1} {e_3 }\; \frac{\partial } {\partial k}
		\left( \frac{A^{\,vm}} {e_3 }\; \frac{\partial \textbf{U}_h } {\partial k} \right)  \right)\; dv    &&&\\
\end{flalign*}
\begin{flalign*}
\equiv \sum\limits_{i,j,k} \frac{\chi } {e_{1t}\,e_{2t}}
	\biggl\{ 	\Biggr.	\quad
	\delta_{i+1/2} 
		&\left(   \frac{e_{2u}} {e_{3u}} \delta_k 
		   	\left[ \frac{A_u^{\,vm}} {e_{3uw}} \delta_{k+1/2} [u] \right] \right)    &&\\ 
 	\Biggl. 
	+ \delta_{j+1/2} 
		&\left( \frac{e_{1v}} {e_{3v}} \delta_k 
			\left[ \frac{A_v^{\,vm}} {e_{3vw}} \delta_{k+1/2} [v] \right]   \right)
	\Biggr\} \;  e_{1t}\,e_{2t}\,e_{3t}   &&\\ 
\end{flalign*}

\begin{flalign*}
\equiv A^{\,vm} \sum\limits_{i,j,k}  \chi \,
	\biggl\{	\biggr.	\quad
	\delta_{i+1/2}
		&\left( 
		   \delta_k \left[ 
			\frac{1} {e_{3uw}} \delta_{k+1/2} \left[ e_{2u}\,u \right] \right]   \right)    && \\ 
	\biggl. 
	+ \delta_{j+1/2} 
		&\left(    \delta_k \left[ 
			\frac{1} {e_{3vw}} \delta_{k+1/2} \left[ e_{1v}\,v \right] \right]   \right)   \biggr\}    && \\ 
\end{flalign*}

\begin{flalign*}
\equiv -A^{\,vm} \sum\limits_{i,j,k} 
\frac{\delta_{k+1/2} \left[ \chi \right]} {e_{3w}}\; \biggl\{ 
	\delta_{k+1/2} \Bigl[
		   \delta_{i+1/2} \left[ e_{2u}\,u \right]
		+ \delta_{j+1/2} \left[ e_{1v}\,v \right]  \Bigr]    \biggr\}    &&&\\
\end{flalign*}

\begin{flalign*}
\equiv -A^{\,vm} \sum\limits_{i,j,k}
 \frac{1} {e_{3w}} \delta_{k+1/2} \left[ \chi \right]\; \delta_{k+1/2} \left[ e_{1t}\,e_{2t} \;\chi \right]   &&&\\
\end{flalign*}

\begin{flalign*}
\equiv -A^{\,vm} \sum\limits_{i,j,k} 
\frac{e_{1t}\,e_{2t}} {e_{3w}}\; \left( \delta_{k+1/2} \left[ \chi \right]  \right)^2     \quad  \equiv 0    &&&\\
\end{flalign*}

% ================================================================
% Conservation Properties on Tracer Physics
% ================================================================
\section{Conservation properties on tracer physics}
\label{sec:C.8}

The numerical schemes used for tracer subgridscale physics are written such 
that the heat and salt contents are conserved (equations in flux form). 
Since a flux form is used to compute the temperature and salinity, 
the quadratic form of these quantities ($i.e.$ their variance) globally tends to diminish. 
As for the advection term, there is conservation of mass only if the Equation Of Seawater is linear. 

% -------------------------------------------------------------------------------------------------------------
%       Conservation of Tracers
% -------------------------------------------------------------------------------------------------------------
\subsection{Conservation of tracers}
\label{subsec:C.8.1}

constraint of conservation of tracers:
\begin{flalign*}
&\int\limits_D  \nabla  \cdot \left( A\;\nabla T \right)\;dv  &&&\\ 
\\
&\equiv \sum\limits_{i,j,k} 
	\biggl\{ 	\biggr.
	\delta_i 
		\left[ 
		A_u^{\,lT} \frac{e_{2u}\,e_{3u}} {e_{1u}} \delta_{i+1/2} 
			\left[ T \right]
		\right]
	+ \delta_j 
		\left[ 
		A_v^{\,lT} \frac{e_{1v}\,e_{3v}} {e_{2v}} \delta_{j+1/2} 
			\left[ T \right] 
		\right]
	&&\\  & \qquad \qquad \qquad \qquad \qquad \qquad \quad \;\;\;
	+ \delta_k 
		\left[ 
		A_w^{\,vT} \frac{e_{1t}\,e_{2t}} {e_{3t}} \delta_{k+1/2} 
			\left[ T \right] 
		\right]
	\biggr\}   \quad  \equiv 0
	&&\\ 
\end{flalign*}

In fact, this property simply results from the flux form of the operator. 

% -------------------------------------------------------------------------------------------------------------
%       Dissipation of Tracer Variance
% -------------------------------------------------------------------------------------------------------------
\subsection{Dissipation of tracer variance}
\label{subsec:C.8.2}

constraint on the dissipation of tracer variance:
\begin{flalign*}
\int\limits_D T\;\nabla & \cdot \left( A\;\nabla T \right)\;dv	&&&\\ 
&\equiv   \sum\limits_{i,j,k} \; T
\biggl\{  \biggr.
     \delta_i \left[ A_u^{\,lT} \frac{e_{2u}\,e_{3u}} {e_{1u}} \delta_{i+1/2} \left[T\right] \right]
& + \delta_j \left[ A_v^{\,lT} \frac{e_{1v} \,e_{3v}} {e_{2v}} \delta_{j+1/2} \left[T\right] \right] 
		\quad&& \\ 
 \biggl. 
&&+ \delta_k \left[A_w^{\,vT}\frac{e_{1t}\,e_{2t}} {e_{3t}}\delta_{k+1/2}\left[T\right]\right]
\biggr\} && 
\end{flalign*}
\begin{flalign*}
\equiv - \sum\limits_{i,j,k} 
	\biggl\{ 	\biggr.	\quad
	&    A_u^{\,lT} \left( \frac{1} {e_{1u}} \delta_{i+1/2} \left[ T \right]  \right)^2   e_{1u}\,e_{2u}\,e_{3u}    && \\
	& + A_v^{\,lT} \left( \frac{1} {e_{2v}} \delta_{j+1/2} \left[ T \right]  \right)^2   e_{1v}\,e_{2v}\,e_{3v}     && \\ \biggl. 
	& + A_w^{\,vT} \left( \frac{1} {e_{3w}} \delta_{k+1/2} \left[ T \right]   \right)^2    e_{1w}\,e_{2w}\,e_{3w}   \biggr\} 
	\quad      \leq 0      && \\ 
\end{flalign*}


%%%%  end of appendix in gm comment
%}
\end{document}
