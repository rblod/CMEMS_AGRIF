%% ================================================================
%% Abstract
%% ================================================================

%% Common part between NEMO-SI3-TOP
\NEMO\ (``Nucleus for European Modelling of the Ocean'') is a framework of ocean-related engines.
It is intended to be a flexible tool for studying the ocean dynamics and thermodynamics (``blue ocean''),
as well as its interactions with the components of the Earth climate system over
a wide range of space and time scales.
Within \NEMO, the ocean engine is interfaced with a sea-ice model (\SIcube\ or
\href{http://github.com/CICE-Consortium/CICE}{CICE}),
passive tracers and biogeochemical models (\TOP) and,
via the \href{http://portal.enes.org/oasis}{OASIS} coupler,
with several atmospheric general circulation models.
It also supports two-way grid embedding by means of the \href{http://agrif.imag.fr}{AGRIF} software.

%% Specific part
The primitive equation model is adapted to regional and global ocean circulation problems down to
kilometric scale.
Prognostic variables are the three-dimensional velocity field, a non-linear sea surface height,
the \textit{Conservative} Temperature and the \textit{Absolute} Salinity.
In the horizontal direction, the model uses a curvilinear orthogonal grid and
in the vertical direction, a full or partial step $z$-coordinate, or $s$-coordinate, or
a mixture of the two.
The distribution of variables is a three-dimensional Arakawa C-type grid.
Various physical choices are available to describe ocean physics,
so as various HPC functionalities to improve performances.
