
\chapter*{Introduction}

%\chaptertoc

%\paragraph{Changes record} ~\\

%\thispagestyle{plain}

%{\footnotesize
%  \begin{tabularx}{\textwidth}{l||X|X}
%    Release & Author(s) & Modifications \\
%    \hline
%    {\em x.x} & {\em ...} & {\em ...}   \\
%    {\em ...} & {\em ...} & {\em ...}   \\
%  \end{tabularx}
%}

%\clearpage

The \textbf{N}ucleus for \textbf{E}uropean \textbf{M}odelling of the \textbf{O}cean (\NEMO) is
a framework of ocean related engines, namely the aforementioned for
the ocean dynamics and thermodynamics,
\SIcube \footnote{\textbf{S}ea-\textbf{I}ce modelling \textbf{I}ntegrated \textbf{I}nitiative}
for the sea-ice dynamics and thermodynamics,
\TOP \footnote{\textbf{T}racer in the \textbf{O}cean \textbf{P}aradigm} for
the biogeochemistry (both transport and sources minus sinks
(\PISCES \footnote{
  \textbf{P}elagic \textbf{I}nteractions \textbf{S}cheme for
  \textbf{C}arbon and \textbf{E}cosystem \textbf{S}tudies
}
)).
The ocean component has been developed from the legacy of
the \OPA \footnote{\textbf{O}c\'{e}an \textbf{PA}rall\'{e}lis\'{e} (French)}
model, described in \citet{madec.delecluse.ea_NPM98}.
This model has been used for a wide range of applications, both regional or global,
as a forced ocean model and as a model coupled with the sea-ice and/or the atmosphere.

This manual provides information about the physics represented by the ocean component of \NEMO\ and
the rationale for the choice of numerical schemes and the model design.
For the use of framework,
a guide which gathers the \texttt{README} files spread out in the source code can be build and
exported in a web or printable format (see \path{./doc/rst}).
Also a online copy is available on the \href{http://forge.ipsl.jussieu.fr/nemo}{forge platform}.

%% =================================================================================================
\section*{Manual outline}

\subsection*{Chapters}

The manual mirrors the organization of the model and it is organised in as follows:
after the presentation of the continuous equations
(primitive equations with temperature and salinity, and an equation of seawater) in the next chapter,
the following chapters refer to specific terms of the equations each associated with
a group of modules.

\begin{description}
\item [\nameref{chap:MB}] presents the equations and their assumptions, the vertical coordinates used,
and the subgrid scale physics.
The equations are written in a curvilinear coordinate system, with a choice of vertical coordinates
($z$, $s$, \zstar, \sstar, \ztilde, \stilde, and a mix of them).
Momentum equations are formulated in vector invariant or flux form.
Dimensional units in the meter, kilogram, second (MKS) international system are used throughout.
The following chapters deal with the discrete equations.
\item [\nameref{chap:TD}] presents the model time stepping environment.
it is a three level scheme in which the tendency terms of the equations are evaluated either
centered in time, or forward, or backward depending of the nature of the term.
\item [\nameref{chap:DOM}] presents the model \textbf{DOM}ain.
It is discretised on a staggered grid (Arakawa C grid) with masking of land areas.
Vertical discretisation used depends on both how the bottom topography is represented and whether
the free surface is linear or not.
Full step or partial step $z$-coordinate or $s$- (terrain-following) coordinate is used with
linear free surface (level position are then fixed in time).
In non-linear free surface, the corresponding rescaled height coordinate formulation
(\zstar or \sstar) is used
(the level position then vary in time as a function of the sea surface heigh).
\item [\nameref{chap:TRA} and \nameref{chap:DYN}] describe the discretisation of
the prognostic equations for the active \textbf{TRA}cers (potential temperature and salinity) and
the momentum (\textbf{DYN}amic).
Explicit, split-explicit and filtered free surface formulations are implemented.
A number of numerical schemes are available for momentum advection,
for the computation of the pressure gradients, as well as for the advection of tracers
(second or higher order advection schemes, including positive ones).
\item [\nameref{chap:SBC}] can be implemented as prescribed fluxes,
or bulk formulations for the surface fluxes (wind stress, heat, freshwater).
The model allows penetration of solar radiation.
There is an optional geothermal heating at the ocean bottom.
Within the \NEMO\ system the ocean model is interactively coupled with
a sea ice model (\SIcube) and a biogeochemistry model (\PISCES).
Interactive coupling to Atmospheric models is possible via the \OASIS\ coupler.
Two-way nesting is also available through an interface to the \AGRIF\ package,
\ie\ \textbf{A}daptative \textbf{G}rid \textbf{R}efinement in \textbf{F}ortran
\citep{debreu.vouland.ea_CG08}.
The interface code for coupling to an alternative sea ice model (\CICE) has now been upgraded so that
it works for both global and regional domains.
\item [\nameref{chap:LBC}] presents the \textbf{L}ateral
\textbf{B}oun\textbf{D}ar\textbf{Y} \textbf{C}onditions.
Global configurations of the model make use of the ORCA tripolar grid,
with special north fold boundary condition.
Free-slip or no-slip boundary conditions are allowed at land boundaries.
Closed basin geometries as well as periodic domains and open boundary conditions are possible.
\item [\nameref{chap:LDF} and \nameref{chap:ZDF}] describe the physical parameterisations
(\textbf{L}ateral \textbf{D}i\textbf{F}fusion and vertical \textbf{Z} \textbf{D}i\textbf{F}fusion)
The model includes an implicit treatment of vertical viscosity and diffusivity.
The lateral Laplacian and biharmonic viscosity and diffusion can be rotated following
a geopotential or neutral direction.
There is an optional eddy induced velocity \citep{gent.mcwilliams_JPO90} with
a space and time variable coefficient \citet{treguier.held.ea_JPO97}.
The model has vertical harmonic viscosity and diffusion with a space and time variable coefficient,
with options to compute the coefficients with \citet{blanke.delecluse_JPO93},
\citet{pacanowski.philander_JPO81}, or \citet{umlauf.burchard_JMR03} mixing schemes.
\item [\nameref{chap:DIA}] describes model \textbf{I}n-\textbf{O}utputs \textbf{M}anagement and
specific online \textbf{DIA}gnostics.
The diagnostics includes the output of all the tendencies of the momentum and tracers equations,
the output of tracers \textbf{TR}en\textbf{D}s averaged over the time evolving mixed layer,
the output of the tendencies of the barotropic vorticity equation,
the computation of on-line \textbf{FLO}ats trajectories...
\item [\nameref{chap:OBS}] describes a tool which reads in \textbf{OBS}ervation files
(profile temperature and salinity, sea surface temperature, sea level anomaly and
sea ice concentration) and calculates an interpolated model equivalent value at
the observation location and nearest model timestep.
Originally developed of data assimilation, it is a fantastic tool for model and data comparison.
\item [\nameref{chap:ASM}] describes how increments produced by
data \textbf{A}s\textbf{S}i\textbf{M}ilation may be applied to the model equations.
\item [\nameref{chap:STO}]
\item [\nameref{chap:MISC}] (including solvers)
\item [\nameref{chap:CFGS}] provides finally a brief introduction to
the pre-defined model configurations
(water column model \texttt{C1D}, ORCA and GYRE families of configurations).
\end{description}

%% =================================================================================================
\subsection*{Appendices}

\begin{description}
\item [\nameref{apdx:SCOORD}]
\item [\nameref{apdx:DIFFOPERS}]
\item [\nameref{apdx:INVARIANTS}]
\item [\nameref{apdx:TRIADS}]
\item [\nameref{apdx:DOMCFG}]
\item [\nameref{apdx:CODING}]
\end{description}
