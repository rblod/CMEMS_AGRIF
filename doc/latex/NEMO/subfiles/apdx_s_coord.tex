\documentclass[../main/NEMO_manual]{subfiles}

\begin{document}

\chapter{Curvilinear $s-$Coordinate Equations}
\label{apdx:SCOORD}

%    {\em 4.0} & {\em Mike Bell} & {\em review}  \\
%    {\em 3.x} & {\em Gurvan Madec} & {\em original}  \\

\thispagestyle{plain}

\chaptertoc

\paragraph{Changes record} ~\\

{\footnotesize
  \begin{tabularx}{\textwidth}{l||X|X}
    Release & Author(s) & Modifications \\
    \hline
    {\em   4.0} & {\em ...} & {\em ...} \\
    {\em   3.6} & {\em ...} & {\em ...} \\
    {\em   3.4} & {\em ...} & {\em ...} \\
    {\em <=3.4} & {\em ...} & {\em ...}
  \end{tabularx}
}

\clearpage

\section{Chain rule for $s-$coordinates}
\label{sec:SCOORD_chain}

In order to establish the set of Primitive Equation in curvilinear $s$-coordinates
(\ie\ an orthogonal curvilinear coordinate in the horizontal and
an Arbitrary Lagrangian Eulerian (ALE) coordinate in the vertical),
we start from the set of equations established in \autoref{subsec:MB_zco_Eq} for
the special case $k = z$ and thus $e_3 = 1$,
and we introduce an arbitrary vertical coordinate $a = a(i,j,z,t)$.
Let us define a new vertical scale factor by $e_3 = \partial z / \partial s$ (which now depends on $(i,j,z,t)$) and
the horizontal slope of $s-$surfaces by:
\begin{equation}
  \label{eq:SCOORD_s_slope}
  \sigma_1 =\frac{1}{e_1 } \; \left. {\frac{\partial z}{\partial i}} \right|_s
  \quad \text{and} \quad
  \sigma_2 =\frac{1}{e_2 } \; \left. {\frac{\partial z}{\partial j}} \right|_s .
\end{equation}

The model fields (e.g. pressure $p$) can be viewed as functions of $(i,j,z,t)$ (e.g. $p(i,j,z,t)$) or as
functions of $(i,j,s,t)$ (e.g. $p(i,j,s,t)$). The symbol $\bullet$ will be used to represent any one of
these fields.  Any ``infinitesimal'' change in $\bullet$ can be written in two forms:
\begin{equation}
  \label{eq:SCOORD_s_infin_changes}
  \begin{aligned}
    & \delta \bullet =  \delta i \left. \frac{ \partial \bullet }{\partial i} \right|_{j,s,t}
                + \delta j \left. \frac{ \partial \bullet }{\partial i} \right|_{i,s,t}
                + \delta s \left. \frac{ \partial \bullet }{\partial s} \right|_{i,j,t}
                + \delta t \left. \frac{ \partial \bullet }{\partial t} \right|_{i,j,s} , \\
    & \delta \bullet =  \delta i \left. \frac{ \partial \bullet }{\partial i} \right|_{j,z,t}
                + \delta j \left. \frac{ \partial \bullet }{\partial i} \right|_{i,z,t}
                + \delta z \left. \frac{ \partial \bullet }{\partial z} \right|_{i,j,t}
                + \delta t \left. \frac{ \partial \bullet }{\partial t} \right|_{i,j,z} .
  \end{aligned}
\end{equation}
Using the first form and considering a change $\delta i$ with $j, z$ and $t$ held constant, shows that
\begin{equation}
  \label{eq:SCOORD_s_chain_rule1}
      \left. {\frac{\partial \bullet }{\partial i}} \right|_{j,z,t}  =
      \left. {\frac{\partial \bullet }{\partial i}} \right|_{j,s,t}
    + \left. {\frac{\partial s       }{\partial i}} \right|_{j,z,t} \;
      \left. {\frac{\partial \bullet }{\partial s}} \right|_{i,j,t} .
\end{equation}
The term $\left. \partial s / \partial i \right|_{j,z,t}$ can be related to the slope of constant $s$ surfaces,
(\autoref{eq:SCOORD_s_slope}), by applying the second of (\autoref{eq:SCOORD_s_infin_changes}) with $\bullet$ set to
$s$ and $j, t$ held constant
\begin{equation}
\label{eq:SCOORD_delta_s}
\delta s|_{j,t} =
         \delta i \left. \frac{ \partial s }{\partial i} \right|_{j,z,t}
       + \delta z \left. \frac{ \partial s }{\partial z} \right|_{i,j,t} .
\end{equation}
Choosing to look at a direction in the $(i,z)$ plane in which $\delta s = 0$ and using
(\autoref{eq:SCOORD_s_slope}) we obtain
\begin{equation}
\left. \frac{ \partial s }{\partial i} \right|_{j,z,t} =
         -  \left. \frac{ \partial z }{\partial i} \right|_{j,s,t} \;
            \left. \frac{ \partial s }{\partial z} \right|_{i,j,t}
	 = - \frac{e_1 }{e_3 }\sigma_1  .
\label{eq:SCOORD_ds_di_z}
\end{equation}
Another identity, similar in form to (\autoref{eq:SCOORD_ds_di_z}), can be derived
by choosing $\bullet$ to be $s$ and using the second form of (\autoref{eq:SCOORD_s_infin_changes}) to consider
changes in which $i , j$ and $s$ are constant. This shows that
\begin{equation}
\label{eq:SCOORD_w_in_s}
w_s = \left. \frac{ \partial z }{\partial t} \right|_{i,j,s} =
- \left. \frac{ \partial z }{\partial s} \right|_{i,j,t}
  \left. \frac{ \partial s }{\partial t} \right|_{i,j,z}
  = - e_3 \left. \frac{ \partial s }{\partial t} \right|_{i,j,z} .
\end{equation}

In what follows, for brevity, indication of the constancy of the $i, j$ and $t$ indices is
usually omitted. Using the arguments outlined above one can show that the chain rules needed to establish
the model equations in the curvilinear $s-$coordinate system are:
\begin{equation}
  \label{eq:SCOORD_s_chain_rule2}
  \begin{aligned}
    &\left. {\frac{\partial \bullet }{\partial t}} \right|_z  =
    \left. {\frac{\partial \bullet }{\partial t}} \right|_s
	 + \frac{\partial \bullet }{\partial s}\; \frac{\partial s}{\partial t} , \\
    &\left. {\frac{\partial \bullet }{\partial i}} \right|_z  =
    \left. {\frac{\partial \bullet }{\partial i}} \right|_s
    +\frac{\partial \bullet }{\partial s}\; \frac{\partial s}{\partial i}=
    \left. {\frac{\partial \bullet }{\partial i}} \right|_s
    -\frac{e_1 }{e_3 }\sigma_1 \frac{\partial \bullet }{\partial s} , \\
    &\left. {\frac{\partial \bullet }{\partial j}} \right|_z  =
    \left. {\frac{\partial \bullet }{\partial j}} \right|_s
    + \frac{\partial \bullet }{\partial s}\;\frac{\partial s}{\partial j}=
    \left. {\frac{\partial \bullet }{\partial j}} \right|_s
    - \frac{e_2 }{e_3 }\sigma_2 \frac{\partial \bullet }{\partial s} , \\
    &\;\frac{\partial \bullet }{\partial z}  \;\; = \frac{1}{e_3 }\frac{\partial \bullet }{\partial s} .
  \end{aligned}
\end{equation}

%% =================================================================================================
\section{Continuity equation in $s-$coordinates}
\label{sec:SCOORD_continuity}

Using (\autoref{eq:SCOORD_s_chain_rule1}) and
the fact that the horizontal scale factors $e_1$ and $e_2$ do not depend on the vertical coordinate,
the divergence of the velocity relative to the ($i$,$j$,$z$) coordinate system is transformed as follows in order to
obtain its expression in the curvilinear $s-$coordinate system:

\begin{subequations}
  \begin{align*}
    {
    \begin{array}{*{20}l}
      \nabla \cdot {\mathrm {\mathbf U}}
      &= \frac{1}{e_1 \,e_2 }  \left[ \left. {\frac{\partial (e_2 \,u)}{\partial i}} \right|_z
        +\left. {\frac{\partial(e_1 \,v)}{\partial j}} \right|_z  \right]
        + \frac{\partial w}{\partial z} \\ \\
      &     = \frac{1}{e_1 \,e_2 }  \left[
		  \left.   \frac{\partial (e_2 \,u)}{\partial i}    \right|_s
		  - \frac{e_1 }{e_3 } \sigma_1 \frac{\partial (e_2 \,u)}{\partial s}
        + \left.   \frac{\partial (e_1 \,v)}{\partial j}    \right|_s
		  - \frac{e_2 }{e_3 } \sigma_2 \frac{\partial (e_1 \,v)}{\partial s}	\right]
        + \frac{\partial w}{\partial s} \; \frac{\partial s}{\partial z} \\ \\
      &     = \frac{1}{e_1 \,e_2 }   \left[
		  \left.   \frac{\partial (e_2 \,u)}{\partial i}    \right|_s
        + \left.   \frac{\partial (e_1 \,v)}{\partial j}    \right|_s       	\right]
        + \frac{1}{e_3 }\left[        \frac{\partial w}{\partial s}
        -  \sigma_1 \frac{\partial u}{\partial s}
        -  \sigma_2 \frac{\partial v}{\partial s}      \right] \\ \\
      &     = \frac{1}{e_1 \,e_2 \,e_3 }   \left[
		  \left.   \frac{\partial (e_2 \,e_3 \,u)}{\partial i}    \right|_s
		  -\left.    e_2 \,u    \frac{\partial e_3 }{\partial i}     \right|_s
        + \left.  \frac{\partial (e_1 \,e_3 \,v)}{\partial j}    \right|_s
		  - \left.    e_1 v      \frac{\partial e_3 }{\partial j}    \right|_s   \right] \\
      & \qquad \qquad \qquad \qquad \qquad \qquad \qquad \qquad \qquad
        + \frac{1}{e_3 } \left[        \frac{\partial w}{\partial s}
        -  \sigma_1 \frac{\partial u}{\partial s}
        -  \sigma_2 \frac{\partial v}{\partial s}      \right]      \\
      %
      \intertext{Noting that $
      \frac{1}{e_1} \left.{ \frac{\partial e_3}{\partial i}} \right|_s
      =\frac{1}{e_1} \left.{ \frac{\partial^2 z}{\partial i\,\partial s}} \right|_s
      =\frac{\partial}{\partial s} \left( {\frac{1}{e_1 } \left.{ \frac{\partial z}{\partial i} }\right|_s } \right)
      =\frac{\partial \sigma_1}{\partial s}
      $ and $
      \frac{1}{e_2 }\left. {\frac{\partial e_3 }{\partial j}} \right|_s
      =\frac{\partial \sigma_2}{\partial s}
      $, it becomes:}
    %
      \nabla \cdot {\mathrm {\mathbf U}}
      & = \frac{1}{e_1 \,e_2 \,e_3 }  \left[
		  \left.  \frac{\partial (e_2 \,e_3 \,u)}{\partial i} \right|_s
        +\left.  \frac{\partial (e_1 \,e_3 \,v)}{\partial j} \right|_s        \right] \\
      & \qquad \qquad \qquad \qquad \quad
        +\frac{1}{e_3 }\left[ {\frac{\partial w}{\partial s}-u\frac{\partial \sigma_1 }{\partial s}-v\frac{\partial \sigma_2 }{\partial s}-\sigma_1 \frac{\partial u}{\partial s}-\sigma_2 \frac{\partial v}{\partial s}} \right] \\
      \\
      & = \frac{1}{e_1 \,e_2 \,e_3 }  \left[
		  \left.  \frac{\partial (e_2 \,e_3 \,u)}{\partial i} \right|_s
        +\left.  \frac{\partial (e_1 \,e_3 \,v)}{\partial j} \right|_s        \right]
        + \frac{1}{e_3 } \; \frac{\partial}{\partial s}   \left[  w -  u\;\sigma_1  - v\;\sigma_2  \right]
    \end{array}
        }
  \end{align*}
\end{subequations}

Here, $w$ is the vertical velocity relative to the $z-$coordinate system.
Using the first form of (\autoref{eq:SCOORD_s_infin_changes})
and the definitions (\autoref{eq:SCOORD_s_slope}) and (\autoref{eq:SCOORD_w_in_s}) for $\sigma_1$, $\sigma_2$ and  $w_s$,
one can show that the vertical velocity, $w_p$ of a point
moving with the horizontal velocity of the fluid along an $s$ surface is given by
\begin{equation}
\label{eq:SCOORD_w_p}
\begin{split}
w_p  = & \left. \frac{ \partial z }{\partial t} \right|_s
     + \frac{u}{e_1} \left. \frac{ \partial z }{\partial i} \right|_s
     + \frac{v}{e_2} \left. \frac{ \partial z }{\partial j} \right|_s \\
     = & w_s + u \sigma_1 + v \sigma_2 .
\end{split}
\end{equation}
 The vertical velocity across this surface is denoted by
\begin{equation}
  \label{eq:SCOORD_w_s}
  \omega  = w - w_p = w - ( w_s + \sigma_1 \,u + \sigma_2 \,v )  .
\end{equation}
Hence
\begin{equation}
\frac{1}{e_3 } \frac{\partial}{\partial s}   \left[  w -  u\;\sigma_1  - v\;\sigma_2  \right] =
\frac{1}{e_3 } \frac{\partial}{\partial s} \left[  \omega + w_s \right] =
   \frac{1}{e_3 } \left[ \frac{\partial \omega}{\partial s}
 + \left. \frac{ \partial }{\partial t} \right|_s \frac{\partial z}{\partial s} \right] =
   \frac{1}{e_3 } \frac{\partial \omega}{\partial s} + \frac{1}{e_3 } \left. \frac{ \partial e_3}{\partial t} . \right|_s
\end{equation}

Using (\autoref{eq:SCOORD_w_s}) in our expression for $\nabla \cdot {\mathrm {\mathbf U}}$ we obtain
our final expression for the divergence of the velocity in the curvilinear $s-$coordinate system:
\begin{equation}
      \nabla \cdot {\mathrm {\mathbf U}} =
         \frac{1}{e_1 \,e_2 \,e_3 }    \left[
		  \left.  \frac{\partial (e_2 \,e_3 \,u)}{\partial i} \right|_s
        +\left.  \frac{\partial (e_1 \,e_3 \,v)}{\partial j} \right|_s        \right]
        + \frac{1}{e_3 } \frac{\partial \omega }{\partial s}
        + \frac{1}{e_3 } \left. \frac{\partial e_3}{\partial t} \right|_s .
\end{equation}

As a result, the continuity equation \autoref{eq:MB_PE_continuity} in the $s-$coordinates is:
\begin{equation}
  \label{eq:SCOORD_sco_Continuity}
  \frac{1}{e_3 } \frac{\partial e_3}{\partial t}
  + \frac{1}{e_1 \,e_2 \,e_3 }\left[
    {\left. {\frac{\partial (e_2 \,e_3 \,u)}{\partial i}} \right|_s
      +  \left. {\frac{\partial (e_1 \,e_3 \,v)}{\partial j}} \right|_s } \right]
  +\frac{1}{e_3 }\frac{\partial \omega }{\partial s} = 0 .
\end{equation}
An additional term has appeared that takes into account
the contribution of the time variation of the vertical coordinate to the volume budget.

%% =================================================================================================
\section{Momentum equation in $s-$coordinate}
\label{sec:SCOORD_momentum}

Here we only consider the first component of the momentum equation,
the generalization to the second one being straightforward.

$\bullet$ \textbf{Total derivative in vector invariant form}

Let us consider \autoref{eq:MB_dyn_vect}, the first component of the momentum equation in the vector invariant form.
Its total $z-$coordinate time derivative,
$\left. \frac{D u}{D t} \right|_z$ can be transformed as follows in order to obtain
its expression in the curvilinear $s-$coordinate system:

\begin{subequations}
  \begin{align*}
    {
    \begin{array}{*{20}l}
      \left. \frac{D u}{D t} \right|_z
      &= \left. {\frac{\partial u }{\partial t}} \right|_z
        - \left. \zeta \right|_z v
        + \frac{1}{2e_1} \left.{ \frac{\partial (u^2+v^2)}{\partial i}} \right|_z
        + w \;\frac{\partial u}{\partial z} \\ \\
      &= \left. {\frac{\partial u }{\partial t}} \right|_z
        -  \frac{1}{e_1 \,e_2 }\left[ { \left.{ \frac{\partial (e_2 \,v)}{\partial i} }\right|_z
        -\left.{ \frac{\partial (e_1 \,u)}{\partial j} }\right|_z } \right] \; v
        +  \frac{1}{2e_1} \left.{ \frac{\partial (u^2+v^2)}{\partial i} } \right|_z
        +  w \;\frac{\partial u}{\partial z}      \\
        %
      \intertext{introducing the chain rule (\autoref{eq:SCOORD_s_chain_rule1}) }
      %
      &= \left. {\frac{\partial u }{\partial t}} \right|_z
        - \frac{1}{e_1\,e_2}\left[ { \left.{ \frac{\partial (e_2 \,v)}{\partial i} } \right|_s
        -\left.{ \frac{\partial (e_1 \,u)}{\partial j} } \right|_s } \right.
        \left. {-\frac{e_1}{e_3}\sigma_1 \frac{\partial (e_2 \,v)}{\partial s}
        +\frac{e_2}{e_3}\sigma_2 \frac{\partial (e_1 \,u)}{\partial s}} \right] \; v  \\
      & \qquad \qquad \qquad \qquad
        {
        + \frac{1}{2e_1} \left(                                  \left.  \frac{\partial (u^2+v^2)}{\partial i} \right|_s
        - \frac{e_1}{e_3}\sigma_1 \frac{\partial (u^2+v^2)}{\partial s}               \right)
        + \frac{w}{e_3 } \;\frac{\partial u}{\partial s}
        } \\ \\
      &= \left. {\frac{\partial u }{\partial t}} \right|_z
        - \left. \zeta \right|_s \;v
        + \frac{1}{2\,e_1}\left. {\frac{\partial (u^2+v^2)}{\partial i}} \right|_s \\
      &\qquad \qquad \qquad \quad
        + \frac{w}{e_3 } \;\frac{\partial u}{\partial s}
        + \left[   {\frac{\sigma_1 }{e_3 }\frac{\partial v}{\partial s}
        - \frac{\sigma_2 }{e_3 }\frac{\partial u}{\partial s}}   \right]\;v
        - \frac{\sigma_1 }{2e_3 }\frac{\partial (u^2+v^2)}{\partial s} \\ \\
      &= \left. {\frac{\partial u }{\partial t}} \right|_z
        - \left. \zeta \right|_s \;v
        + \frac{1}{2\,e_1}\left. {\frac{\partial (u^2+v^2)}{\partial i}} \right|_s \\
      &\qquad \qquad \qquad \quad
        + \frac{1}{e_3} \left[    {w\frac{\partial u}{\partial s}
        +\sigma_1 v\frac{\partial v}{\partial s} - \sigma_2 v\frac{\partial u}{\partial s}
        - \sigma_1 u\frac{\partial u}{\partial s} - \sigma_1 v\frac{\partial v}{\partial s}} \right] \\ \\
      &= \left. {\frac{\partial u }{\partial t}} \right|_z
        - \left. \zeta \right|_s \;v
        + \frac{1}{2\,e_1}\left. {\frac{\partial (u^2+v^2)}{\partial i}} \right|_s
        + \frac{1}{e_3} \left[  w - \sigma_2 v - \sigma_1 u  \right]
        \; \frac{\partial u}{\partial s} .  \\
        %
      \intertext{Introducing $\omega$, the dia-s-surface velocity given by (\autoref{eq:SCOORD_w_s}) }
      %
      &= \left. {\frac{\partial u }{\partial t}} \right|_z
        - \left. \zeta \right|_s \;v
        + \frac{1}{2\,e_1}\left. {\frac{\partial (u^2+v^2)}{\partial i}} \right|_s
        + \frac{1}{e_3 } \left( \omega + w_s \right) \frac{\partial u}{\partial s}   \\
    \end{array}
    }
  \end{align*}
\end{subequations}
Applying the time derivative chain rule (first equation of (\autoref{eq:SCOORD_s_chain_rule1})) to $u$ and
using (\autoref{eq:SCOORD_w_in_s}) provides the expression of the last term of the right hand side,
\[
  {
    \begin{array}{*{20}l}
      \frac{w_s}{e_3}  \;\frac{\partial u}{\partial s}
      = - \left. \frac{\partial s}{\partial t} \right|_z \;  \frac{\partial u }{\partial s}
      = \left. {\frac{\partial u }{\partial t}} \right|_s  - \left. {\frac{\partial u }{\partial t}} \right|_z \ .
    \end{array}
  }
\]
This leads to the $s-$coordinate formulation of the total $z-$coordinate time derivative,
\ie\ the total $s-$coordinate time derivative :
\begin{align}
  \label{eq:SCOORD_sco_Dt_vect}
  \left. \frac{D u}{D t} \right|_s
  = \left. {\frac{\partial u }{\partial t}} \right|_s
  - \left. \zeta \right|_s \;v
  + \frac{1}{2\,e_1}\left. {\frac{\partial (u^2+v^2)}{\partial i}} \right|_s
  + \frac{1}{e_3 } \omega \;\frac{\partial u}{\partial s} .
\end{align}
Therefore, the vector invariant form of the total time derivative has exactly the same mathematical form in
$z-$ and $s-$coordinates.
This is not the case for the flux form as shown in next paragraph.

$\bullet$ \textbf{Total derivative in flux form}

Let us start from the total time derivative in the curvilinear $s-$coordinate system we have just establish.
Following the procedure used to establish (\autoref{eq:MB_flux_form}), it can be transformed into :
% \begin{subequations}
\begin{align*}
  {
  \begin{array}{*{20}l}
    \left. \frac{D u}{D t} \right|_s  &= \left. {\frac{\partial u }{\partial t}} \right|_s
    & -  \zeta \;v
      + \frac{1}{2\;e_1 } \frac{\partial \left( {u^2+v^2} \right)}{\partial i}
      + \frac{1}{e_3} \omega \;\frac{\partial u}{\partial s} \\ \\
                                      &= \left. {\frac{\partial u }{\partial t}} \right|_s
    &+\frac{1}{e_1\;e_2}  \left(    \frac{\partial \left( {e_2 \,u\,u } \right)}{\partial i}
      + \frac{\partial \left( {e_1 \,u\,v } \right)}{\partial j}     \right)
      + \frac{1}{e_3 } \frac{\partial \left( {\omega\,u} \right)}{\partial s} \\ \\
                                      &&- \,u \left[     \frac{1}{e_1 e_2 } \left(    \frac{\partial(e_2 u)}{\partial i}
                                         + \frac{\partial(e_1 v)}{\partial j}    \right)
                                         + \frac{1}{e_3}        \frac{\partial \omega}{\partial s}                       \right] \\ \\
                                      &&- \frac{v}{e_1 e_2 }\left(    v	\;\frac{\partial e_2 }{\partial i}
                                         -u	\;\frac{\partial e_1 }{\partial j} 	\right) . \\
  \end{array}
  }
\end{align*}
Introducing the vertical scale factor inside the horizontal derivative of the first two terms
(\ie\ the horizontal divergence), it becomes :
\begin{align*}
  {
  \begin{array}{*{20}l}
    % \begin{align*} {\begin{array}{*{20}l}
    %     {\begin{array}{*{20}l} \left. \frac{D u}{D t} \right|_s
    &= \left. {\frac{\partial u }{\partial t}} \right|_s
    &+ \frac{1}{e_1\,e_2\,e_3}  \left(  \frac{\partial( e_2 e_3 \,u^2 )}{\partial i}
      + \frac{\partial( e_1 e_3 \,u v )}{\partial j}
      -  e_2 u u \frac{\partial e_3}{\partial i}
      -  e_1 u v \frac{\partial e_3 }{\partial j}    \right)
      + \frac{1}{e_3} \frac{\partial \left( {\omega\,u} \right)}{\partial s} \\ \\
    && - \,u \left[  \frac{1}{e_1 e_2 e_3} \left(   \frac{\partial(e_2 e_3 \, u)}{\partial i}
       + \frac{\partial(e_1 e_3 \, v)}{\partial j}
       -  e_2 u \;\frac{\partial e_3 }{\partial i}
       -  e_1 v \;\frac{\partial e_3 }{\partial j}   \right)
       + \frac{1}{e_3}        \frac{\partial \omega}{\partial s}                       \right] \\ \\
    && - \frac{v}{e_1 e_2 }\left( 	v  \;\frac{\partial e_2 }{\partial i}
       -u  \;\frac{\partial e_1 }{\partial j} 	\right) \\ \\
    &= \left. {\frac{\partial u }{\partial t}} \right|_s
    &+ \frac{1}{e_1\,e_2\,e_3}  \left(  \frac{\partial( e_2 e_3 \,u\,u )}{\partial i}
      + \frac{\partial( e_1 e_3 \,u\,v )}{\partial j}    \right)
      + \frac{1}{e_3 } \frac{\partial \left( {\omega\,u} \right)}{\partial s} \\ \\
    && - \,u \left[  \frac{1}{e_1 e_2 e_3} \left(   \frac{\partial(e_2 e_3 \, u)}{\partial i}
       + \frac{\partial(e_1 e_3 \, v)}{\partial j}  \right)
       + \frac{1}{e_3}        \frac{\partial \omega}{\partial s}                       \right]
       - \frac{v}{e_1 e_2 }\left( 	v   \;\frac{\partial e_2 }{\partial i}
       -u   \;\frac{\partial e_1 }{\partial j} 	\right)     .             \\
     %
    \intertext {Introducing a more compact form for the divergence of the momentum fluxes,
    and using (\autoref{eq:SCOORD_sco_Continuity}), the $s-$coordinate continuity equation,
    it becomes : }
  %
    &= \left. {\frac{\partial u }{\partial t}} \right|_s
    &+ \left.  \nabla \cdot \left(   {{\mathrm {\mathbf U}}\,u}   \right)    \right|_s
      + \,u \frac{1}{e_3 } \frac{\partial e_3}{\partial t}
      - \frac{v}{e_1 e_2 }\left(    v  \;\frac{\partial e_2 }{\partial i}
      -u  \;\frac{\partial e_1 }{\partial j} 	\right)
    \\
  \end{array}
  }
\end{align*}
which leads to the $s-$coordinate flux formulation of the total $s-$coordinate time derivative,
\ie\ the total $s-$coordinate time derivative in flux form:
\begin{flalign}
  \label{eq:SCOORD_sco_Dt_flux}
  \left. \frac{D u}{D t} \right|_s   = \frac{1}{e_3}  \left. \frac{\partial ( e_3\,u)}{\partial t} \right|_s
  + \left.  \nabla \cdot \left(   {{\mathrm {\mathbf U}}\,u}   \right)    \right|_s
  - \frac{v}{e_1 e_2 }\left(    v  \;\frac{\partial e_2 }{\partial i}
    -u  \;\frac{\partial e_1 }{\partial j}            \right).
\end{flalign}
which is the total time derivative expressed in the curvilinear $s-$coordinate system.
It has the same form as in the $z-$coordinate but for
the vertical scale factor that has appeared inside the time derivative which
comes from the modification of (\autoref{eq:SCOORD_sco_Continuity}),
the continuity equation.

$\bullet$ \textbf{horizontal pressure gradient}

The horizontal pressure gradient term can be transformed as follows:
\[
  \begin{split}
    -\frac{1}{\rho_o \, e_1 }\left. {\frac{\partial p}{\partial i}} \right|_z
    & =-\frac{1}{\rho_o e_1 }\left[ {\left. {\frac{\partial p}{\partial i}} \right|_s -\frac{e_1 }{e_3 }\sigma_1 \frac{\partial p}{\partial s}} \right] \\
    & =-\frac{1}{\rho_o \,e_1 }\left. {\frac{\partial p}{\partial i}} \right|_s +\frac{\sigma_1 }{\rho_o \,e_3 }\left( {-g\;\rho \;e_3 } \right) \\
    &=-\frac{1}{\rho_o \,e_1 }\left. {\frac{\partial p}{\partial i}} \right|_s -\frac{g\;\rho }{\rho_o }\sigma_1 .
  \end{split}
\]
Applying similar manipulation to the second component and
replacing $\sigma_1$ and $\sigma_2$ by their expression \autoref{eq:SCOORD_s_slope}, it becomes:
\begin{equation}
  \label{eq:SCOORD_grad_p_1}
  \begin{split}
    -\frac{1}{\rho_o \, e_1 } \left. {\frac{\partial p}{\partial i}} \right|_z
    &=-\frac{1}{\rho_o \,e_1 } \left(     \left.              {\frac{\partial p}{\partial i}} \right|_s
      + g\;\rho  \;\left. {\frac{\partial z}{\partial i}} \right|_s    \right) \\
             %
    -\frac{1}{\rho_o \, e_2 }\left. {\frac{\partial p}{\partial j}} \right|_z
    &=-\frac{1}{\rho_o \,e_2 } \left(    \left.               {\frac{\partial p}{\partial j}} \right|_s
      + g\;\rho \;\left. {\frac{\partial z}{\partial j}} \right|_s   \right) . \\
  \end{split}
\end{equation}

An additional term appears in (\autoref{eq:SCOORD_grad_p_1}) which accounts for
the tilt of $s-$surfaces with respect to geopotential $z-$surfaces.

As in $z$-coordinate,
the horizontal pressure gradient can be split in two parts following \citet{marsaleix.auclair.ea_OM08}.
Let defined a density anomaly, $d$, by $d=(\rho - \rho_o)/ \rho_o$,
and a hydrostatic pressure anomaly, $p_h'$, by $p_h'= g \; \int_z^\eta d \; e_3 \; dk$.
The pressure is then given by:
\[
  \begin{split}
    p &= g\; \int_z^\eta \rho \; e_3 \; dk = g\; \int_z^\eta \rho_o \left( d + 1 \right) \; e_3 \; dk   \\
    &= g \, \rho_o \; \int_z^\eta d \; e_3 \; dk + \rho_o g \, \int_z^\eta e_3 \; dk .
  \end{split}
\]
Therefore, $p$ and $p_h'$ are linked through:
\begin{equation}
  \label{eq:SCOORD_pressure}
  p = \rho_o \; p_h' + \rho_o \, g \, ( \eta - z )
\end{equation}
and the hydrostatic pressure balance expressed in terms of $p_h'$ and $d$ is:
\[
  \frac{\partial p_h'}{\partial k} = - d \, g \, e_3 .
\]

Substituing \autoref{eq:SCOORD_pressure} in \autoref{eq:SCOORD_grad_p_1} and
using the definition of the density anomaly it becomes an expression in two parts:
\begin{equation}
  \label{eq:SCOORD_grad_p_2}
  \begin{split}
    -\frac{1}{\rho_o \, e_1 } \left. {\frac{\partial p}{\partial i}} \right|_z
    &=-\frac{1}{e_1 } \left(     \left.              {\frac{\partial p_h'}{\partial i}} \right|_s
      + g\; d  \;\left. {\frac{\partial z}{\partial i}} \right|_s    \right)  - \frac{g}{e_1 } \frac{\partial \eta}{\partial i} ,  \\
             %
    -\frac{1}{\rho_o \, e_2 }\left. {\frac{\partial p}{\partial j}} \right|_z
    &=-\frac{1}{e_2 } \left(    \left.               {\frac{\partial p_h'}{\partial j}} \right|_s
      + g\; d \;\left. {\frac{\partial z}{\partial j}} \right|_s   \right)  - \frac{g}{e_2 } \frac{\partial \eta}{\partial j} . \\
  \end{split}
\end{equation}
This formulation of the pressure gradient is characterised by the appearance of
a term depending on the sea surface height only
(last term on the right hand side of expression \autoref{eq:SCOORD_grad_p_2}).
This term will be loosely termed \textit{surface pressure gradient} whereas
the first term will be termed the \textit{hydrostatic pressure gradient} by analogy to
the $z$-coordinate formulation.
In fact, the true surface pressure gradient is $1/\rho_o \nabla (\rho \eta)$,
and $\eta$ is implicitly included in the computation of $p_h'$ through the upper bound of the vertical integration.

$\bullet$ \textbf{The other terms of the momentum equation}

The coriolis and forcing terms as well as the the vertical physics remain unchanged as
they involve neither time nor space derivatives.
The form of the lateral physics is discussed in \autoref{apdx:DIFFOPERS}.

$\bullet$ \textbf{Full momentum equation}

To sum up, in a curvilinear $s$-coordinate system,
the vector invariant momentum equation solved by the model has the same mathematical expression as
the one in a curvilinear $z-$coordinate, except for the pressure gradient term:
\begin{subequations}
  \label{eq:SCOORD_dyn_vect}
  \begin{multline}
    \label{eq:SCOORD_PE_dyn_vect_u}
    \frac{\partial u}{\partial t}=
    +   \left( {\zeta +f} \right)\,v
    -   \frac{1}{2\,e_1} \frac{\partial}{\partial i} \left(  u^2+v^2   \right)
    -   \frac{1}{e_3} \omega \frac{\partial u}{\partial k}       \\
    -   \frac{1}{e_1 } \left(    \frac{\partial p_h'}{\partial i} + g\; d  \; \frac{\partial z}{\partial i}    \right)
    -   \frac{g}{e_1 } \frac{\partial \eta}{\partial i}
    +   D_u^{\vect{U}}  +   F_u^{\vect{U}} ,
  \end{multline}
  \begin{multline}
    \label{eq:SCOORD_dyn_vect_v}
    \frac{\partial v}{\partial t}=
    -   \left( {\zeta +f} \right)\,u
    -   \frac{1}{2\,e_2 }\frac{\partial }{\partial j}\left(  u^2+v^2  \right)
    -   \frac{1}{e_3 } \omega \frac{\partial v}{\partial k}         \\
    -   \frac{1}{e_2 } \left(    \frac{\partial p_h'}{\partial j} + g\; d  \; \frac{\partial z}{\partial j}    \right)
    -   \frac{g}{e_2 } \frac{\partial \eta}{\partial j}
    +  D_v^{\vect{U}}  +   F_v^{\vect{U}} .
  \end{multline}
\end{subequations}
whereas the flux form momentum equation differs from it by
the formulation of both the time derivative and the pressure gradient term:
\begin{subequations}
  \label{eq:SCOORD_dyn_flux}
  \begin{multline}
    \label{eq:SCOORD_PE_dyn_flux_u}
    \frac{1}{e_3} \frac{\partial \left(  e_3\,u  \right) }{\partial t} =
    - \nabla \cdot \left(   {{\mathrm {\mathbf U}}\,u}   \right)
    +   \left\{ {f + \frac{1}{e_1 e_2 }\left(    v  \;\frac{\partial e_2 }{\partial i}
          -u  \;\frac{\partial e_1 }{\partial j}            \right)} \right\} \,v     \\
    -   \frac{1}{e_1 } \left(    \frac{\partial p_h'}{\partial i} + g\; d  \; \frac{\partial z}{\partial i}    \right)
    -   \frac{g}{e_1 } \frac{\partial \eta}{\partial i}
    +   D_u^{\vect{U}}  +   F_u^{\vect{U}} ,
  \end{multline}
  \begin{multline}
    \label{eq:SCOORD_dyn_flux_v}
    \frac{1}{e_3}\frac{\partial \left(  e_3\,v  \right) }{\partial t}=
    -  \nabla \cdot \left(   {{\mathrm {\mathbf U}}\,v}   \right)
    -   \left\{ {f + \frac{1}{e_1 e_2 }\left(    v  \;\frac{\partial e_2 }{\partial i}
          -u  \;\frac{\partial e_1 }{\partial j}            \right)} \right\} \,u     \\
    -   \frac{1}{e_2 } \left(    \frac{\partial p_h'}{\partial j} + g\; d  \; \frac{\partial z}{\partial j}    \right)
    -   \frac{g}{e_2 } \frac{\partial \eta}{\partial j}
    +  D_v^{\vect{U}}  +   F_v^{\vect{U}} .
  \end{multline}
\end{subequations}
Both formulation share the same hydrostatic pressure balance expressed in terms of
hydrostatic pressure and density anomalies, $p_h'$ and $d=( \frac{\rho}{\rho_o}-1 )$:
\begin{equation}
  \label{eq:SCOORD_dyn_zph}
  \frac{\partial p_h'}{\partial k} = - d \, g \, e_3 .
\end{equation}

It is important to realize that the change in coordinate system has only concerned the position on the vertical.
It has not affected (\textbf{i},\textbf{j},\textbf{k}), the orthogonal curvilinear set of unit vectors.
($u$,$v$) are always horizontal velocities so that their evolution is driven by \emph{horizontal} forces,
in particular the pressure gradient.
By contrast, $\omega$ is not $w$, the third component of the velocity, but the dia-surface velocity component,
\ie\ the volume flux across the moving $s$-surfaces per unit horizontal area.

%% =================================================================================================
\section{Tracer equation}
\label{sec:SCOORD_tracer}

The tracer equation is obtained using the same calculation as for the continuity equation and then
regrouping the time derivative terms in the left hand side :

\begin{multline}
  \label{eq:SCOORD_tracer}
  \frac{1}{e_3} \frac{\partial \left(  e_3 T  \right)}{\partial t}
  = -\frac{1}{e_1 \,e_2 \,e_3}
  \left[           \frac{\partial }{\partial i} \left( {e_2 \,e_3 \;Tu} \right)
    +   \frac{\partial }{\partial j} \left( {e_1 \,e_3 \;Tv} \right)               \right]       \\
  -  \frac{1}{e_3}  \frac{\partial }{\partial k} \left(                   Tw  \right)
  +  D^{T} +F^{T}
\end{multline}

The expression for the advection term is a straight consequence of (\autoref{eq:SCOORD_sco_Continuity}),
the expression of the 3D divergence in the $s-$coordinates established above.

\subinc{\input{../../global/epilogue}}

\end{document}
