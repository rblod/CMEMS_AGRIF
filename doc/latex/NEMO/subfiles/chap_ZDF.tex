\documentclass[../main/NEMO_manual]{subfiles}

\begin{document}
% ================================================================
% Chapter  Vertical Ocean Physics (ZDF)
% ================================================================
\chapter{Vertical Ocean Physics (ZDF)}
\label{chap:ZDF}

\minitoc

%gm% Add here a small introduction to ZDF and naming of the different physics (similar to what have been written for TRA and DYN.

\newpage

% ================================================================
% Vertical Mixing
% ================================================================
\section{Vertical mixing}
\label{sec:ZDF_zdf}

The discrete form of the ocean subgrid scale physics has been presented in
\autoref{sec:TRA_zdf} and \autoref{sec:DYN_zdf}.
At the surface and bottom boundaries, the turbulent fluxes of momentum, heat and salt have to be defined.
At the surface they are prescribed from the surface forcing (see \autoref{chap:SBC}),
while at the bottom they are set to zero for heat and salt,
unless a geothermal flux forcing is prescribed as a bottom boundary condition (\ie \key{trabbl} defined,
see \autoref{subsec:TRA_bbc}), and specified through a bottom friction parameterisation for momentum
(see \autoref{sec:ZDF_bfr}).

In this section we briefly discuss the various choices offered to compute the vertical eddy viscosity and
diffusivity coefficients, $A_u^{vm}$ , $A_v^{vm}$ and $A^{vT}$ ($A^{vS}$), defined at $uw$-, $vw$- and $w$- points,
respectively (see \autoref{sec:TRA_zdf} and \autoref{sec:DYN_zdf}).
These coefficients can be assumed to be either constant, or a function of the local Richardson number,
or computed from a turbulent closure model (either TKE or GLS formulation).
The computation of these coefficients is initialized in the \mdl{zdfini} module and performed in
the \mdl{zdfric}, \mdl{zdftke} or \mdl{zdfgls} modules.
The trends due to the vertical momentum and tracer diffusion, including the surface forcing,
are computed and added to the general trend in the \mdl{dynzdf} and \mdl{trazdf} modules, respectively. 
These trends can be computed using either a forward time stepping scheme
(namelist parameter \np{ln\_zdfexp}\forcode{ = .true.}) or a backward time stepping scheme
(\np{ln\_zdfexp}\forcode{ = .false.}) depending on the magnitude of the mixing coefficients,
and thus of the formulation used (see \autoref{chap:STP}).

% -------------------------------------------------------------------------------------------------------------
%        Constant 
% -------------------------------------------------------------------------------------------------------------
\subsection{Constant (\protect\key{zdfcst})}
\label{subsec:ZDF_cst}
%--------------------------------------------namzdf---------------------------------------------------------

\nlst{namzdf}
%--------------------------------------------------------------------------------------------------------------

Options are defined through the \ngn{namzdf} namelist variables.
When \key{zdfcst} is defined, the momentum and tracer vertical eddy coefficients are set to
constant values over the whole ocean.
This is the crudest way to define the vertical ocean physics.
It is recommended that this option is only used in process studies, not in basin scale simulations.
Typical values used in this case are:
\begin{align*}
  A_u^{vm} = A_v^{vm} &= 1.2\ 10^{-4}~m^2.s^{-1} 	\\
  A^{vT} = A^{vS} &= 1.2\ 10^{-5}~m^2.s^{-1}
\end{align*}

These values are set through the \np{rn\_avm0} and \np{rn\_avt0} namelist parameters. 
In all cases, do not use values smaller that those associated with the molecular viscosity and diffusivity,
that is $\sim10^{-6}~m^2.s^{-1}$ for momentum, $\sim10^{-7}~m^2.s^{-1}$ for temperature and
$\sim10^{-9}~m^2.s^{-1}$ for salinity.

% -------------------------------------------------------------------------------------------------------------
%        Richardson Number Dependent
% -------------------------------------------------------------------------------------------------------------
\subsection{Richardson number dependent (\protect\key{zdfric})}
\label{subsec:ZDF_ric}

%--------------------------------------------namric---------------------------------------------------------

\nlst{namzdf_ric}
%--------------------------------------------------------------------------------------------------------------

When \key{zdfric} is defined, a local Richardson number dependent formulation for the vertical momentum and
tracer eddy coefficients is set through the \ngn{namzdf\_ric} namelist variables.
The vertical mixing coefficients are diagnosed from the large scale variables computed by the model. 
\textit{In situ} measurements have been used to link vertical turbulent activity to large scale ocean structures.
The hypothesis of a mixing mainly maintained by the growth of Kelvin-Helmholtz like instabilities leads to
a dependency between the vertical eddy coefficients and the local Richardson number
(\ie the ratio of stratification to vertical shear).
Following \citet{Pacanowski_Philander_JPO81}, the following formulation has been implemented:
\[
  % \label{eq:zdfric}
  \left\{
    \begin{aligned}
      A^{vT} &= \frac {A_{ric}^{vT}}{\left( 1+a \; Ri \right)^n} + A_b^{vT}       \\
      A^{vm} &= \frac{A^{vT}        }{\left( 1+ a \;Ri  \right)   } + A_b^{vm}
    \end{aligned}
  \right.
\]
where $Ri = N^2 / \left(\partial_z \textbf{U}_h \right)^2$ is the local Richardson number,
$N$ is the local Brunt-Vais\"{a}l\"{a} frequency (see \autoref{subsec:TRA_bn2}), 
$A_b^{vT} $ and $A_b^{vm}$ are the constant background values set as in the constant case
(see \autoref{subsec:ZDF_cst}), and $A_{ric}^{vT} = 10^{-4}~m^2.s^{-1}$ is the maximum value that
can be reached by the coefficient when $Ri\leq 0$, $a=5$ and $n=2$.
The last three values can be modified by setting the \np{rn\_avmri}, \np{rn\_alp} and
\np{nn\_ric} namelist parameters, respectively.

A simple mixing-layer model to transfer and dissipate the atmospheric forcings
(wind-stress and buoyancy fluxes) can be activated setting the \np{ln\_mldw}\forcode{ = .true.} in the namelist.

In this case, the local depth of turbulent wind-mixing or "Ekman depth" $h_{e}(x,y,t)$ is evaluated and
the vertical eddy coefficients prescribed within this layer.

This depth is assumed proportional to the "depth of frictional influence" that is limited by rotation:
\[
  h_{e} = Ek \frac {u^{*}} {f_{0}}
\]
where, $Ek$ is an empirical parameter, $u^{*}$ is the friction velocity and $f_{0}$ is the Coriolis parameter.

In this similarity height relationship, the turbulent friction velocity:
\[
  u^{*} = \sqrt \frac {|\tau|} {\rho_o}
\]
is computed from the wind stress vector $|\tau|$ and the reference density $ \rho_o$.
The final $h_{e}$ is further constrained by the adjustable bounds \np{rn\_mldmin} and \np{rn\_mldmax}.
Once $h_{e}$ is computed, the vertical eddy coefficients within $h_{e}$ are set to
the empirical values \np{rn\_wtmix} and \np{rn\_wvmix} \citep{Lermusiaux2001}.

% -------------------------------------------------------------------------------------------------------------
%        TKE Turbulent Closure Scheme 
% -------------------------------------------------------------------------------------------------------------
\subsection{TKE turbulent closure scheme (\protect\key{zdftke})}
\label{subsec:ZDF_tke}

%--------------------------------------------namzdf_tke--------------------------------------------------

\nlst{namzdf_tke}
%--------------------------------------------------------------------------------------------------------------

The vertical eddy viscosity and diffusivity coefficients are computed from a TKE turbulent closure model based on
a prognostic equation for $\bar{e}$, the turbulent kinetic energy,
and a closure assumption for the turbulent length scales.
This turbulent closure model has been developed by \citet{Bougeault1989} in the atmospheric case,
adapted by \citet{Gaspar1990} for the oceanic case, and embedded in OPA, the ancestor of NEMO,
by \citet{Blanke1993} for equatorial Atlantic simulations.
Since then, significant modifications have been introduced by \citet{Madec1998} in both the implementation and
the formulation of the mixing length scale.
The time evolution of $\bar{e}$ is the result of the production of $\bar{e}$ through vertical shear,
its destruction through stratification, its vertical diffusion, and its dissipation of \citet{Kolmogorov1942} type:
\begin{equation}
  \label{eq:zdftke_e}
  \frac{\partial \bar{e}}{\partial t} =
  \frac{K_m}{{e_3}^2 }\;\left[ {\left( {\frac{\partial u}{\partial k}} \right)^2
      +\left( {\frac{\partial v}{\partial k}} \right)^2} \right]
  -K_\rho\,N^2
  +\frac{1}{e_3}	\;\frac{\partial }{\partial k}\left[ {\frac{A^{vm}}{e_3 }
      \;\frac{\partial \bar{e}}{\partial k}} \right]
  - c_\epsilon \;\frac{\bar {e}^{3/2}}{l_\epsilon }
\end{equation}
\[
  % \label{eq:zdftke_kz}
  \begin{split}
    K_m &= C_k\  l_k\  \sqrt {\bar{e}\; }  	\\
    K_\rho &= A^{vm} / P_{rt}
  \end{split}
\]
where $N$ is the local Brunt-Vais\"{a}l\"{a} frequency (see \autoref{subsec:TRA_bn2}), 
$l_{\epsilon }$ and $l_{\kappa }$ are the dissipation and mixing length scales, 
$P_{rt}$ is the Prandtl number, $K_m$ and $K_\rho$ are the vertical eddy viscosity and diffusivity coefficients.
The constants $C_k =  0.1$ and $C_\epsilon = \sqrt {2} /2$ $\approx 0.7$ are designed to deal with
vertical mixing at any depth \citep{Gaspar1990}. 
They are set through namelist parameters \np{nn\_ediff} and \np{nn\_ediss}.
$P_{rt}$ can be set to unity or, following \citet{Blanke1993}, be a function of the local Richardson number, $R_i$:
\begin{align*}
  % \label{eq:prt}
  P_{rt} =
  \begin{cases}
    \ \ \ 1 &      \text{if $\ R_i \leq 0.2$} 	\\
    5\,R_i &      \text{if $\ 0.2 \leq R_i \leq 2$} 	\\
    \ \ 10 &      \text{if $\ 2 \leq R_i$}
  \end{cases}
\end{align*}
Options are defined through the  \ngn{namzdfy\_tke} namelist variables.
The choice of $P_{rt}$ is controlled by the \np{nn\_pdl} namelist variable.

At the sea surface, the value of $\bar{e}$ is prescribed from the wind stress field as
$\bar{e}_o = e_{bb} |\tau| / \rho_o$, with $e_{bb}$ the \np{rn\_ebb} namelist parameter.
The default value of $e_{bb}$ is 3.75. \citep{Gaspar1990}), however a much larger value can be used when
taking into account the surface wave breaking (see below Eq. \autoref{eq:ZDF_Esbc}).
The bottom value of TKE is assumed to be equal to the value of the level just above.
The time integration of the $\bar{e}$ equation may formally lead to negative values because
the numerical scheme does not ensure its positivity.
To overcome this problem, a cut-off in the minimum value of $\bar{e}$ is used (\np{rn\_emin} namelist parameter).
Following \citet{Gaspar1990}, the cut-off value is set to $\sqrt{2}/2~10^{-6}~m^2.s^{-2}$.
This allows the subsequent formulations to match that of \citet{Gargett1984} for the diffusion in
the thermocline and deep ocean :  $K_\rho = 10^{-3} / N$.
In addition, a cut-off is applied on $K_m$ and $K_\rho$ to avoid numerical instabilities associated with
too weak vertical diffusion.
They must be specified at least larger than the molecular values, and are set through \np{rn\_avm0} and
\np{rn\_avt0} (namzdf namelist, see \autoref{subsec:ZDF_cst}).

\subsubsection{Turbulent length scale}

For computational efficiency, the original formulation of the turbulent length scales proposed by
\citet{Gaspar1990} has been simplified.
Four formulations are proposed, the choice of which is controlled by the \np{nn\_mxl} namelist parameter.
The first two are based on the following first order approximation \citep{Blanke1993}:
\begin{equation}
  \label{eq:tke_mxl0_1}
  l_k = l_\epsilon = \sqrt {2 \bar{e}\; } / N
\end{equation}
which is valid in a stable stratified region with constant values of the Brunt-Vais\"{a}l\"{a} frequency.
The resulting length scale is bounded by the distance to the surface or to the bottom
(\np{nn\_mxl}\forcode{ = 0}) or by the local vertical scale factor (\np{nn\_mxl}\forcode{ = 1}).
\citet{Blanke1993} notice that this simplification has two major drawbacks:
it makes no sense for locally unstable stratification and the computation no longer uses all
the information contained in the vertical density profile.
To overcome these drawbacks, \citet{Madec1998} introduces the \np{nn\_mxl}\forcode{ = 2..3} cases,
which add an extra assumption concerning the vertical gradient of the computed length scale.
So, the length scales are first evaluated as in \autoref{eq:tke_mxl0_1} and then bounded such that:
\begin{equation}
  \label{eq:tke_mxl_constraint}
  \frac{1}{e_3 }\left| {\frac{\partial l}{\partial k}} \right| \leq 1
  \qquad \text{with }\  l =  l_k = l_\epsilon
\end{equation}
\autoref{eq:tke_mxl_constraint} means that the vertical variations of the length scale cannot be larger than
the variations of depth.
It provides a better approximation of the \citet{Gaspar1990} formulation while being much less 
time consuming.
In particular, it allows the length scale to be limited not only by the distance to the surface or
to the ocean bottom but also by the distance to a strongly stratified portion of the water column such as
the thermocline (\autoref{fig:mixing_length}).
In order to impose the \autoref{eq:tke_mxl_constraint} constraint, we introduce two additional length scales:
$l_{up}$ and $l_{dwn}$, the upward and downward length scales, and
evaluate the dissipation and mixing length scales as
(and note that here we use numerical indexing):
%>>>>>>>>>>>>>>>>>>>>>>>>>>>>
\begin{figure}[!t]
  \begin{center}
    \includegraphics[width=1.00\textwidth]{Fig_mixing_length}
    \caption{
      \protect\label{fig:mixing_length}
      Illustration of the mixing length computation.
    }
  \end{center}
\end{figure}
%>>>>>>>>>>>>>>>>>>>>>>>>>>>>
\[
  % \label{eq:tke_mxl2}
  \begin{aligned}
    l_{up\ \ }^{(k)} &= \min \left(  l^{(k)} \ , \ l_{up}^{(k+1)} + e_{3t}^{(k)}\ \ \ \;  \right)
    \quad &\text{ from $k=1$ to $jpk$ }\ \\
    l_{dwn}^{(k)} &= \min \left(  l^{(k)} \ , \ l_{dwn}^{(k-1)} + e_{3t}^{(k-1)}  \right)
    \quad &\text{ from $k=jpk$ to $1$ }\ \\
  \end{aligned}
\]
where $l^{(k)}$ is computed using \autoref{eq:tke_mxl0_1}, \ie $l^{(k)} = \sqrt {2 {\bar e}^{(k)} / {N^2}^{(k)} }$.

In the \np{nn\_mxl}\forcode{ = 2} case, the dissipation and mixing length scales take the same value:
$ l_k=  l_\epsilon = \min \left(\ l_{up} \;,\;  l_{dwn}\ \right)$, while in the \np{nn\_mxl}\forcode{ = 3} case,
the dissipation and mixing turbulent length scales are give as in \citet{Gaspar1990}:
\[
  % \label{eq:tke_mxl_gaspar}
  \begin{aligned}
    & l_k          = \sqrt{\  l_{up} \ \ l_{dwn}\ }  	\\
    & l_\epsilon = \min \left(\ l_{up} \;,\;  l_{dwn}\ \right)
  \end{aligned}
\]

At the ocean surface, a non zero length scale is set through the  \np{rn\_mxl0} namelist parameter.
Usually the surface scale is given by $l_o = \kappa \,z_o$ where $\kappa = 0.4$ is von Karman's constant and
$z_o$ the roughness parameter of the surface.
Assuming $z_o=0.1$~m \citep{Craig_Banner_JPO94} leads to a 0.04~m, the default value of \np{rn\_mxl0}.
In the ocean interior a minimum length scale is set to recover the molecular viscosity when
$\bar{e}$ reach its minimum value ($1.10^{-6}= C_k\, l_{min} \,\sqrt{\bar{e}_{min}}$ ).

\subsubsection{Surface wave breaking parameterization}
%-----------------------------------------------------------------------%

Following \citet{Mellor_Blumberg_JPO04}, the TKE turbulence closure model has been modified to
include the effect of surface wave breaking energetics.
This results in a reduction of summertime surface temperature when the mixed layer is relatively shallow.
The \citet{Mellor_Blumberg_JPO04} modifications acts on surface length scale and TKE values and
air-sea drag coefficient. 
The latter concerns the bulk formulea and is not discussed here. 

Following \citet{Craig_Banner_JPO94}, the boundary condition on surface TKE value is :
\begin{equation}
  \label{eq:ZDF_Esbc}
  \bar{e}_o = \frac{1}{2}\,\left(  15.8\,\alpha_{CB} \right)^{2/3} \,\frac{|\tau|}{\rho_o}
\end{equation}
where $\alpha_{CB}$ is the \citet{Craig_Banner_JPO94} constant of proportionality which depends on the ''wave age'',
ranging from 57 for mature waves to 146 for younger waves \citep{Mellor_Blumberg_JPO04}. 
The boundary condition on the turbulent length scale follows the Charnock's relation:
\begin{equation}
  \label{eq:ZDF_Lsbc}
  l_o = \kappa \beta \,\frac{|\tau|}{g\,\rho_o}
\end{equation}
where $\kappa=0.40$ is the von Karman constant, and $\beta$ is the Charnock's constant.
\citet{Mellor_Blumberg_JPO04} suggest $\beta = 2.10^{5}$ the value chosen by
\citet{Stacey_JPO99} citing observation evidence, and
$\alpha_{CB} = 100$ the Craig and Banner's value.
As the surface boundary condition on TKE is prescribed through $\bar{e}_o = e_{bb} |\tau| / \rho_o$,
with $e_{bb}$ the \np{rn\_ebb} namelist parameter, setting \np{rn\_ebb}\forcode{ = 67.83} corresponds 
to $\alpha_{CB} = 100$.
Further setting  \np{ln\_mxl0} to true applies \autoref{eq:ZDF_Lsbc} as surface boundary condition on length scale,
with $\beta$ hard coded to the Stacey's value.
Note that a minimal threshold of \np{rn\_emin0}$=10^{-4}~m^2.s^{-2}$ (namelist parameters) is applied on
surface $\bar{e}$ value.

\subsubsection{Langmuir cells}
%--------------------------------------%

Langmuir circulations (LC) can be described as ordered large-scale vertical motions in
the surface layer of the oceans.
Although LC have nothing to do with convection, the circulation pattern is rather similar to
so-called convective rolls in the atmospheric boundary layer.
The detailed physics behind LC is described in, for example, \citet{Craik_Leibovich_JFM76}.
The prevailing explanation is that LC arise from a nonlinear interaction between the Stokes drift and
wind drift currents. 

Here we introduced in the TKE turbulent closure the simple parameterization of Langmuir circulations proposed by
\citep{Axell_JGR02} for a $k-\epsilon$ turbulent closure.
The parameterization, tuned against large-eddy simulation, includes the whole effect of LC in
an extra source terms of TKE, $P_{LC}$.
The presence of $P_{LC}$ in \autoref{eq:zdftke_e}, the TKE equation, is controlled by setting \np{ln\_lc} to
\forcode{.true.} in the namtke namelist.
 
By making an analogy with the characteristic convective velocity scale (\eg, \citet{D'Alessio_al_JPO98}),
$P_{LC}$ is assumed to be : 
\[
P_{LC}(z) = \frac{w_{LC}^3(z)}{H_{LC}}
\]
where $w_{LC}(z)$ is the vertical velocity profile of LC, and $H_{LC}$ is the LC depth.
With no information about the wave field, $w_{LC}$ is assumed to be proportional to 
the Stokes drift $u_s = 0.377\,\,|\tau|^{1/2}$, where $|\tau|$ is the surface wind stress module 
\footnote{Following \citet{Li_Garrett_JMR93}, the surface Stoke drift velocity may be expressed as
  $u_s =  0.016 \,|U_{10m}|$.
  Assuming an air density of $\rho_a=1.22 \,Kg/m^3$ and a drag coefficient of
  $1.5~10^{-3}$ give the expression used of $u_s$ as a function of the module of surface stress
}.
For the vertical variation, $w_{LC}$ is assumed to be zero at the surface as well as at
a finite depth $H_{LC}$ (which is often close to the mixed layer depth),
and simply varies as a sine function in between (a first-order profile for the Langmuir cell structures). 
The resulting expression for $w_{LC}$ is :
\[
  w_{LC}  =
  \begin{cases}
    c_{LC} \,u_s \,\sin(- \pi\,z / H_{LC} )    &      \text{if $-z \leq H_{LC}$} 	\\
    0                 				 &      \text{otherwise}
  \end{cases}
\]
where $c_{LC} = 0.15$ has been chosen by \citep{Axell_JGR02} as a good compromise to fit LES data.
The chosen value yields maximum vertical velocities $w_{LC}$ of the order of a few centimeters per second.
The value of $c_{LC}$ is set through the \np{rn\_lc} namelist parameter,
having in mind that it should stay between 0.15 and 0.54 \citep{Axell_JGR02}. 

The $H_{LC}$ is estimated in a similar way as the turbulent length scale of TKE equations:
$H_{LC}$ is depth to which a water parcel with kinetic energy due to Stoke drift can reach on its own by
converting its kinetic energy to potential energy, according to 
\[
- \int_{-H_{LC}}^0 { N^2\;z  \;dz} = \frac{1}{2} u_s^2
\]

\subsubsection{Mixing just below the mixed layer}
%--------------------------------------------------------------%

Vertical mixing parameterizations commonly used in ocean general circulation models tend to
produce mixed-layer depths that are too shallow during summer months and windy conditions.
This bias is particularly acute over the Southern Ocean.
To overcome this systematic bias, an ad hoc parameterization is introduced into the TKE scheme \cite{Rodgers_2014}. 
The parameterization is an empirical one, \ie not derived from theoretical considerations,
but rather is meant to account for observed processes that affect the density structure of 
the ocean’s planetary boundary layer that are not explicitly captured by default in the TKE scheme 
(\ie near-inertial oscillations and ocean swells and waves).

When using this parameterization (\ie when \np{nn\_etau}\forcode{ = 1}),
the TKE input to the ocean ($S$) imposed by the winds in the form of near-inertial oscillations,
swell and waves is parameterized by \autoref{eq:ZDF_Esbc} the standard TKE surface boundary condition,
plus a depth depend one given by:
\begin{equation}
  \label{eq:ZDF_Ehtau}
  S = (1-f_i) \; f_r \; e_s \; e^{-z / h_\tau}
\end{equation}
where $z$ is the depth, $e_s$ is TKE surface boundary condition, $f_r$ is the fraction of the surface TKE that
penetrate in the ocean, $h_\tau$ is a vertical mixing length scale that controls exponential shape of
the penetration, and $f_i$ is the ice concentration
(no penetration if $f_i=1$, that is if the ocean is entirely covered by sea-ice).
The value of $f_r$, usually a few percents, is specified through \np{rn\_efr} namelist parameter.
The vertical mixing length scale, $h_\tau$, can be set as a 10~m uniform value (\np{nn\_etau}\forcode{ = 0}) or
a latitude dependent value (varying from 0.5~m at the Equator to a maximum value of 30~m at high latitudes
(\np{nn\_etau}\forcode{ = 1}). 

Note that two other option existe, \np{nn\_etau}\forcode{ = 2..3}.
They correspond to applying \autoref{eq:ZDF_Ehtau} only at the base of the mixed layer,
or to using the high frequency part of the stress to evaluate the fraction of TKE that penetrate the ocean. 
Those two options are obsolescent features introduced for test purposes.
They will be removed in the next release. 

% from Burchard et al OM 2008 : 
% the most critical process not reproduced by statistical turbulence models is the activity of 
% internal waves and their interaction with turbulence. After the Reynolds decomposition, 
% internal waves are in principle included in the RANS equations, but later partially 
% excluded by the hydrostatic assumption and the model resolution. 
% Thus far, the representation of internal wave mixing in ocean models has been relatively crude 
% (\eg Mellor, 1989; Large et al., 1994; Meier, 2001; Axell, 2002; St. Laurent and Garrett, 2002).

% -------------------------------------------------------------------------------------------------------------
%        TKE discretization considerations
% -------------------------------------------------------------------------------------------------------------
\subsection{TKE discretization considerations (\protect\key{zdftke})}
\label{subsec:ZDF_tke_ene}

%>>>>>>>>>>>>>>>>>>>>>>>>>>>>
\begin{figure}[!t]
  \begin{center}
    \includegraphics[width=1.00\textwidth]{Fig_ZDF_TKE_time_scheme}
    \caption{
      \protect\label{fig:TKE_time_scheme}
      Illustration of the TKE time integration and its links to the momentum and tracer time integration.
    }
  \end{center}  
\end{figure}
%>>>>>>>>>>>>>>>>>>>>>>>>>>>>

The production of turbulence by vertical shear (the first term of the right hand side of
\autoref{eq:zdftke_e}) should balance the loss of kinetic energy associated with the vertical momentum diffusion
(first line in \autoref{eq:PE_zdf}).
To do so a special care have to be taken for both the time and space discretization of
the TKE equation \citep{Burchard_OM02,Marsaleix_al_OM08}.

Let us first address the time stepping issue. \autoref{fig:TKE_time_scheme} shows how
the two-level Leap-Frog time stepping of the momentum and tracer equations interplays with
the one-level forward time stepping of TKE equation.
With this framework, the total loss of kinetic energy (in 1D for the demonstration) due to
the vertical momentum diffusion is obtained by multiplying this quantity by $u^t$ and
summing the result vertically:   
\begin{equation}
  \label{eq:energ1}
  \begin{split}
    \int_{-H}^{\eta}  u^t \,\partial_z &\left( {K_m}^t \,(\partial_z u)^{t+\rdt}  \right) \,dz   \\
    &= \Bigl[  u^t \,{K_m}^t \,(\partial_z u)^{t+\rdt} \Bigr]_{-H}^{\eta}
    - \int_{-H}^{\eta}{ {K_m}^t \,\partial_z{u^t} \,\partial_z u^{t+\rdt} \,dz }
  \end{split}
\end{equation}
Here, the vertical diffusion of momentum is discretized backward in time with a coefficient, $K_m$,
known at time $t$ (\autoref{fig:TKE_time_scheme}), as it is required when using the TKE scheme
(see \autoref{sec:STP_forward_imp}).
The first term of the right hand side of \autoref{eq:energ1} represents the kinetic energy transfer at
the surface (atmospheric forcing) and at the bottom (friction effect).
The second term is always negative.
It is the dissipation rate of kinetic energy, and thus minus the shear production rate of $\bar{e}$.
\autoref{eq:energ1} implies that, to be energetically consistent,
the production rate of $\bar{e}$ used to compute $(\bar{e})^t$ (and thus ${K_m}^t$) should be expressed as
${K_m}^{t-\rdt}\,(\partial_z u)^{t-\rdt} \,(\partial_z u)^t$
(and not by the more straightforward $K_m \left( \partial_z u \right)^2$ expression taken at time $t$ or $t-\rdt$).

A similar consideration applies on the destruction rate of $\bar{e}$ due to stratification
(second term of the right hand side of \autoref{eq:zdftke_e}).
This term must balance the input of potential energy resulting from vertical mixing.
The rate of change of potential energy (in 1D for the demonstration) due vertical mixing is obtained by
multiplying vertical density diffusion tendency by $g\,z$ and and summing the result vertically:
\begin{equation}
  \label{eq:energ2}
  \begin{split}
    \int_{-H}^{\eta} g\,z\,\partial_z &\left( {K_\rho}^t \,(\partial_k \rho)^{t+\rdt}   \right) \,dz    \\
    &= \Bigl[  g\,z \,{K_\rho}^t \,(\partial_z \rho)^{t+\rdt} \Bigr]_{-H}^{\eta}
    - \int_{-H}^{\eta}{ g \,{K_\rho}^t \,(\partial_k \rho)^{t+\rdt} } \,dz   \\
    &= - \Bigl[  z\,{K_\rho}^t \,(N^2)^{t+\rdt} \Bigr]_{-H}^{\eta}
    + \int_{-H}^{\eta}{  \rho^{t+\rdt} \, {K_\rho}^t \,(N^2)^{t+\rdt} \,dz  }
  \end{split}
\end{equation}
where we use $N^2 = -g \,\partial_k \rho / (e_3 \rho)$. 
The first term of the right hand side of \autoref{eq:energ2} is always zero because
there is no diffusive flux through the ocean surface and bottom).
The second term is minus the destruction rate of  $\bar{e}$ due to stratification.
Therefore \autoref{eq:energ1} implies that, to be energetically consistent,
the product ${K_\rho}^{t-\rdt}\,(N^2)^t$ should be used in \autoref{eq:zdftke_e}, the TKE equation.

Let us now address the space discretization issue.
The vertical eddy coefficients are defined at $w$-point whereas the horizontal velocity components are in
the centre of the side faces of a $t$-box in staggered C-grid (\autoref{fig:cell}).
A space averaging is thus required to obtain the shear TKE production term.
By redoing the \autoref{eq:energ1} in the 3D case, it can be shown that the product of eddy coefficient by
the shear at $t$ and $t-\rdt$ must be performed prior to the averaging.
Furthermore, the possible time variation of $e_3$ (\key{vvl} case) have to be taken into account.

The above energetic considerations leads to the following final discrete form for the TKE equation:
\begin{equation}
  \label{eq:zdftke_ene}
  \begin{split}
    \frac { (\bar{e})^t - (\bar{e})^{t-\rdt} } {\rdt}  \equiv
    \Biggl\{ \Biggr.
    &\overline{ \left( \left(\overline{K_m}^{\,i+1/2}\right)^{t-\rdt} \,\frac{\delta_{k+1/2}[u^{t+\rdt}]}{{e_3u}^{t+\rdt} }
        \ \frac{\delta_{k+1/2}[u^ t         ]}{{e_3u}^ t          }  \right) }^{\,i} \\
    +&\overline{  \left( \left(\overline{K_m}^{\,j+1/2}\right)^{t-\rdt} \,\frac{\delta_{k+1/2}[v^{t+\rdt}]}{{e_3v}^{t+\rdt} }
        \ \frac{\delta_{k+1/2}[v^ t         ]}{{e_3v}^ t          }  \right) }^{\,j}
    \Biggr. \Biggr\}   \\
    %
    - &{K_\rho}^{t-\rdt}\,{(N^2)^t}    \\
    %
    +&\frac{1}{{e_3w}^{t+\rdt}}  \;\delta_{k+1/2} \left[   {K_m}^{t-\rdt} \,\frac{\delta_{k}[(\bar{e})^{t+\rdt}]} {{e_3w}^{t+\rdt}}   \right]   \\
    %
    - &c_\epsilon \; \left( \frac{\sqrt{\bar {e}}}{l_\epsilon}\right)^{t-\rdt}\,(\bar {e})^{t+\rdt}
  \end{split}
\end{equation}
where the last two terms in \autoref{eq:zdftke_ene} (vertical diffusion and Kolmogorov dissipation)
are time stepped using a backward scheme (see\autoref{sec:STP_forward_imp}).
Note that the Kolmogorov term has been linearized in time in order to render the implicit computation possible.
The restart of the TKE scheme requires the storage of $\bar {e}$, $K_m$, $K_\rho$ and $l_\epsilon$ as
they all appear in the right hand side of \autoref{eq:zdftke_ene}.
For the latter, it is in fact the ratio $\sqrt{\bar{e}}/l_\epsilon$ which is stored. 

% -------------------------------------------------------------------------------------------------------------
%        GLS Generic Length Scale Scheme 
% -------------------------------------------------------------------------------------------------------------
\subsection{GLS: Generic Length Scale (\protect\key{zdfgls})}
\label{subsec:ZDF_gls}

%--------------------------------------------namzdf_gls---------------------------------------------------------

\nlst{namzdf_gls}
%--------------------------------------------------------------------------------------------------------------

The Generic Length Scale (GLS) scheme is a turbulent closure scheme based on two prognostic equations:
one for the turbulent kinetic energy $\bar {e}$, and another for the generic length scale,
$\psi$ \citep{Umlauf_Burchard_JMS03, Umlauf_Burchard_CSR05}.
This later variable is defined as: $\psi = {C_{0\mu}}^{p} \ {\bar{e}}^{m} \ l^{n}$, 
where the triplet $(p, m, n)$ value given in Tab.\autoref{tab:GLS} allows to recover a number of
well-known turbulent closures ($k$-$kl$ \citep{Mellor_Yamada_1982}, $k$-$\epsilon$ \citep{Rodi_1987},
$k$-$\omega$ \citep{Wilcox_1988} among others \citep{Umlauf_Burchard_JMS03,Kantha_Carniel_CSR05}). 
The GLS scheme is given by the following set of equations:
\begin{equation}
  \label{eq:zdfgls_e}
  \frac{\partial \bar{e}}{\partial t} =
  \frac{K_m}{\sigma_e e_3 }\;\left[ {\left( \frac{\partial u}{\partial k} \right)^2
      +\left( \frac{\partial v}{\partial k} \right)^2} \right]
  -K_\rho \,N^2
  +\frac{1}{e_3}\,\frac{\partial}{\partial k} \left[ \frac{K_m}{e_3}\,\frac{\partial \bar{e}}{\partial k} \right]
  - \epsilon
\end{equation}

\[
  % \label{eq:zdfgls_psi}
  \begin{split}
    \frac{\partial \psi}{\partial t} =& \frac{\psi}{\bar{e}} \left\{
      \frac{C_1\,K_m}{\sigma_{\psi} {e_3}}\;\left[ {\left( \frac{\partial u}{\partial k} \right)^2
          +\left( \frac{\partial v}{\partial k} \right)^2} \right]
      - C_3 \,K_\rho\,N^2   - C_2 \,\epsilon \,Fw   \right\}             \\
    &+\frac{1}{e_3}  \;\frac{\partial }{\partial k}\left[ {\frac{K_m}{e_3 }
        \;\frac{\partial \psi}{\partial k}} \right]\;
  \end{split}
\]

\[
  % \label{eq:zdfgls_kz}
  \begin{split}
    K_m    &= C_{\mu} \ \sqrt {\bar{e}} \ l         \\
    K_\rho &= C_{\mu'}\ \sqrt {\bar{e}} \ l
  \end{split}
\]

\[
  % \label{eq:zdfgls_eps}
  {\epsilon} = C_{0\mu} \,\frac{\bar {e}^{3/2}}{l} \;
\]
where $N$ is the local Brunt-Vais\"{a}l\"{a} frequency (see \autoref{subsec:TRA_bn2}) and
$\epsilon$ the dissipation rate. 
The constants $C_1$, $C_2$, $C_3$, ${\sigma_e}$, ${\sigma_{\psi}}$ and the wall function ($Fw$) depends of
the choice of the turbulence model.
Four different turbulent models are pre-defined (Tab.\autoref{tab:GLS}).
They are made available through the \np{nn\_clo} namelist parameter. 

%--------------------------------------------------TABLE--------------------------------------------------
\begin{table}[htbp]
  \begin{center}
    % \begin{tabular}{cp{70pt}cp{70pt}cp{70pt}cp{70pt}cp{70pt}cp{70pt}c}
    \begin{tabular}{ccccc}
      &   $k-kl$   & $k-\epsilon$ & $k-\omega$ &   generic   \\
      % & \citep{Mellor_Yamada_1982} &  \citep{Rodi_1987}       & \citep{Wilcox_1988} &                 \\
      \hline
      \hline
      \np{nn\_clo}     & \textbf{0} &   \textbf{1}  &   \textbf{2}   &    \textbf{3}   \\
      \hline
      $( p , n , m )$	       &   ( 0 , 1 , 1 )   & ( 3 , 1.5 , -1 )   & ( -1 , 0.5 , -1 )    &  ( 2 , 1 , -0.67 )  \\
      $\sigma_k$      &    2.44         &     1.              &      2.                &      0.8          \\
      $\sigma_\psi$  &    2.44         &     1.3            &      2.                 &       1.07       \\
      $C_1$              &      0.9         &     1.44          &      0.555          &       1.           \\
      $C_2$              &      0.5         &     1.92          &      0.833          &       1.22       \\
      $C_3$              &      1.           &     1.              &      1.                &       1.           \\
      $F_{wall}$        &      Yes        &       --             &     --                  &      --          \\
      \hline
      \hline
    \end{tabular}
    \caption{
      \protect\label{tab:GLS}
      Set of predefined GLS parameters, or equivalently predefined turbulence models available with
      \protect\key{zdfgls} and controlled by the \protect\np{nn\_clos} namelist variable in \protect\ngn{namzdf\_gls}.
    }
  \end{center}
\end{table}
%--------------------------------------------------------------------------------------------------------------

In the Mellor-Yamada model, the negativity of $n$ allows to use a wall function to force the convergence of
the mixing length towards $K z_b$ ($K$: Kappa and $z_b$: rugosity length) value near physical boundaries
(logarithmic boundary layer law).
$C_{\mu}$ and $C_{\mu'}$ are calculated from stability function proposed by \citet{Galperin_al_JAS88},
or by \citet{Kantha_Clayson_1994} or one of the two functions suggested by \citet{Canuto_2001}
(\np{nn\_stab\_func}\forcode{ = 0..3}, resp.). 
The value of $C_{0\mu}$ depends of the choice of the stability function.

The surface and bottom boundary condition on both $\bar{e}$ and $\psi$ can be calculated thanks to Dirichlet or
Neumann condition through \np{nn\_tkebc\_surf} and \np{nn\_tkebc\_bot}, resp.
As for TKE closure, the wave effect on the mixing is considered when
\np{ln\_crban}\forcode{ = .true.} \citep{Craig_Banner_JPO94, Mellor_Blumberg_JPO04}.
The \np{rn\_crban} namelist parameter is $\alpha_{CB}$ in \autoref{eq:ZDF_Esbc} and
\np{rn\_charn} provides the value of $\beta$ in \autoref{eq:ZDF_Lsbc}. 

The $\psi$ equation is known to fail in stably stratified flows, and for this reason
almost all authors apply a clipping of the length scale as an \textit{ad hoc} remedy.
With this clipping, the maximum permissible length scale is determined by $l_{max} = c_{lim} \sqrt{2\bar{e}}/ N$.
A value of $c_{lim} = 0.53$ is often used \citep{Galperin_al_JAS88}.
\cite{Umlauf_Burchard_CSR05} show that the value of the clipping factor is of crucial importance for
the entrainment depth predicted in stably stratified situations,
and that its value has to be chosen in accordance with the algebraic model for the turbulent fluxes.
The clipping is only activated if \np{ln\_length\_lim}\forcode{ = .true.},
and the $c_{lim}$ is set to the \np{rn\_clim\_galp} value.

The time and space discretization of the GLS equations follows the same energetic consideration as for
the TKE case described in \autoref{subsec:ZDF_tke_ene} \citep{Burchard_OM02}.
Examples of performance of the 4 turbulent closure scheme can be found in \citet{Warner_al_OM05}.

% -------------------------------------------------------------------------------------------------------------
%        OSM OSMOSIS BL Scheme 
% -------------------------------------------------------------------------------------------------------------
\subsection{OSM: OSMOSIS boundary layer scheme (\protect\key{zdfosm})}
\label{subsec:ZDF_osm}

%--------------------------------------------namzdf_osm---------------------------------------------------------

\nlst{namzdf_osm}
%--------------------------------------------------------------------------------------------------------------

The OSMOSIS turbulent closure scheme is based on......   TBC

% ================================================================
% Convection
% ================================================================
\section{Convection}
\label{sec:ZDF_conv}

%--------------------------------------------namzdf--------------------------------------------------------

\nlst{namzdf}
%--------------------------------------------------------------------------------------------------------------

Static instabilities (\ie light potential densities under heavy ones) may occur at particular ocean grid points.
In nature, convective processes quickly re-establish the static stability of the water column.
These processes have been removed from the model via the hydrostatic assumption so they must be parameterized.
Three parameterisations are available to deal with convective processes:
a non-penetrative convective adjustment or an enhanced vertical diffusion,
or/and the use of a turbulent closure scheme.

% -------------------------------------------------------------------------------------------------------------
%       Non-Penetrative Convective Adjustment 
% -------------------------------------------------------------------------------------------------------------
\subsection[Non-penetrative convective adjmt (\protect\np{ln\_tranpc}\forcode{ = .true.})]
				{Non-penetrative convective adjustment (\protect\np{ln\_tranpc}\forcode{ = .true.})}
\label{subsec:ZDF_npc}

%--------------------------------------------namzdf--------------------------------------------------------

\nlst{namzdf}
%--------------------------------------------------------------------------------------------------------------

%>>>>>>>>>>>>>>>>>>>>>>>>>>>>
\begin{figure}[!htb]
  \begin{center}
    \includegraphics[width=0.90\textwidth]{Fig_npc}
    \caption{
      \protect\label{fig:npc}
      Example of an unstable density profile treated by the non penetrative convective adjustment algorithm.
      $1^{st}$ step: the initial profile is checked from the surface to the bottom.
      It is found to be unstable between levels 3 and 4.
      They are mixed.
      The resulting $\rho$ is still larger than $\rho$(5): levels 3 to 5 are mixed.
      The resulting $\rho$ is still larger than $\rho$(6): levels 3 to 6 are mixed.
      The $1^{st}$ step ends since the density profile is then stable below the level 3.
      $2^{nd}$ step: the new $\rho$ profile is checked following the same procedure as in $1^{st}$ step:
      levels 2 to 5 are mixed.
      The new density profile is checked.
      It is found stable: end of algorithm.
    }
  \end{center}
\end{figure}
%>>>>>>>>>>>>>>>>>>>>>>>>>>>>

Options are defined through the \ngn{namzdf} namelist variables.
The non-penetrative convective adjustment is used when \np{ln\_zdfnpc}\forcode{ = .true.}.
It is applied at each \np{nn\_npc} time step and mixes downwards instantaneously the statically unstable portion of
the water column, but only until the density structure becomes neutrally stable
(\ie until the mixed portion of the water column has \textit{exactly} the density of the water just below)
\citep{Madec_al_JPO91}.
The associated algorithm is an iterative process used in the following way (\autoref{fig:npc}):
starting from the top of the ocean, the first instability is found.
Assume in the following that the instability is located between levels $k$ and $k+1$.
The temperature and salinity in the two levels are vertically mixed, conserving the heat and salt contents of
the water column.
The new density is then computed by a linear approximation.
If the new density profile is still unstable between levels $k+1$ and $k+2$,
levels $k$, $k+1$ and $k+2$ are then mixed.
This process is repeated until stability is established below the level $k$
(the mixing process can go down to the ocean bottom).
The algorithm is repeated to check if the density profile between level $k-1$ and $k$ is unstable and/or
if there is no deeper instability.

This algorithm is significantly different from mixing statically unstable levels two by two.
The latter procedure cannot converge with a finite number of iterations for some vertical profiles while
the algorithm used in \NEMO converges for any profile in a number of iterations which is less than
the number of vertical levels.
This property is of paramount importance as pointed out by \citet{Killworth1989}:
it avoids the existence of permanent and unrealistic static instabilities at the sea surface.
This non-penetrative convective algorithm has been proved successful in studies of the deep water formation in
the north-western Mediterranean Sea \citep{Madec_al_JPO91, Madec_al_DAO91, Madec_Crepon_Bk91}.

The current implementation has been modified in order to deal with any non linear equation of seawater
(L. Brodeau, personnal communication).
Two main differences have been introduced compared to the original algorithm:
$(i)$ the stability is now checked using the Brunt-V\"{a}is\"{a}l\"{a} frequency 
(not the the difference in potential density); 
$(ii)$ when two levels are found unstable, their thermal and haline expansion coefficients are vertically mixed in
the same way their temperature and salinity has been mixed.
These two modifications allow the algorithm to perform properly and accurately with TEOS10 or EOS-80 without
having to recompute the expansion coefficients at each mixing iteration.

% -------------------------------------------------------------------------------------------------------------
%       Enhanced Vertical Diffusion 
% -------------------------------------------------------------------------------------------------------------
\subsection{Enhanced vertical diffusion (\protect\np{ln\_zdfevd}\forcode{ = .true.})}
\label{subsec:ZDF_evd}

%--------------------------------------------namzdf--------------------------------------------------------

\nlst{namzdf}
%--------------------------------------------------------------------------------------------------------------

Options are defined through the  \ngn{namzdf} namelist variables.
The enhanced vertical diffusion parameterisation is used when \np{ln\_zdfevd}\forcode{ = .true.}.
In this case, the vertical eddy mixing coefficients are assigned very large values
(a typical value is $10\;m^2s^{-1})$ in regions where the stratification is unstable
(\ie when $N^2$ the Brunt-Vais\"{a}l\"{a} frequency is negative) \citep{Lazar_PhD97, Lazar_al_JPO99}.
This is done either on tracers only (\np{nn\_evdm}\forcode{ = 0}) or
on both momentum and tracers (\np{nn\_evdm}\forcode{ = 1}).

In practice, where $N^2\leq 10^{-12}$, $A_T^{vT}$ and $A_T^{vS}$, and if \np{nn\_evdm}\forcode{ = 1},
the four neighbouring $A_u^{vm} \;\mbox{and}\;A_v^{vm}$ values also, are set equal to
the namelist parameter \np{rn\_avevd}.
A typical value for $rn\_avevd$ is between 1 and $100~m^2.s^{-1}$.
This parameterisation of convective processes is less time consuming than
the convective adjustment algorithm presented above when mixing both tracers and
momentum in the case of static instabilities.
It requires the use of an implicit time stepping on vertical diffusion terms
(\ie np{ln\_zdfexp}\forcode{ = .false.}).

Note that the stability test is performed on both \textit{before} and \textit{now} values of $N^2$.
This removes a potential source of divergence of odd and even time step in
a leapfrog environment \citep{Leclair_PhD2010} (see \autoref{sec:STP_mLF}).

% -------------------------------------------------------------------------------------------------------------
%       Turbulent Closure Scheme 
% -------------------------------------------------------------------------------------------------------------
\subsection[Turbulent closure scheme (\protect\key{zdf}\{tke,gls,osm\})]{Turbulent Closure Scheme (\protect\key{zdftke}, \protect\key{zdfgls} or \protect\key{zdfosm})}
\label{subsec:ZDF_tcs}

The turbulent closure scheme presented in \autoref{subsec:ZDF_tke} and \autoref{subsec:ZDF_gls}
(\key{zdftke} or \key{zdftke} is defined) in theory solves the problem of statically unstable density profiles.
In such a case, the term corresponding to the destruction of turbulent kinetic energy through stratification in
\autoref{eq:zdftke_e} or \autoref{eq:zdfgls_e} becomes a source term, since $N^2$ is negative. 
It results in large values of $A_T^{vT}$ and  $A_T^{vT}$, and also the four neighbouring $A_u^{vm} {and}\;A_v^{vm}$
(up to $1\;m^2s^{-1}$).
These large values restore the static stability of the water column in a way similar to that of
the enhanced vertical diffusion parameterisation (\autoref{subsec:ZDF_evd}).
However, in the vicinity of the sea surface (first ocean layer), the eddy coefficients computed by
the turbulent closure scheme do not usually exceed $10^{-2}m.s^{-1}$,
because the mixing length scale is bounded by the distance to the sea surface.
It can thus be useful to combine the enhanced vertical diffusion with the turbulent closure scheme,
\ie setting the \np{ln\_zdfnpc} namelist parameter to true and
defining the turbulent closure CPP key all together.

The KPP turbulent closure scheme already includes enhanced vertical diffusion in the case of convection,
as governed by the variables $bvsqcon$ and $difcon$ found in \mdl{zdfkpp},
therefore \np{ln\_zdfevd}\forcode{ = .false.} should be used with the KPP scheme.
% gm%  + one word on non local flux with KPP scheme trakpp.F90 module...

% ================================================================
% Double Diffusion Mixing
% ================================================================
\section{Double diffusion mixing (\protect\key{zdfddm})}
\label{sec:ZDF_ddm}

%-------------------------------------------namzdf_ddm-------------------------------------------------
%
%\nlst{namzdf_ddm}
%--------------------------------------------------------------------------------------------------------------

Options are defined through the  \ngn{namzdf\_ddm} namelist variables.
Double diffusion occurs when relatively warm, salty water overlies cooler, fresher water, or vice versa.
The former condition leads to salt fingering and the latter to diffusive convection.
Double-diffusive phenomena contribute to diapycnal mixing in extensive regions of the ocean.
\citet{Merryfield1999} include a parameterisation of such phenomena in a global ocean model and show that 
it leads to relatively minor changes in circulation but exerts significant regional influences on
temperature and salinity.
This parameterisation has been introduced in \mdl{zdfddm} module and is controlled by the \key{zdfddm} CPP key.

Diapycnal mixing of S and T are described by diapycnal diffusion coefficients 
\begin{align*}
  % \label{eq:zdfddm_Kz}
  &A^{vT} = A_o^{vT}+A_f^{vT}+A_d^{vT}	\\
  &A^{vS} = A_o^{vS}+A_f^{vS}+A_d^{vS}
\end{align*}
where subscript $f$ represents mixing by salt fingering, $d$ by diffusive convection,
and $o$ by processes other than double diffusion.
The rates of double-diffusive mixing depend on the buoyancy ratio
$R_\rho = \alpha \partial_z T / \beta \partial_z S$, where $\alpha$ and $\beta$ are coefficients of
thermal expansion and saline contraction (see \autoref{subsec:TRA_eos}).
To represent mixing of $S$ and $T$ by salt fingering, we adopt the diapycnal diffusivities suggested by Schmitt
(1981):
\begin{align}
  \label{eq:zdfddm_f}
  A_f^{vS} &=
             \begin{cases}
               \frac{A^{\ast v}}{1+(R_\rho / R_c)^n   } &\text{if  $R_\rho > 1$ and $N^2>0$ } \\
               0 				  					    &\text{otherwise}
             \end{cases}
  \\ 		    \label{eq:zdfddm_f_T}
  A_f^{vT} &= 0.7 \ A_f^{vS} / R_\rho 
\end{align}

%>>>>>>>>>>>>>>>>>>>>>>>>>>>>
\begin{figure}[!t]
  \begin{center}
    \includegraphics[width=0.99\textwidth]{Fig_zdfddm}
    \caption{
      \protect\label{fig:zdfddm}
      From \citet{Merryfield1999} :
      (a) Diapycnal diffusivities $A_f^{vT}$ and $A_f^{vS}$ for temperature and salt in regions of salt fingering.
      Heavy curves denote $A^{\ast v} = 10^{-3}~m^2.s^{-1}$ and thin curves $A^{\ast v} = 10^{-4}~m^2.s^{-1}$;
      (b) diapycnal diffusivities $A_d^{vT}$ and $A_d^{vS}$ for temperature and salt in regions of
      diffusive convection.
      Heavy curves denote the Federov parameterisation and thin curves the Kelley parameterisation.
      The latter is not implemented in \NEMO.
    }
  \end{center}
\end{figure}
%>>>>>>>>>>>>>>>>>>>>>>>>>>>>

The factor 0.7 in \autoref{eq:zdfddm_f_T} reflects the measured ratio $\alpha F_T /\beta F_S \approx  0.7$ of
buoyancy flux of heat to buoyancy flux of salt (\eg, \citet{McDougall_Taylor_JMR84}).
Following  \citet{Merryfield1999}, we adopt $R_c = 1.6$, $n = 6$, and $A^{\ast v} = 10^{-4}~m^2.s^{-1}$.

To represent mixing of S and T by diffusive layering,  the diapycnal diffusivities suggested by
Federov (1988) is used: 
\begin{align}
  % \label{eq:zdfddm_d}
  A_d^{vT} &=
             \begin{cases}
               1.3635 \, \exp{\left( 4.6\, \exp{ \left[  -0.54\,( R_{\rho}^{-1} - 1 )  \right] }    \right)}
               &\text{if  $0<R_\rho < 1$ and $N^2>0$ } \\
               0 								&\text{otherwise}
             \end{cases}
                                       \nonumber \\
  \label{eq:zdfddm_d_S}
  A_d^{vS} &=
             \begin{cases}
               A_d^{vT}\ \left( 1.85\,R_{\rho} - 0.85 \right) &\text{if  $0.5 \leq R_\rho<1$ and $N^2>0$ } \\
               A_d^{vT} \ 0.15 \ R_\rho               &\text{if  $\ \ 0 < R_\rho<0.5$ and $N^2>0$ } \\
               0 								&\text{otherwise}
             \end{cases}
\end{align}

The dependencies of \autoref{eq:zdfddm_f} to \autoref{eq:zdfddm_d_S} on $R_\rho$ are illustrated in
\autoref{fig:zdfddm}.
Implementing this requires computing $R_\rho$ at each grid point on every time step.
This is done in \mdl{eosbn2} at the same time as $N^2$ is computed.
This avoids duplication in the computation of $\alpha$ and $\beta$ (which is usually quite expensive).

% ================================================================
% Bottom Friction
% ================================================================
\section{Bottom and top friction (\protect\mdl{zdfbfr})}
\label{sec:ZDF_bfr}

%--------------------------------------------nambfr--------------------------------------------------------
%
%\nlst{nambfr}
%--------------------------------------------------------------------------------------------------------------

Options to define the top and bottom friction are defined through the \ngn{nambfr} namelist variables.
The bottom friction represents the friction generated by the bathymetry.
The top friction represents the friction generated by the ice shelf/ocean interface.
As the friction processes at the top and bottom are treated in similar way,
only the bottom friction is described in detail below.


Both the surface momentum flux (wind stress) and the bottom momentum flux (bottom friction) enter the equations as
a condition on the vertical diffusive flux.
For the bottom boundary layer, one has:
\[
  % \label{eq:zdfbfr_flux}
  A^{vm} \left( \partial {\textbf U}_h / \partial z \right) = {{\cal F}}_h^{\textbf U}
\]
where ${\cal F}_h^{\textbf U}$ is represents the downward flux of horizontal momentum outside
the logarithmic turbulent boundary layer (thickness of the order of 1~m in the ocean).
How ${\cal F}_h^{\textbf U}$ influences the interior depends on the vertical resolution of the model near
the bottom relative to the Ekman layer depth.
For example, in order to obtain an Ekman layer depth $d = \sqrt{2\;A^{vm}} / f = 50$~m,
one needs a vertical diffusion coefficient $A^{vm} = 0.125$~m$^2$s$^{-1}$
(for a Coriolis frequency $f = 10^{-4}$~m$^2$s$^{-1}$).
With a background diffusion coefficient $A^{vm} = 10^{-4}$~m$^2$s$^{-1}$, the Ekman layer depth is only 1.4~m. 
When the vertical mixing coefficient is this small, using a flux condition is equivalent to
entering the viscous forces (either wind stress or bottom friction) as a body force over the depth of the top or
bottom model layer.
To illustrate this, consider the equation for $u$ at $k$, the last ocean level:
\begin{equation}
  \label{eq:zdfbfr_flux2}
  \frac{\partial u_k}{\partial t} = \frac{1}{e_{3u}} \left[ \frac{A_{uw}^{vm}}{e_{3uw}} \delta_{k+1/2}\;[u] - {\cal F}^u_h \right] \approx - \frac{{\cal F}^u_{h}}{e_{3u}}
\end{equation}
If the bottom layer thickness is 200~m, the Ekman transport will be distributed over that depth.
On the other hand, if the vertical resolution is high (1~m or less) and a turbulent closure model is used,
the turbulent Ekman layer will be represented explicitly by the model.
However, the logarithmic layer is never represented in current primitive equation model applications:
it is \emph{necessary} to parameterize the flux ${\cal F}^u_h $.
Two choices are available in \NEMO: a linear and a quadratic bottom friction.
Note that in both cases, the rotation between the interior velocity and the bottom friction is neglected in
the present release of \NEMO.

In the code, the bottom friction is imposed by adding the trend due to the bottom friction to
the general momentum trend in \mdl{dynbfr}.
For the time-split surface pressure gradient algorithm, the momentum trend due to
the barotropic component needs to be handled separately.
For this purpose it is convenient to compute and store coefficients which can be simply combined with
bottom velocities and geometric values to provide the momentum trend due to bottom friction.
These coefficients are computed in \mdl{zdfbfr} and generally take the form $c_b^{\textbf U}$ where:
\begin{equation}
  \label{eq:zdfbfr_bdef}
  \frac{\partial {\textbf U_h}}{\partial t} =
  - \frac{{\cal F}^{\textbf U}_{h}}{e_{3u}} = \frac{c_b^{\textbf U}}{e_{3u}} \;{\textbf U}_h^b
\end{equation}
where $\textbf{U}_h^b = (u_b\;,\;v_b)$ is the near-bottom, horizontal, ocean velocity.

% -------------------------------------------------------------------------------------------------------------
%       Linear Bottom Friction
% -------------------------------------------------------------------------------------------------------------
\subsection{Linear bottom friction (\protect\np{nn\_botfr}\forcode{ = 0..1})}
\label{subsec:ZDF_bfr_linear}

The linear bottom friction parameterisation (including the special case of a free-slip condition) assumes that
the bottom friction is proportional to the interior velocity (\ie the velocity of the last model level):
\[
  % \label{eq:zdfbfr_linear}
  {\cal F}_h^\textbf{U} = \frac{A^{vm}}{e_3} \; \frac{\partial \textbf{U}_h}{\partial k} = r \; \textbf{U}_h^b
\]
where $r$ is a friction coefficient expressed in ms$^{-1}$.
This coefficient is generally estimated by setting a typical decay time $\tau$ in the deep ocean, 
and setting $r = H / \tau$, where $H$ is the ocean depth.
Commonly accepted values of $\tau$ are of the order of 100 to 200 days \citep{Weatherly_JMR84}.
A value $\tau^{-1} = 10^{-7}$~s$^{-1}$ equivalent to 115 days, is usually used in quasi-geostrophic models.
One may consider the linear friction as an approximation of quadratic friction, $r \approx 2\;C_D\;U_{av}$
(\citet{Gill1982}, Eq. 9.6.6).
For example, with a drag coefficient $C_D = 0.002$, a typical speed of tidal currents of $U_{av} =0.1$~m\;s$^{-1}$,
and assuming an ocean depth $H = 4000$~m, the resulting friction coefficient is $r = 4\;10^{-4}$~m\;s$^{-1}$.
This is the default value used in \NEMO. It corresponds to a decay time scale of 115~days.
It can be changed by specifying \np{rn\_bfri1} (namelist parameter).

For the linear friction case the coefficients defined in the general expression \autoref{eq:zdfbfr_bdef} are: 
\[
  % \label{eq:zdfbfr_linbfr_b}
  \begin{split}
    c_b^u &= - r\\
    c_b^v &= - r\\
  \end{split}
\]
When \np{nn\_botfr}\forcode{ = 1}, the value of $r$ used is \np{rn\_bfri1}.
Setting \np{nn\_botfr}\forcode{ = 0} is equivalent to setting $r=0$ and
leads to a free-slip bottom boundary condition.
These values are assigned in \mdl{zdfbfr}.
From v3.2 onwards there is support for local enhancement of these values via an externally defined 2D mask array
(\np{ln\_bfr2d}\forcode{ = .true.}) given in the \ifile{bfr\_coef} input NetCDF file.
The mask values should vary from 0 to 1.
Locations with a non-zero mask value will have the friction coefficient increased by
$mask\_value$*\np{rn\_bfrien}*\np{rn\_bfri1}.

% -------------------------------------------------------------------------------------------------------------
%       Non-Linear Bottom Friction
% -------------------------------------------------------------------------------------------------------------
\subsection{Non-linear bottom friction (\protect\np{nn\_botfr}\forcode{ = 2})}
\label{subsec:ZDF_bfr_nonlinear}

The non-linear bottom friction parameterisation assumes that the bottom friction is quadratic: 
\[
  % \label{eq:zdfbfr_nonlinear}
  {\cal F}_h^\textbf{U} = \frac{A^{vm}}{e_3 }\frac{\partial \textbf {U}_h
  }{\partial k}=C_D \;\sqrt {u_b ^2+v_b ^2+e_b } \;\; \textbf {U}_h^b
\]
where $C_D$ is a drag coefficient, and $e_b $ a bottom turbulent kinetic energy due to tides,
internal waves breaking and other short time scale currents.
A typical value of the drag coefficient is $C_D = 10^{-3} $.
As an example, the CME experiment \citep{Treguier_JGR92} uses $C_D = 10^{-3}$ and
$e_b = 2.5\;10^{-3}$m$^2$\;s$^{-2}$, while the FRAM experiment \citep{Killworth1992} uses $C_D = 1.4\;10^{-3}$ and
$e_b =2.5\;\;10^{-3}$m$^2$\;s$^{-2}$.
The CME choices have been set as default values (\np{rn\_bfri2} and \np{rn\_bfeb2} namelist parameters).

As for the linear case, the bottom friction is imposed in the code by adding the trend due to
the bottom friction to the general momentum trend in \mdl{dynbfr}.
For the non-linear friction case the terms computed in \mdl{zdfbfr} are:
\[
  % \label{eq:zdfbfr_nonlinbfr}
  \begin{split}
    c_b^u &= - \; C_D\;\left[ u^2 + \left(\bar{\bar{v}}^{i+1,j}\right)^2 + e_b \right]^{1/2}\\
    c_b^v &= - \; C_D\;\left[  \left(\bar{\bar{u}}^{i,j+1}\right)^2 + v^2 + e_b \right]^{1/2}\\
  \end{split}
\]

The coefficients that control the strength of the non-linear bottom friction are initialised as namelist parameters:
$C_D$= \np{rn\_bfri2}, and $e_b$ =\np{rn\_bfeb2}.
Note for applications which treat tides explicitly a low or even zero value of \np{rn\_bfeb2} is recommended.
From v3.2 onwards a local enhancement of $C_D$ is possible via an externally defined 2D mask array
(\np{ln\_bfr2d}\forcode{ = .true.}).
This works in the same way as for the linear bottom friction case with non-zero masked locations increased by
$mask\_value$*\np{rn\_bfrien}*\np{rn\_bfri2}.

% -------------------------------------------------------------------------------------------------------------
%       Bottom Friction Log-layer
% -------------------------------------------------------------------------------------------------------------
\subsection[Log-layer btm frict enhncmnt (\protect\np{nn\_botfr}\forcode{ = 2}, \protect\np{ln\_loglayer}\forcode{ = .true.})]
				{Log-layer bottom friction enhancement (\protect\np{nn\_botfr}\forcode{ = 2}, \protect\np{ln\_loglayer}\forcode{ = .true.})}
\label{subsec:ZDF_bfr_loglayer}

In the non-linear bottom friction case, the drag coefficient, $C_D$, can be optionally enhanced using
a "law of the wall" scaling.
If  \np{ln\_loglayer} = .true., $C_D$ is no longer constant but is related to the thickness of
the last wet layer in each column by:
\[
  C_D = \left ( {\kappa \over {\rm log}\left ( 0.5e_{3t}/rn\_bfrz0 \right ) } \right )^2
\]

\noindent where $\kappa$ is the von-Karman constant and \np{rn\_bfrz0} is a roughness length provided via
the namelist.

For stability, the drag coefficient is bounded such that it is kept greater or equal to
the base \np{rn\_bfri2} value and it is not allowed to exceed the value of an additional namelist parameter:
\np{rn\_bfri2\_max}, \ie
\[
  rn\_bfri2 \leq C_D \leq rn\_bfri2\_max
\]

\noindent Note also that a log-layer enhancement can also be applied to the top boundary friction if
under ice-shelf cavities are in use (\np{ln\_isfcav}\forcode{ = .true.}).
In this case, the relevant namelist parameters are \np{rn\_tfrz0}, \np{rn\_tfri2} and \np{rn\_tfri2\_max}.

% -------------------------------------------------------------------------------------------------------------
%       Bottom Friction stability
% -------------------------------------------------------------------------------------------------------------
\subsection{Bottom friction stability considerations}
\label{subsec:ZDF_bfr_stability}

Some care needs to exercised over the choice of parameters to ensure that the implementation of
bottom friction does not induce numerical instability.
For the purposes of stability analysis, an approximation to \autoref{eq:zdfbfr_flux2} is:
\begin{equation}
  \label{eq:Eqn_bfrstab}
  \begin{split}
    \Delta u &= -\frac{{{\cal F}_h}^u}{e_{3u}}\;2 \rdt    \\
    &= -\frac{ru}{e_{3u}}\;2\rdt\\
  \end{split}
\end{equation}
\noindent where linear bottom friction and a leapfrog timestep have been assumed.
To ensure that the bottom friction cannot reverse the direction of flow it is necessary to have:
\[
  |\Delta u| < \;|u| 
\]
\noindent which, using \autoref{eq:Eqn_bfrstab}, gives:
\[
  r\frac{2\rdt}{e_{3u}} < 1 \qquad  \Rightarrow \qquad r < \frac{e_{3u}}{2\rdt}\\
\]
This same inequality can also be derived in the non-linear bottom friction case if
a velocity of 1 m.s$^{-1}$ is assumed.
Alternatively, this criterion can be rearranged to suggest a minimum bottom box thickness to ensure stability:
\[
  e_{3u} > 2\;r\;\rdt
\]
\noindent which it may be necessary to impose if partial steps are being used.
For example, if $|u| = 1$ m.s$^{-1}$, $rdt = 1800$ s, $r = 10^{-3}$ then $e_{3u}$ should be greater than 3.6 m.
For most applications, with physically sensible parameters these restrictions should not be of concern.
But caution may be necessary if attempts are made to locally enhance the bottom friction parameters. 
To ensure stability limits are imposed on the bottom friction coefficients both
during initialisation and at each time step.
Checks at initialisation are made in \mdl{zdfbfr} (assuming a 1 m.s$^{-1}$ velocity in the non-linear case).
The number of breaches of the stability criterion are reported as well as
the minimum and maximum values that have been set.
The criterion is also checked at each time step, using the actual velocity, in \mdl{dynbfr}.
Values of the bottom friction coefficient are reduced as necessary to ensure stability;
these changes are not reported.

Limits on the bottom friction coefficient are not imposed if the user has elected to
handle the bottom friction implicitly (see \autoref{subsec:ZDF_bfr_imp}).
The number of potential breaches of the explicit stability criterion are still reported for information purposes.

% -------------------------------------------------------------------------------------------------------------
%       Implicit Bottom Friction
% -------------------------------------------------------------------------------------------------------------
\subsection{Implicit bottom friction (\protect\np{ln\_bfrimp}\forcode{ = .true.})}
\label{subsec:ZDF_bfr_imp}

An optional implicit form of bottom friction has been implemented to improve model stability.
We recommend this option for shelf sea and coastal ocean applications, especially for split-explicit time splitting.
This option can be invoked by setting \np{ln\_bfrimp} to \forcode{.true.} in the \textit{nambfr} namelist.
This option requires \np{ln\_zdfexp} to be \forcode{.false.} in the \textit{namzdf} namelist. 

This implementation is realised in \mdl{dynzdf\_imp} and \mdl{dynspg\_ts}. In \mdl{dynzdf\_imp},
the bottom boundary condition is implemented implicitly.

\[
  % \label{eq:dynzdf_bfr}
  \left.{\left( {\frac{A^{vm} }{e_3 }\ \frac{\partial \textbf{U}_h}{\partial k}} \right)} \right|_{mbk}
  = \binom{c_{b}^{u}u^{n+1}_{mbk}}{c_{b}^{v}v^{n+1}_{mbk}}
\]

where $mbk$ is the layer number of the bottom wet layer.
Superscript $n+1$ means the velocity used in the friction formula is to be calculated, so, it is implicit.

If split-explicit time splitting is used, care must be taken to avoid the double counting of the bottom friction in
the 2-D barotropic momentum equations.
As NEMO only updates the barotropic pressure gradient and Coriolis' forcing terms in the 2-D barotropic calculation,
we need to remove the bottom friction induced by these two terms which has been included in the 3-D momentum trend 
and update it with the latest value.
On the other hand, the bottom friction contributed by the other terms
(\eg the advection term, viscosity term) has been included in the 3-D momentum equations and
should not be added in the 2-D barotropic mode.

The implementation of the implicit bottom friction in \mdl{dynspg\_ts} is done in two steps as the following:

\[
  % \label{eq:dynspg_ts_bfr1}
  \frac{\textbf{U}_{med}-\textbf{U}^{m-1}}{2\Delta t}=-g\nabla\eta-f\textbf{k}\times\textbf{U}^{m}+c_{b}
  \left(\textbf{U}_{med}-\textbf{U}^{m-1}\right)
\]
\[
  \frac{\textbf{U}^{m+1}-\textbf{U}_{med}}{2\Delta t}=\textbf{T}+
  \left(g\nabla\eta^{'}+f\textbf{k}\times\textbf{U}^{'}\right)-
  2\Delta t_{bc}c_{b}\left(g\nabla\eta^{'}+f\textbf{k}\times\textbf{u}_{b}\right)
\]

where $\textbf{T}$ is the vertical integrated 3-D momentum trend.
We assume the leap-frog time-stepping is used here.
$\Delta t$ is the barotropic mode time step and $\Delta t_{bc}$ is the baroclinic mode time step.
$c_{b}$ is the friction coefficient.
$\eta$ is the sea surface level calculated in the barotropic loops while $\eta^{'}$ is the sea surface level used in
the 3-D baroclinic mode.
$\textbf{u}_{b}$ is the bottom layer horizontal velocity.

% -------------------------------------------------------------------------------------------------------------
%       Bottom Friction with split-explicit time splitting
% -------------------------------------------------------------------------------------------------------------
\subsection[Bottom friction w/ split-explicit time splitting (\protect\np{ln\_bfrimp})]
				{Bottom friction with split-explicit time splitting (\protect\np{ln\_bfrimp})}
\label{subsec:ZDF_bfr_ts}

When calculating the momentum trend due to bottom friction in \mdl{dynbfr},
the bottom velocity at the before time step is used.
This velocity includes both the baroclinic and barotropic components which is appropriate when
using either the explicit or filtered surface pressure gradient algorithms
(\key{dynspg\_exp} or \key{dynspg\_flt}).
Extra attention is required, however, when using split-explicit time stepping (\key{dynspg\_ts}).
In this case the free surface equation is solved with a small time step \np{rn\_rdt}/\np{nn\_baro},
while the three dimensional prognostic variables are solved with the longer time step of \np{rn\_rdt} seconds.
The trend in the barotropic momentum due to bottom friction appropriate to this method is that given by
the selected parameterisation (\ie linear or non-linear bottom friction) computed with
the evolving velocities at each barotropic timestep. 

In the case of non-linear bottom friction, we have elected to partially linearise the problem by
keeping the coefficients fixed throughout the barotropic time-stepping to those computed in
\mdl{zdfbfr} using the now timestep.
This decision allows an efficient use of the $c_b^{\vect{U}}$ coefficients to:

\begin{enumerate}
\item On entry to \rou{dyn\_spg\_ts}, remove the contribution of the before barotropic velocity to
  the bottom friction component of the vertically integrated momentum trend.
  Note the same stability check that is carried out on the bottom friction coefficient in \mdl{dynbfr} has to
  be applied here to ensure that the trend removed matches that which was added in \mdl{dynbfr}.
\item At each barotropic step, compute the contribution of the current barotropic velocity to
  the trend due to bottom friction.
  Add this contribution to the vertically integrated momentum trend.
  This contribution is handled implicitly which eliminates the need to impose a stability criteria on
  the values of the bottom friction coefficient within the barotropic loop. 
\end{enumerate}

Note that the use of an implicit formulation within the barotropic loop for the bottom friction trend means that
any limiting of the bottom friction coefficient in \mdl{dynbfr} does not adversely affect the solution when
using split-explicit time splitting.
This is because the major contribution to bottom friction is likely to come from the barotropic component which
uses the unrestricted value of the coefficient.
However, if the limiting is thought to be having a major effect
(a more likely prospect in coastal and shelf seas applications) then
the fully implicit form of the bottom friction should be used (see \autoref{subsec:ZDF_bfr_imp})
which can be selected by setting \np{ln\_bfrimp} $=$ \forcode{.true.}.

Otherwise, the implicit formulation takes the form:
\[
  % \label{eq:zdfbfr_implicitts}
  \bar{U}^{t+ \rdt} = \; \left [ \bar{U}^{t-\rdt}\; + 2 \rdt\;RHS \right ] / \left [ 1 - 2 \rdt \;c_b^{u} / H_e \right ]
\]
where $\bar U$ is the barotropic velocity, $H_e$ is the full depth (including sea surface height), 
$c_b^u$ is the bottom friction coefficient as calculated in \rou{zdf\_bfr} and
$RHS$ represents all the components to the vertically integrated momentum trend except for
that due to bottom friction.

% ================================================================
% Tidal Mixing
% ================================================================
\section{Tidal mixing (\protect\key{zdftmx})}
\label{sec:ZDF_tmx}

%--------------------------------------------namzdf_tmx--------------------------------------------------
%
%\nlst{namzdf_tmx}
%--------------------------------------------------------------------------------------------------------------


% -------------------------------------------------------------------------------------------------------------
%        Bottom intensified tidal mixing 
% -------------------------------------------------------------------------------------------------------------
\subsection{Bottom intensified tidal mixing}
\label{subsec:ZDF_tmx_bottom}

Options are defined through the  \ngn{namzdf\_tmx} namelist variables.
The parameterization of tidal mixing follows the general formulation for the vertical eddy diffusivity proposed by
\citet{St_Laurent_al_GRL02} and first introduced in an OGCM by \citep{Simmons_al_OM04}. 
In this formulation an additional vertical diffusivity resulting from internal tide breaking,
$A^{vT}_{tides}$ is expressed as a function of $E(x,y)$,
the energy transfer from barotropic tides to baroclinic tides:
\begin{equation}
  \label{eq:Ktides}
  A^{vT}_{tides} =  q \,\Gamma \,\frac{ E(x,y) \, F(z) }{ \rho \, N^2 }
\end{equation}
where $\Gamma$ is the mixing efficiency, $N$ the Brunt-Vais\"{a}l\"{a} frequency (see \autoref{subsec:TRA_bn2}),
$\rho$ the density, $q$ the tidal dissipation efficiency, and $F(z)$ the vertical structure function. 

The mixing efficiency of turbulence is set by $\Gamma$ (\np{rn\_me} namelist parameter) and
is usually taken to be the canonical value of $\Gamma = 0.2$ (Osborn 1980). 
The tidal dissipation efficiency is given by the parameter $q$ (\np{rn\_tfe} namelist parameter)
represents the part of the internal wave energy flux $E(x, y)$ that is dissipated locally,
with the remaining $1-q$ radiating away as low mode internal waves and
contributing to the background internal wave field.
A value of $q=1/3$ is typically used \citet{St_Laurent_al_GRL02}.
The vertical structure function $F(z)$ models the distribution of the turbulent mixing in the vertical.
It is implemented as a simple exponential decaying upward away from the bottom,
with a vertical scale of $h_o$ (\np{rn\_htmx} namelist parameter,
with a typical value of $500\,m$) \citep{St_Laurent_Nash_DSR04}, 
\[
  % \label{eq:Fz}
  F(i,j,k) = \frac{ e^{ -\frac{H+z}{h_o} } }{ h_o \left( 1- e^{ -\frac{H}{h_o} } \right) }
\]
and is normalized so that vertical integral over the water column is unity. 

The associated vertical viscosity is calculated from the vertical diffusivity assuming a Prandtl number of 1,
\ie $A^{vm}_{tides}=A^{vT}_{tides}$.
In the limit of $N \rightarrow 0$ (or becoming negative), the vertical diffusivity is capped at $300\,cm^2/s$ and
impose a lower limit on $N^2$ of \np{rn\_n2min} usually set to $10^{-8} s^{-2}$.
These bounds are usually rarely encountered.

The internal wave energy map, $E(x, y)$ in \autoref{eq:Ktides}, is derived from a barotropic model of
the tides utilizing a parameterization of the conversion of barotropic tidal energy into internal waves.
The essential goal of the parameterization is to represent the momentum exchange between the barotropic tides and
the unrepresented internal waves induced by the tidal flow over rough topography in a stratified ocean.
In the current version of \NEMO, the map is built from the output of
the barotropic global ocean tide model MOG2D-G \citep{Carrere_Lyard_GRL03}.
This model provides the dissipation associated with internal wave energy for the M2 and K1 tides component
(\autoref{fig:ZDF_M2_K1_tmx}).
The S2 dissipation is simply approximated as being $1/4$ of the M2 one.
The internal wave energy is thus : $E(x, y) = 1.25 E_{M2} + E_{K1}$.
Its global mean value is $1.1$ TW,
in agreement with independent estimates \citep{Egbert_Ray_Nat00, Egbert_Ray_JGR01}. 

%>>>>>>>>>>>>>>>>>>>>>>>>>>>>
\begin{figure}[!t]
  \begin{center}
    \includegraphics[width=0.90\textwidth]{Fig_ZDF_M2_K1_tmx}
    \caption{
      \protect\label{fig:ZDF_M2_K1_tmx}
      (a) M2 and (b) K1 internal wave drag energy from \citet{Carrere_Lyard_GRL03} ($W/m^2$).
    }
  \end{center}
\end{figure}
%>>>>>>>>>>>>>>>>>>>>>>>>>>>> 
 
% -------------------------------------------------------------------------------------------------------------
%        Indonesian area specific treatment 
% -------------------------------------------------------------------------------------------------------------
\subsection{Indonesian area specific treatment (\protect\np{ln\_zdftmx\_itf})}
\label{subsec:ZDF_tmx_itf}

When the Indonesian Through Flow (ITF) area is included in the model domain,
a specific treatment of tidal induced mixing in this area can be used.
It is activated through the namelist logical \np{ln\_tmx\_itf}, and the user must provide an input NetCDF file,
\ifile{mask\_itf}, which contains a mask array defining the ITF area where the specific treatment is applied.

When \np{ln\_tmx\_itf}\forcode{ = .true.}, the two key parameters $q$ and $F(z)$ are adjusted following
the parameterisation developed by \citet{Koch-Larrouy_al_GRL07}:

First, the Indonesian archipelago is a complex geographic region with a series of
large, deep, semi-enclosed basins connected via numerous narrow straits.
Once generated, internal tides remain confined within this semi-enclosed area and hardly radiate away.
Therefore all the internal tides energy is consumed within this area.
So it is assumed that $q = 1$, \ie all the energy generated is available for mixing.
Note that for test purposed, the ITF tidal dissipation efficiency is a namelist parameter (\np{rn\_tfe\_itf}).
A value of $1$ or close to is this recommended for this parameter.

Second, the vertical structure function, $F(z)$, is no more associated with a bottom intensification of the mixing,
but with a maximum of energy available within the thermocline.
\citet{Koch-Larrouy_al_GRL07} have suggested that the vertical distribution of
the energy dissipation proportional to $N^2$ below the core of the thermocline and to $N$ above. 
The resulting $F(z)$ is:
\[
  % \label{eq:Fz_itf}
  F(i,j,k) \sim     \left\{
    \begin{aligned}
      \frac{q\,\Gamma E(i,j) } {\rho N \, \int N     dz}    \qquad \text{when $\partial_z N < 0$} \\
      \frac{q\,\Gamma E(i,j) } {\rho     \, \int N^2 dz}    \qquad \text{when $\partial_z N > 0$}
    \end{aligned}
  \right.
\]

Averaged over the ITF area, the resulting tidal mixing coefficient is $1.5\,cm^2/s$, 
which agrees with the independent estimates inferred from observations.
Introduced in a regional OGCM, the parameterization improves the water mass characteristics in
the different Indonesian seas, suggesting that the horizontal and vertical distributions of
the mixing are adequately prescribed \citep{Koch-Larrouy_al_GRL07, Koch-Larrouy_al_OD08a, Koch-Larrouy_al_OD08b}.
Note also that such a parameterisation has a significant impact on the behaviour of
global coupled GCMs \citep{Koch-Larrouy_al_CD10}.

% ================================================================
% Internal wave-driven mixing
% ================================================================
\section{Internal wave-driven mixing (\protect\key{zdftmx\_new})}
\label{sec:ZDF_tmx_new}

%--------------------------------------------namzdf_tmx_new------------------------------------------
%
%\nlst{namzdf_tmx_new}
%--------------------------------------------------------------------------------------------------------------

The parameterization of mixing induced by breaking internal waves is a generalization of
the approach originally proposed by \citet{St_Laurent_al_GRL02}.
A three-dimensional field of internal wave energy dissipation $\epsilon(x,y,z)$ is first constructed,
and the resulting diffusivity is obtained as 
\[
  % \label{eq:Kwave}
  A^{vT}_{wave} =  R_f \,\frac{ \epsilon }{ \rho \, N^2 }
\]
where $R_f$ is the mixing efficiency and $\epsilon$ is a specified three dimensional distribution of
the energy available for mixing.
If the \np{ln\_mevar} namelist parameter is set to false, the mixing efficiency is taken as constant and
equal to 1/6 \citep{Osborn_JPO80}.
In the opposite (recommended) case, $R_f$ is instead a function of
the turbulence intensity parameter $Re_b = \frac{ \epsilon}{\nu \, N^2}$,
with $\nu$ the molecular viscosity of seawater, following the model of \cite{Bouffard_Boegman_DAO2013} and
the implementation of \cite{de_lavergne_JPO2016_efficiency}.
Note that $A^{vT}_{wave}$ is bounded by $10^{-2}\,m^2/s$, a limit that is often reached when
the mixing efficiency is constant.

In addition to the mixing efficiency, the ratio of salt to heat diffusivities can chosen to vary 
as a function of $Re_b$ by setting the \np{ln\_tsdiff} parameter to true, a recommended choice. 
This parameterization of differential mixing, due to \cite{Jackson_Rehmann_JPO2014},
is implemented as in \cite{de_lavergne_JPO2016_efficiency}.

The three-dimensional distribution of the energy available for mixing, $\epsilon(i,j,k)$,
is constructed from three static maps of column-integrated internal wave energy dissipation,
$E_{cri}(i,j)$, $E_{pyc}(i,j)$, and $E_{bot}(i,j)$, combined to three corresponding vertical structures
(de Lavergne et al., in prep):
\begin{align*}
  F_{cri}(i,j,k) &\propto e^{-h_{ab} / h_{cri} }\\
  F_{pyc}(i,j,k) &\propto N^{n\_p}\\
  F_{bot}(i,j,k) &\propto N^2 \, e^{- h_{wkb} / h_{bot} }
\end{align*} 
In the above formula, $h_{ab}$ denotes the height above bottom,
$h_{wkb}$ denotes the WKB-stretched height above bottom, defined by
\[
  h_{wkb} = H \, \frac{ \int_{-H}^{z} N \, dz' } { \int_{-H}^{\eta} N \, dz'  } \; ,
\]
The $n_p$ parameter (given by \np{nn\_zpyc} in \ngn{namzdf\_tmx\_new} namelist)
controls the stratification-dependence of the pycnocline-intensified dissipation.
It can take values of 1 (recommended) or 2.
Finally, the vertical structures $F_{cri}$ and $F_{bot}$ require the specification of
the decay scales $h_{cri}(i,j)$ and $h_{bot}(i,j)$, which are defined by two additional input maps.
$h_{cri}$ is related to the large-scale topography of the ocean (etopo2) and
$h_{bot}$ is a function of the energy flux $E_{bot}$, the characteristic horizontal scale of
the abyssal hill topography \citep{Goff_JGR2010} and the latitude.

% ================================================================

\biblio

\pindex

\end{document}
